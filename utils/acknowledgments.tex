\chapter*{Acknowledgments}
\label{ch:acknowledgments}
\addcontentsline{toc}{chapter}{\nameref*{ch:acknowledgments}}
\markboth{\nameref*{ch:acknowledgments}}{\nameref*{ch:acknowledgments}}

% 1 Acknowledgments
% Recognize those who contributed to your research.
% How did your supervisors and others contribute?
% Do you need to thank those who provided help during data-collection, including participants?
% Did anyone provide pastoral support, including friends and family?
% Did you have any funding bodies?
% Were any professionals involved, including typists, transcribers, or proofreaders?

First, I would like to thank my supervisor Prof. Chu-Ren Huang for his guidance and all the support and kindness he gave me in these three years, ever since my first email. He not only believed in my topic, but he taught me new ways to think about words, meanings, concepts, and language. I would also like to thank PolyU and the department for equipping me with a workspace and for the never-ending answers to all my questions regarding the PhD program. I want to praise the Pao Yue-Kong Library and its staff for supplying us with space, workshops, a wide range of books and materials, and their willingness to grant requests so promptly whenever we suggest a loan or purchase. The lovely people running the LibCafé also deserve a mention. Next, I would like to express my gratitude towards the two external examiners, Prof. Victor H. Mair and Prof. Baoya Chen, who took their time to read this work, gave invaluable comments, and encouraged me to continue on this path. I must also thank the Research Grants Council (RGC) of the University Grants Committee (UGC) of Hong Kong, who funded this PhD program under the Hong Kong Ph.D. Fellowship Scheme (HKPFS, ref. no. PF18-22990). Lastly, I want to thank the people of Hong Kong, who welcomed me to their intriguing city.

\medskip

As probably this is the end of my \textit{official} journey as a student---\textit{official} because in reality we all know this road never ends---I want to thank all my teachers and professors from whom I learned so much throughout these years, especially at my alma mater the Eötvös Loránd University (ELTE) in Budapest. The list is so very long, but I must definitely mention Tamás Iványi, Zoltán Szombathy, Dóra Zsom, Sándor Fodor, István Ormos, István Hajnal, Saber el-Adly, Ágnes Birtalan, Mária Négyesi, Rama Yadav, Csaba Dezső, Bo-Guk Kim, Beatrix Mecsi, Mózes Csoma, Valéria Simon, Katalin Torma, Orsolya Saraç, and Panda Stojanovska. I am grateful for all the knowledge and insight I received from everyone. 

\medskip

Most importantly, I want to thank my friends in Hong Kong, who helped me all the way, especially Yun and Andreas, and my parents back home, who have always supported this life of study with patience.





% Fendler Károly, Péri Benedek, Alexa Péter

% Baráth Zoltán, Piriti János