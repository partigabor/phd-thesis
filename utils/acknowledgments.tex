\chapter*{Acknowledgments}
\label{ch:acknowledgments}
\addcontentsline{toc}{chapter}{\nameref*{ch:acknowledgments}}
\markboth{\nameref*{ch:acknowledgments}}{\nameref*{ch:acknowledgments}}

% 1 Acknowledgments
% Recognize those who contributed to your research.
% How did your supervisors and others contribute?
% Do you need to thank those who provided help during data-collection, including participants?
% Did anyone provide pastoral support, including friends and family?
% Did you have any funding bodies?
% Were any professionals involved, including typists, transcribers, or proofreaders?

First, I would like to thank my supervisor Prof. Chu-Ren Huang for his guidance and all the support and kindness he gave me in these three years, ever since my first email. He not only believed in my topic, but he taught me new ways to think about words, meanings, concepts, and language. I would also like to thank PolyU and the department of Chinese and Bilingual Studies (CBS) who equipped me with a workspace and never-ending answers to all my questions regarding the PhD program. I want to highlight the Pao Yue-Kong Library and its staff supplying us with space, workshops, a wide range of books and materials, who can grant requests so promptly whenever we suggest a purchase; truly helpful. I also cannot leave out the lovely people who run the LibCafé. Next, I would like to express my gratitude to my two external examiners, Prof. Victor H. Mair and Prof. Baoya Chen, who took their time to read this work, gave invaluable comments, and encouraged me to continue on this path. I must also thank the Research Grants Council (RGC) of the University Grants Committee (UGC) of Hong Kong, who funded this PhD program under the Hong Kong Ph.D. Fellowship Scheme (HKPFS, ref. no. PF18-22990). Lastly, I want to thank the people of Hong Kong, who welcomed me to their intriguing city.

As probably this is the end of my \textit{official} journey as a student---\textit{official} because in reality we all know this path never ends---I want to thank all my teachers and professors from whom I learned so much throughout these years at my alma mater, the Eötvös Loránd University (ELTE) in Budapest. Especially ?? the list is so long. I am grateful for all the knowledge and insight I received from everyone.

Most importantly, I want to thank my friends here who helped me all the way, especially Yun and Andreas, and of course my parents back home, who have always supported this life of study with patience.

% Tamás Iványi, Zoltán Szombathy, Sándor Fodor, Dóra Zsom, István Ormos, István Hajnal, El-Adly Saber, Rashid Daher, Ágnes Birtalan, Mária Négyesi, Csaba Dezső, Rama Yadav, 

% Beatrix Mecsi, Mózes Csoma, Fendler Károly, Kim Bo-Guk, Valéria Simon, András Bereczki, Kata Torma, Péri Benedek, Orsolya Saraç, Alexa Péter