\chapter*{Abstract}
\label{ch:abstract}
\addcontentsline{toc}{chapter}{\nameref*{ch:abstract}}
\markboth{\nameref*{ch:abstract}}{\nameref*{ch:abstract}}

% Abstract (consisting of a summary of the work done with 200-500 words)

% 2 Abstract
% Describes what happened during the research.
% What is the reason for writing the thesis?
% What are the current approaches and gaps in the literature?
% What are your research question(s) and aims?
% Which methodology have you used?
% What are the main findings?
% What are the main conclusions and implications?

\leftskip1.2cm\relax
\rightskip1.2cm\relax

The majority of existing literature on spices is found in the areas of gastronomy, botany, and history. This study investigates spices on a linguistic level and aims to be a comprehensive linguistic account on the items of the spice trade. Some of these dried plant matter were highly desired at certain points in history, due to their attractive aroma and medicinal value, thus they were ideal products of trade early on. Cultural contact and exchange, and the introduction of new cultural items begets situations of language contact and linguistic acculturation, and so in the case of spices, we not only have a set of items that traveled around the world, but also a set of names. This domain is very rich in loanwords and \textit{Wanderwörter}, but also supplies us with a myriad of cases where spice names are conventional innovations. To make it more interesting, the thesis compares English, Arabic, and Chinese, languages that represent major powers in the spice trade at different times. After selecting a set of 24 spices, I have collected data on their names and related etymologies, and introduced 6 of them in detail regarding their identity, botany, history, spread, and names. The thesis has two main parts. Part one represents the geographic and linguistic diffusion of spices and their names. Basically, I track and explain word origins and subsequent spread by tracing the materials and the propagation of the accompanying \textit{Wanderwort}. This part relies on philological literature, and tools from historical linguistics, such as etymological research, as well as geospatial visualizations. Part two examines the language of spices, referring to the terminology and nomenclature related to the spice domain from linguistic-cognitive perspectives. Focusing on the structure and components of 360 collected spice names, it is a systematic investigation on how humans name spices: what are the mechanism and motivations behind the naming principles, and how this possibly relates to the salient sensory features of the products (strong gustatory, olfactory, or visual stimuli). Conclusions are made on the connections between the physical properties of the spices, their patterns of diffusion, and the prototypical spices and their effect of naming principles. Besides being a novel and original approach to research and categorize spices from a linguistic point of view, this study offers new insights to our knowledge about wandering loanwords, and the effect of the highly sensory nature of spices in the naming process when adopted by a community. It is also intended to be a basis for a useful working database for future research, and aims to dispel some of the chaos and confusion surrounding spice names.

\leftskip0cm\relax
\rightskip0cm\relax