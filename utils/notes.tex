
% A Wörter und Sachen ('Szavak és dolgok') a Hugo Schuchardt által kidolgozott, a nyelvészetre és a néprajzra mély hatással lévő módszer, a tárgyak és szavak kölcsönös kutatásának elve, eredetileg R. Meringer indoeurópai nyelvész 1909-ben indított folyóiratának címe.
% A módszer lényege: a szavak vándorlásában (kölcsönszavak) nem elegendő pusztán csak a megfelelő korban működő és ható hangtörvényekre és esetleg a szó jelentésére támaszkodni, hanem a nyelvi tények tanulmányozásán kívül vizsgálni kell a szavaktól jelölt dolgokat és tárgyakat. Természetesen itt számolni kell még művelődéstörténeti tényezőkkel is.
% A módszert Ferdinand Blumentritt, valamint Jankó János és Bátky Zsigmond is sikeresen használta fel. 

% Wörter und Sachen

% R. Meringer indoeurópai nyelvész 1909-ben megindított folyóirata, amelyben ő és munkatársai azt a módszert képviselték, hogy a szavak jelentésfejlődésének és eredetének kutatásánál a néprajz – a tárgyak, jelenségek, intézmények, szokások vizsgálatánál pedig a nyelvészet (elsősorban a szótörténet, etimológia) eredményeit figyelembe kell venni. A néprajzi tárgykutatás jelentőségét emelte ki Meringer akkor, amikor 1906-ban azt írta, hogy „Ohne Sachwissenschaft keine Sprachwissenschaft mehr!” (’Tárgyi tudomány nélkül nincs többé nyelvtudomány’). A Wörter und Sachen a tárgyak és szavak kölcsönös kutatásának elvét Jankó János és Bátky Zsigmond eredetmagyarázataiknál (halászat, építkezés) körültekintően alkalkazták. A Wörter und Sachen módszerével tanulságos eredményeket ért el újabban Balassa Iván: A magyar kukorica (Bp., 1960). A Wörter und Sachen módszere különösen fejlett a finn (U. T. Sirelius, K. Vilkuna, N. Valonen, T. Vuorela) és az észt (F. Linnus, G. Ränk, A. Viires) etnográfusok körében. A Wörter und Sachen módszerével dolgozta fel F. Krüger a Pireneusok néprajzát. – Irod. Meringer, R.: Indogermanische Forschungen (Wien, 1906).