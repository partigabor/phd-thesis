% Glossary
\newglossaryentry{latex}{name=latex,description={is a markup language specially suited for scientific documents}}

\newglossaryentry{wanderwort}{name=Wanderwort,plural={Wanderwörter},description={a word borrowed from one language to another across a broad geographical area often as a result of trade or adoption of newly introduced items or cultural practices\footcite[]{mw}}}

\newglossaryentry{materia medica}{name=materia medica,description={an encyclopedic treatise of medicinal substances of the plant, animal, and mineral kingdoms}}

\newglossaryentry{bencao}{name=bencao, description={本草, the Chinese term for \textit{materia medica}, books that record the sources and applications of medicinal materials\footcite[]{zhao_concise_2018}}}

\newglossaryentry{pharmacopeia}{name=pharmacopeia, description={...}}

\newglossaryentry{Glottolog}{name={Glottolog},description={comprehensive reference information for the world's languages, especially the lesser known languages}}

\newglossaryentry{cultigen}{name=cultigen,description={is a cultivated plant species or variety of which no wild ancestor has been identified; (Botany)}} %OED
\newglossaryentry{cultivar}{name=cultivar,description={is a plant variety that has arisen or persists only in cultivation; abbreviated cv.; (Botany)}} %OED
\newglossaryentry{taxon}{name=taxon,plural={taxa},description={taxonomic group or unit, esp. when its rank in the taxonomic hierarchy is not specified (back-formation from `taxonomy', it refers to a group or species as a unit in biology.)}} %OED

% \newglossaryentry{schizocarp}{name={schizocarp},description={is a dry compound fruit which splits into two or more one-seeded carpels (mericarps) without dehiscing; (Botany)}} 
\newglossaryentry{umbel}{name={umbel},description={is a flower cluster in which stalks of nearly equal length spring from a common center and form a flat or curved surface, characteristic of the parsley family; (Botany)}} %OL

\newglossaryentry{wiktionary}{name=Wiktionary,description={...}}
\newglossaryentry{glottolog}{name=Glottolog,description={...}}
\newglossaryentry{phytonym}{name=phytonym, description={...}}
\newglossaryentry{pharmacopoeia}{name=pharmacopoeia, description={a pharmaceutical encyclopedia}}
\newglossaryentry{Ayurveda}{name={Ayurveda}, description={traditional Indian medicine}}


% \newglossaryentry{prakrit}{name=Prakrit,description={The Prakrits were a group of Middle Indo-Aryan language vernaculars spoken between about 500 B.C. and 500 A.D. Prakrit (prākṛta) means ‘derived’, a name contrasting with Sanskrit (saṃskṛta) ‘complete, perfected’, reflecting the fact that the Prakrit languages were considered historically secondary to, and less prestigious than, Sanskrit. (\url{https://www.orinst.ox.ac.uk/prakrit}) See more at \href{https://www.britannica.com/topic/Prakrit-languages}{Britannica}}}

% Acronyms
\newacronym{ABC}{ABC}{ABC Chinese-English Comprehensive Dictionary}
\newacronym{AHD}{AHD}{The American Heritage Dictionary of the English Language}
\newacronym{BHL}{BHL}{Biodiversity Heritage Library}
\newacronym{BT}{BT}{Bosworth Toller's An Anglo-Saxon Dictionary}
\newacronym{CAD}{CAD}{Chicago Assyrian Dictionary}
\newacronym{CBS}{CBS}{Chinese and Bilingual Studies}
\newacronym{CBETA}{CBETA}{Chinese Buddhist Electronic Texts Association}
\newacronym{CEC}{CEC}{Cambridge English-Chinese (Traditional) Dictionary}
\newacronym{CTP}{CTP}{Chinese Text Project}
\newacronym{DLE}{DLE}{Diccionario de la Lengua Española}
\newacronym{DSAL}{DSAL}{Digital South Asia Library}
\newacronym{EE}{EE}{The Concise Oxford Dictionary of English Etymology}
\newacronym{EHBC}{EHBC}{English Historical Book Collection}
\newacronym{EI2}{EI2}{Encyclopaedia of Islam, Second Edition}
\newacronym{EIr}{EIr}{Encyclopaedia Iranica}
\newacronym{EJ}{EJ}{Encyclopaedia Judaica}
\newacronym{EQ}{EQ}{Encyclopaedia of the Qur'ān }
\newacronym{FAOSTAT}{FAOSTAT}{Food and Agriculture Organization Corporate Statistical Database}
\newacronym{FOC}{FoC}{Flora of China}
\newacronym{GBIF}{GBIF}{Global Biodiversity Information Facility}
\newacronym{HW}{HW}{Hans-Wehr: A Dictionary of Modern Written Arabic}
\newacronym{LS}{LS}{Lewis \& Short: A Latin Dictionary}
\newacronym{LSJ}{LSJ}{Liddel-Scott-Jones: A Greek-English Lexicon}
\newacronym{MC}{MC}{Middle Chinese}
\newacronym{MED}{MED}{Middle English Dictionary}
\newacronym{MW}{MW}{Merriam-Webster's Unabridged Dictionary}
\newacronym{NS}{NS}{Nişanyan Sözlük - Türkçe Etimolojik Sözlük [Turkish Etymological Dictionary]}
\newacronym{OE}{OE}{Online Etymology Dictionary}
\newacronym{OED}{OED}{Oxford English Dictionary}
% \newacronym{PR}{Sheth}{Paia-sadda-mahannavo: A Comprehensive Prakrit Hindi Dictionary, with Sanskrit Equivalents, Quotations and Complete References}
% \newacronym{SR}{SR}{Monier-Williams Sanskrit-English Dictionary (Monier-Williams)}
\newacronym{SEAL}{SEAlang}{Southeast Asian Languages Library}
\newacronym{IPNI}{IPNI}{International Plant Names Index}
\newacronym{PIE}{PIE}{Proto-Indo-European}
\newacronym{POWO}{POWO}{Plants of The World Online}
\newacronym{OC}{OC}{Old Chinese}
\newacronym{OND}{OND}{Online Nahuatl Dictionary}
\newacronym{NRSV}{NRSV}{New Revised Standard Version of the Bible}
\newacronym{KJV}{KJV}{King James Version of the Bible}
\newacronym{KSUCCA}{KSUCCA}{King Saud University Corpus of Classical Arabic}
\newacronym{MDBG}{MDBG}{MDBG Chinese Dictionary}
\newacronym{PWN}{PWN}{Princeton WordNet}
\newacronym{QTS}{QTS}{Quan Tangshi 全唐詩 [Complete Tang Poems]}
\newacronym{SkE}{SkE}{Sketch Engine}
\newacronym{SS}{SS}{Scripta Sinica}
\newacronym{TCM}{TCM}{Traditional Chinese Medicine}
\newacronym{TLFi}{TLFi}{Trésor de la Langue Française informatisé}
\newacronym{TPL}{TPL}{The Plant List}
\newacronym{WALS}{WALS}{The World Atlas of Language Structures}
\newacronym{WB}{Wb}{Wörterbuch der Ägyptischen Sprache [Dictionary of the Egyptian Language]}
\newacronym{WO}{WO}{Oxford Dictionary of Word Origins}
\newacronym{WOLD}{WOLD}{The World Loanword Database}
\newacronym{WFO}{WFO}{World Flora Online}






\newacronym{TPGJ}{TPGJ}{Taiping Guangji}
\newacronym{NFCM}{NFCM}{Nanfang Caomu Zhuang}
\newacronym{YYZZ}{YYZZ}{Youyang Zazu}
\newacronym{BCGM}{BCGM}{Bencao Gangmu}




% Primary Sources
\newglossaryentry{Periplus}{name={Periplus Maris Erythraei},description={[Periplus of the Erythraean Sea] --- \nth{1} c. \AD}}

\newglossaryentry{Lisan}{type=primary, name=Lisān al-ʿArab, description={{\arabicfont {لسان العرب}} [Tongue of the Arabs] --- 1290}}
\newglossaryentry{Hawi}{type=primary, name=Kitāb al-Ḥāwī fī l-Ṭibb, description={كتاب الحاوي في الطب} [The Comprehensive Book of Medicine] by Abū Bakr al-Rāzī/Rhazes (d. 925/935)}
\newglossaryentry{Qanun}{type=primary,name={al-Qānūn fī l-Ṭibb},description={{\arabicfont{القانون في الطب}} [The Canon of Medicine] by Ibn Sīnā --- 1025.}}

\newglossaryentry{Guangyun}{type=primary, name=Guangyun, description={廣韻 [Broad Rhymes] (1008) - rhyme dictionary}}
\newglossaryentry{Kangxi}{type=primary, name=Kangxi Zidian, description={康熙字典 [Kangxi Dictionary] --- 1716}}
\newglossaryentry{Shuowen}{type=primary, name=Shuowen Jiezi, description={說文解字 [Discussing Writing and Explaining Characters] --- 100 \AD{}}}
\newglossaryentry{Liji}{type=primary, name=Liji, description={{禮記} [The Book of Rites] Warring States period 475--221 BC}}
\newglossaryentry{Bowuzhi}{type=primary, name=Bowuzhi, description={博物志} [Records of Diverse Matters] by Zhang Hua --- c. 290 AD}
\newglossaryentry{Zhouli}{type=primary, name=Zhouli, description={周禮} [Rites of Zhou] Warring States (\nth{2} c. \BC)}
\newglossaryentry{Hou Hanshu}{type=primary, name=Hou Hanshu, description={後漢書 [Book of the Later Han] \nth{5} c.}}
\newglossaryentry{Shennong}{type=primary, name=Shennong Bencaojing, description={神農本草經 [Shennong's Classic Herbal] --- ca. 200}}
\newglossaryentry{Bencao}{type=primary,name=Bencao Gangmu,description={本草綱目 [Compendium of Materia Medica] by Li Shizhen --- 1578.}}
\newglossaryentry{Youyang}{type=primary,name={Youyang Zazu},description={酉陽雜俎 [Miscellaneous Morsels from Youyang] by Duan Chengsi --- \nth{9} c.}}
\newglossaryentry{Nanfang}{type=primary,name={Nanfang Caomu Zhuang},description={南方草木狀 [Plants of the Southern Regions] --- \nth{4} c.}}
\newglossaryentry{Taiping}{type=primary,name={Taiping Guangji},description={太平廣記 [Extensive Records of the Taiping Era] --- 978}}

% \newacronym{QTS}{QTS}{Quan Tangshi 全唐詩 [Complete Tang Poems]}

\newglossaryentry{Sushruta}{type=primary, name={Suśrutasaṃhitā},description={{\devanagarifont सुश्रुतसंहिता} [Suśruta's Compendium] --- ca. 600 \AD - Sanskrit medical text, Ayurveda}}



% Symbols and notation
\glsxtrnewsymbol[description={reconstructed form}]{rec}{*}
\glsxtrnewsymbol[description={developed from}]{devf}{<}
\glsxtrnewsymbol[description={developed into}]{devi}{>}
\glsxtrnewsymbol[description={uncertain development}]{unc}{<\textsuperscript{?}}
\glsxtrnewsymbol[description={obsolete}]{obs}{\textsuperscript{†}}
\glsxtrnewsymbol[description={\textit{ante}, attested before the year}]{ante}{a.}
\glsxtrnewsymbol[description={\textit{circa}, around the year/century}]{circa}{ca.}
\glsxtrnewsymbol[description={italic: lexical item, a word or phrase}]{it}{\textit{fragrance}}
\glsxtrnewsymbol[description={square brackets: gloss, literal meaning}]{gloss}{[fragrance]}
\glsxtrnewsymbol[description={single quotation marks: meaning, sense}]{meaning}{`fragrance'}
\glsxtrnewsymbol[description={small capitals: a concept}]{concept}{\textsc{fragrance}}
% \glsxtrnewsymbol[description={consonant}]{con}{C}
% \glsxtrnewsymbol[description={vowel}]{vow}{V}
% \glsxtrnewsymbol[description={`under the word', look up a word instead of page number (sub verbo or sub voce)}]{sv}{s.v.}




