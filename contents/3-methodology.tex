% 6 Methodology
% Who what why when and how?
% What did you do to achieve the research aims?
% Why did you choose this particular approach over others?
% How does it relate to your epistemological and ontological positions?
% What tools did you use to collect data and why? What are the implications?
% When did you collect data, and from whom?
% What tools have you used to analyze the data and why? What are the implications?
% Are there ethical considerations to take into account?

% Bottom-up

\chapter{Methodology}
\label{ch:methodology}

% Who what why when and how?

\lettrine[lines=\iniciale]{\textcolor{\accentcolor}{R}}{ealizing} that there is little work done on building a spice name database, or on analyzing spice nomenclature from historical and linguistic-cognitive perspectives, I have set out to assemble one that would facilitate this kind of analysis. 
To introduce very briefly, I have built a database of spices and spice terminology by combing through secondary and primary literature, botanical databases, encyclopedias and dictionaries, and searching for the spices in contemporary and historical corpora. I then used a few selected features of these materials (region of origin, spreadability, etc.) and the corresponding terms (analyzability, borrowed status, etc.) and looked at the set of spices as a whole, trying to find patterns and make some observations about the geographic and linguistic diffusion of spices, and various aspects of their naming.

% What did you do to achieve the research aims?

\section{Research Design Principles}

% Therefore, the dataset accompanying this thesis is to be grounded in the following principles: (1) correct botanical identification of a plant and the obtained substance; (2) awareness of the substance's physical and botanical properties; (3) knowledge about the substance's origin, spread, history, uses, and cultural/religious significance; (3) the most possible complete collection of various names denoting the substance in the literature, including pre-modern periods; (4) enabling reproducability, review and improvement by citing sources and references.

To achieve these aims, I first needed think of an ideal way to compile and arrange complex sets of information, from sources that are highly interdisciplinary in nature. From the very beginning of the design of this study, the following principles were kept in mind regarding the database of spice names:

(1) The database must be grounded in the close study of the materials---the plants and their products---especially from a historical and botanical standpoint. Awareness of the material's physical journey will help us to contextualize some of the ways the associated names spread. Take for example the Sanskrit term referring to asafoetida (the dried oleoresin gum from \textit{Ferula assa-foetida} et al.): \sa{हिङ्गु} \textit{hiṅgu}, which is the etymon of both Chinese \tc{興蕖} \textit{xīngqú} (\gls{MC} /hɨŋ ɡɨʌ/),
% \tc{形虞} \textit{xíngyú} (\gls{MC} /ɦeŋ  ŋɨo/)
and English \textit{hing}, but they took very different paths: while the Chinese term is a learned loan from during the spread of Buddhist scriptures on the overland Silk Road, the English word is a late \nth{16}-century borrowing via the sea trade with Mughal India. And if we study the source of the materials and learn about the plants, we will also realize that all the asafoetida that was exported from India in the early modern period was in fact imported from Persia and Afghanistan.

(2) The database must be thoroughly cited; every word, statement, date, or other piece of information should be carefully referenced. I have already explained the motivation and necessity behind this practice in \cref{ch:introduction}, it is enough to say that currently no one is citing sources for the names they give (except philologists), and sometimes it hard to find the motivation and inspiration behind a term. It is always a good scholarly practice to record where we found certain pieces of information, and when it comes to spice etymologies, this should make it easier for future experts to verify or refute the findings.
% Example?

(3) The database should be easily expandable. Because of the limited time, it is impossible for me to include \emph{every} spice. Therefore, I tried to create a pipeline, where a new material and its names can be easily added to the fold, and quickly analyzed. This in principle can also accommodate for the future inclusion of incense, perfume, and herbs, which I will mention in \cref{sec:future_studies} when discussing future plans to expand on this research.

\subsection{Identification, Confusion, Adulteration, Clarification}

The ideal first step of all types of research related to spices, herbs, incense, and other aromatics is to identify the product exactly. In the case of spices and incense, this is overwhelmingly a botanical question, while in the case other exotic aromatics, such as musk or ambergris, we must involve the animal kingdom. Medical, pharmaceutical, and food industry studies are heavy on the hard sciences---chemistry, biology---but they sometimes also contain valuable information about both common and scientific names. All medical studies must start with the proper identification of the substance, in fact, there is a range of studies about various techniques on identification and differentiation \autocite[cf.][]{ford_cinnamon_2019}. The reasons for this are twofold. 

Firstly, in many cases it is not a straightforward task to tell the substances apart, different spices can have very similar physical qualities. E.g., the fruits of Chinese star anise (\textit{Illicium verum}) and Japanese star anise (\textit{Illicium anisatum}) basically look the same, but the latter is toxic; see the excellent points made by \textcite{small_confusion_1996} on the confusion of their common names. Uncertainty in nomenclature and identity poses a further challenge to clinical trials if the origins of a substance is not properly identified. Take for example \textcite{oketch-rabah_cinnamon_2018}, who writes on the confusion of cinnamon and cassia nomenclature and its implications in pharmaceutical research. Consider first cinnamon (\textit{Cinnamomum verum} syn. \textit{C. zeylanicum}). Common names include \textit{true cinnamon} and \textit{Ceylon cinnamon}. However, the cinnamon sold in the US and in the UK markets are generally not the same spice: most of the product labelled as cinnamon on American shelves is in fact cassia (\textit{Cinnamomum cassia} syn. \textit{C. aromaticum}) \autocite{oketch-rabah_cinnamon_2018}, which is sometimes called `fake cinnamon' or `bastard cinnamon'. In retrospect, the Latin scientific name of the former makes bit more sense now: \textit{verum} means `real, true, genuine'. But why is cassia fake cinnamon? This is due to historical reasons, from when the introduction of the much cheaper cassia pushed down the cinnamon prices drastically in the \nth{19} century \autocite{wijesekera_chemistry_1978}. Most scholars consistently refer to \textit{C. zeylanicum} as cinnamon, and to \textit{C. cassia} as cassia but it is not uncommon in everyday language use to confound the two, especially in referring to cassia as cinnamon, out of innocent ignorance. For more detail and on the identity of cinnamon and cassia please see \cref{sec:identity_cinnamon}. Uncovering confusions from under heaps of synonyms lead us to interesting historical events that sometimes explain the vernacular names of a particular product, such as the case of cinnamon and cassia shows. 

Secondly, adulteration and contamination are rampant in the industry. Saffron (\textit{Crocus sativus})---the most expensive spice by weight nearing the price of gold---is famous for being knocked up (and substituted) with the much cheaper safflower petals (\textit{Carthamus tinctorius}). Even their names reflect these practices: although the two are very different and unrelated plants, their similar dyeing properties and constant confusion have left its mark. \textit{Safflower} has been influenced by the French word for saffron, but if they have different origins (both ultimately from Arabic). And, on account of the adulteration, safflower have also come to be known as \textit{bastard saffron}, first attested in 1548. 

We do not need to lurk modern pharmacological studies to find examples of confusion, the identity of saffron was also elusive in ancient China, where at its introduction in the early Middle Ages, it was confused with safflower, and both were casually called \tc{紅花} \textit{honghua}. It is said that Buddhist monks picked up saffron in Kashmir on their way from India to China, but the knowledge about it was not clear until the Yuan dynasty, when it was actually used and imported \autocite{laufer_sino-iranica_1919}. During Tang times, it was connected with the---also strongly yellow---turmeric. Turmeric came first, and got the name \tc{鬱金} \textit{yujin} [yü-gold], and later saffron was named \tc{鬱金香} \textit{yujinxiang} [yü-gold-aromatic] \autocite{schafer_golden_1985}. The confusion of saffron and turmeric (and truthfully every other yellow spice used as a dye) is also observable in Classical Arabic, \textit{kurkum} `turmeric', historically also `saffron' (etymon of the word \textit{curcuma}), and the perceived ``similarity'' of Sanskrit \sa{कुङ्कुम} \textit{kuṅkuma} `saffron' did not help to clear the waters either. See \cref{sec:turmeric} for more on this issue.

Keeping all this in mind, I feel I must lean on rudimentary botanical identification in the investigation, linking the plants and plant parts to the products and their vernacular names. This is important, as it can clear up some of the confusion when two or more product names are used interchangeably, and it will highlight problematic cases from the start.

\subsection{Challenges in Spice Categorization}
% Rewrite

One of the most challenging parts of this project, is to choose a meaningful way to categorize spices and spice names. The design should make sense on multiple dimensions: botanically, historically, and maybe even gastronomically, but at all times keeping in mind the linguistic focus. The main goal is to assign a spice name to the appropriate product/material, which is correctly identified on a botanical level. This is not always straightforward, as some materials can have multiple botanical sources, one plant can yield multiple differently used plant products, and the same names can be used for different substances. 
% We have already discussed the question of spice names vs. plant names in \cref{sec:plant_vs_spice}, but there are are several other issues.

One problem arises from the fact that many terms can have a meaning on different levels of specificity, depending on context and intent. Spice words are rich in senses. For example, according to the \gls{PWN} \textit{black pepper} can be both a hypernym and a hyponym to \textit{pepper}, depending on if it refers to the plant, or the dried fruits with the husks on \autocite{fellbaum_wordnet_1998}. In this specific case, \textit{black pepper\#2} and \textit{white pepper} are sister terms, but \textit{white pepper} is also a subordinate to \textit{black pepper\#1}. This situation is then further complicated with the fruit of the \textit{Capsicum} (and its endless cultivars), that also have the name \textit{pepper}. Thus, it is not immediately clear if we should treat black and white pepper as two different spices, or two manifestations of the same spice. There are many other examples where a term can be understood on different levels: as a plant, a family of similar plants, a specific spice, or a group of spices. In an everyday setting, lexical semantic hierarchies are not always adhered to, and people organize spices in their heads according to their own convictions. One author might mention white pepper under the heading black pepper on account of their biology (a botanically driven categorization), while another might separate them and discuss them as different spices based on their different uses (a culinary approach). As for the historian, mentioning white pepper might be just not at all important. The reasons for these variations are usually determined by what is the intention of the categorization, and who is the target behind the treatise. For us layman however, spice entities are most prominently structured by way of their names: the words are the handrails to cling to if we are not familiar with an item. So what about pink peppers? Pink peppercorns (\textit{Schinus terebinthifolius; S. molle}) are so-called ``false peppers'', meaning that they are not from the \textit{Piperaceae} family. Pink peppers are fruits of botanically unrelated trees in South America, matching the shape and size of peppercorns. Are they a kind of pepper?

My point with the above---admittedly rather confusing example---was to show that if one were to debate whether black pepper and white pepper are the same spice (or not, for that matter), we would need some gastronomical or botanical grounds to make some arguments. And despite all imaginary arguments, the real answer goes beyond the botanical or gastronomical reality (illustrated by adding pink pepper to the problem), and answers the question: What is pepper conceptually? We will see from the names and naming practices, that the word \textit{pepper} and its equivalents carry a sense of `prototype pungent spice', and overarching biology or function it is ultimately a concept of \textsc{pepper} that matters. And, I propose that whenever a novel spice is considered appropriately close to the existing concept, it could also fit the category of pepper, reflected by the names.

Even more challenging for categorization, is when we are not sure which spices were meant under certain names in different times. Cinnamon and cassia are a great example for this (\cref{sec:identity_cinnamon}), as it is not sure whether the cinnamon and cassia of antiquity were the same spices or not. But, parallel to the question of identity, we also have seen that it does not always matter, attitudes differ from place to place: while Europeans do sometimes differentiate, in China and the United States the concept of cinnamon is singular. For the analysis, I had to decide if I treat them as one item, or make a distinction. In a few cases, a spice name became obsolete and got ``lost'', meaning that it cannot be identified with certainty, and we have to guess what the name referred to based on botanical and historical data, and categorize accordingly; as it is the case with \textit{amomum}. The most extreme situation is when a spice goes extinct, as it happened to silphium in antiquity. At present, this thesis does not contain such items. In these cases, we need historical knowledge to say anything about the identity of said spices and where they belong in between the others.

Our knowledge or lack thereof also determines the concept we have of a certain item. For example, most people who know that nutmeg and mace come from the same fruit of the same plant and from the same place will always connect the two in their heads, the two spices are literally inseparable (until harvest, of course). From historical records however, it is clear that the knowledge regarding these substances was spiked with misunderstandings and inaccuracies, even among people who were in the spice business. According to an anecdote, during the Dutch monopoly of the Banda islands, an officer back home have written an angry letter to the colony on the Spice islands, ordering them to plant more mace trees, because there was a higher demand for it than nutmeg---a request that must have raised some eyebrows on the plantations \autocite{national_geographic_nutmeg_2014}. This shows that botanical organization is accessible to those with botanical knowledge.

Lastly, I must also mention that the language and words we use for these materials also defines our understanding of them. Analyzable, descriptive names help us to identify certain materials, while loanwords with forgotten original meanings (cf. \textit{mace}) might not say much. For example, no Chinese would have the above misconception of mace, when faced with its name: \tc{肉豆蔻皮} \textit{roudoukoupi}, which means the `skin/cover of the nutmeg', which is what it actually is. On the converse, the Chinese initially confused some cardamoms and nutmeg (unrelated plants), simply because they were both round, and sourced from the same region. Today, both are \tc{豆蔻} \textit{doukou}, with modifiers attached to distinguish between them.

Another point to make is the myriad of ``fake'' spices that feature especially in English. False peppers, false cardamoms, bastard cinnamon, and bastard saffron, are a reflection of historical economic attitudes, often pointing at the problem of adulteration. Names, such as \textit{true pepper} and \textit{true cinnamon} summon a sense of authenticity. This, however, is highly subjective to a culture and language, after all, bastard cinnamon is just ``normal'' cinnamon for others, and false cardamoms are just cardamoms to those who have a different prototype for what is a cardamom. In a sense, it all boils down to translation, which can be arbitrary. Who decides if Chinese \tc{桂} \textit{gui} should be rendered \textit{cinnamon} or \textit{cassia} in English? 

To avoid getting lost in the details of lengthy binomial names or botanical genera, I have opted to use a set of common names of the spices to be used for identification, under which the various spice names belong. These IDs are sometimes arbitrary (e.g.: all spicy, red, hot, chili peppers of the \textit{Capsicum} genus and their names go under ``chile''), but always clear cut and explained in the data chapter. I have therefore grouped some spices and spice names into larger categories, trying to find a smallest common denominator within the three languages. This only affects a few items: various false cardamoms in the \textit{Amomum} genus will be grouped under the umbrella term: false cardamom. One better way would have been to divide the categories on a purely botanical basis, but I prefer this solution to make this set of closely related spices more accessible to the reader and avoid these items to fritter away in the crowd. Also, they constitute a linguistic and conceptual category as well, using similar prototype words in all three languages in their names. Using common names as identifiers also facilitates for a linguistically driven comparison, and so the IDs are essentially the same as the set of spices determined earlier in \cref{table:set}: allspice, anise, asafoetida, caraway, cardamom, cassia, chile, cinnamon, clove, coriander, cumin, dill, false cardamoms, fennel, fenugreek, ginger, long pepper, nutmeg, pepper, saffron, Sichuan pepper, star anise, turmeric, vanilla.

\section{Data Collection}
\label{sec:data_collection}

% I had select a set of spices that will constitute the basis for this project.

The data collection for this project was conducted in three stages. One for assembling the set of spices, one for gathering and analyzing their names, and one for researching etymologies. The result of these three stages are three datasets open for inspection as the electronic files \texttt{spices.csv}, \texttt{names.csv}, and \texttt{etymologies.csv}, available on my GitHub page: \url{https://github.com/partigabor/phd-thesis-viz/tree/main/data}. \Cref{ch:data} will introduce and explain the data in all three levels.

\subsection{Collecting Spices}
\label{sec:collecting_spices}

In the first stage, after I have assembled the set of spices, I collected information about them from encyclopedic handbooks written by experts in the plant sciences and spice industry professionals. I have made great use of \textcites{van_wyk_culinary_2014}{mabberley_mabberleys_2017}{hu_food_2005} at the start, especially when matching plant products to plants. At this stage, I have focused on the identity and characteristics of spices including geographical distribution and native habitats, especially where I saw any room for confusion. As I collected scientific names, I also recorded the common/vernacular names for each spice as an initial exploration, and I linked them to a botanical database that can supply further information. I have also collected information regarding their cultivation, and basic uses.

Surprisingly, the abundance of synonyms is also palpable in the scientific nomenclature, sometimes one plant species has dozens of binomial \glspl{taxon}. In an attempt towards standardization of taxonomic data, collaborative efforts have sprung across numerous authoritative institutions to assemble and link their respective databases and sources. These online projects are usually run by a consortium of leading botanical institutions worldwide, among the key entities are the Royal Botanic Gardens at Kew and Edinburgh, the Missouri Botanical Garden, the Harvard University Herbaria \& Libraries, Geneva Conservatory and Botanical Garden, the Muséum National d'Histoire Naturelle in Paris, the South African Biodiversity Institute, the Australian National Botanic Gardens, and the Kunming Institute of Botany, just to name a few.

When it comes to botanical information, navigation in the ocean of scientific binomial names hiding the identity of a plant can be overwhelming \autocite{spencer_plant_2020}. To alleviate this, I turned to a range of botanical databases for the purposes of correct identification, and information gathering. I used databases such as \gls{TPL} (\url{http://www.theplantlist.org}), which was recently superseded by the \gls{WFO} (\url{http://www.worldfloraonline.org}); the \gls{IPNI} (\url{http://www.ipni.org}); \gls{POWO} (\url{http://www.plantsoftheworldonline.org}); the \gls{GBIF} (\url{https://www.gbif.org}); the \gls{FOC} hosted on eFloras (\url{http://www.efloras.org/index.aspx}) and the \gls{BHL} (\url{ https://www.biodiversitylibrary.org/}). \gls{TPL} for instance claimed to be ``a working list of all known plant species'', now under \gls{WFO} it is ``an online flora of all known plants'' , and as such also connects different plant checklists and biodiversity databases using the nomenclatural and publishing information. In my dissertation I will frequently refer to \gls{POWO}, which contains botanical descriptions and geographic data (native and introduced habitat), besides the usual taxonomic and botanical information. 

In addition to online databases, I will occasionally also turn to reference books from the field of food technology and nutritional science, such as the \textit{Handbook of Herbs and Spices} \autocite{peter_handbook_2012,peter_handbook_2006}, and \textit{The Encyclopedia of Herbs \& Spices} \autocite{ravindran_encyclopedia_2017}. These encyclopedias, although aimed at chemistry-focused food industry professionals, also contain holistic information on the plant-based products and discuss the origins and vernacular names, besides the usual particulars on usage and medicinal qualities. It is also worth noting that various dictionaries usually mention the scientific names of plants.

Regarding traditional medicine systems, I frequently consulted modern inventories of \gls{TCM} to identify materials and extract Chinese names, including the connecting databases of Hong Kong Baptist University: the HKBU Medicinal Plant Images Database\footnote{\url{https://library.hkbu.edu.hk/electronic/libdbs/mpd/index.html}}, the HKBU Chinese Medicinal Material Images Database\footnote{\url{https://library.hkbu.edu.hk/electronic/libdbs/mmd/index.html}} HKBU Chinese Medicine Specimen Database\footnote{\url{https://libproject.hkbu.edu.hk/was40/search?channelid=44273}}; and the PolyU Chinese Herbal Medicine Database\footnote{\url{https://herbaltcm.sn.polyu.edu.hk/}}. Armed with the botanical knowledge, we shall have an ideally clear picture on the spices, and a firm base to connect linguistic data to.

% Hong Kong Herbarium
% % https://www.herbarium.gov.hk/en/home/index.html

\subsection{Collecting Names, and Their Annotation}
\label{sec:collecting_names}

In the second stage, I have collected the names of spices by combing through the published literature and online databases, whether botanical as described above, historical, or culinary. Always, prioritizing the existing linguistic and philological treatises, of course. I have linked the collected spice names to the respective spices and the result of this is an inventory of around 360 spice names that link to the initial set of 24 spices. For each spice, I tried to collect their names in the three languages, and it was also my goal to record where I have found these names. Therefore, thorough citations are available in the dataset pointing towards books, journal articles, databases, dictionaries, or sometimes even Wikipedia. As a preparatory step for the linguistic analysis, I have added some annotations. 

\subsubsection{Conventionalized Terms}

First and foremost, I have checked the words against dictionaries to see if their use is conventionalized or not, and I have marked words that appear in a dictionary. If a word occurs in multiple dictionaries, I only recorded the one that I deem the most authoritative or reliable, unless they are both extremely interesting entries (or contradict each other).

\subsubsection{Present Status of the Terms}

Then, as an internal operational measure, I have assigned the names into categories regarding their lexicographic status as spice terms: default, alternative, historic, archaic, and obsolete. This was mostly done for myself to better orientate after the terms started to accumulate, and I used the following scheme: 

``Default'' marks the names the spices are mostly prevalently known by today, the terms that most people are familiar with. They comprise the words that should be most commonly found in a dictionary, or most frequent in a corpus. These are also the names you see as section-headers in the thesis, and also act as IDs in my datasets. The term ``default'' as an indicator is somewhat arbitrary, since there is no reason for one item not to have several equally relevant synonyms (e.g., \textit{chili} vs. \textit{chili pepper}), but I needed to choose one main term to represent one spice. The reasons for this are the following: (1) I needed a convenient way to ``call'' each item, so they can be efficiently compared across the three languages. (2) I needed an identifying key for all of the other names of the same spice, and (3) I wanted to avoid any possible confusion between item that have overlapping common names (i.e., \textit{pepper} vs. \textit{pepper} is problematic, so I settled with \textit{pepper} vs. \textit{chile}\footnote{In my dataset and code, I use the more botanically affiliated term, \textit{chile}, to avoid confusions/errors due to spelling.}).

``Alternative'' refers to any other current name that a spice can be known by, regardless of popularity, context, or reason. For example, \textit{aniseed} is an alternative for \textit{anise} (the default term), and \textit{Chinese parsley} is an alternative name (and also an alias) for \textit{coriander}.

``Historic'' refers to once important terms that were the at a certain point in history would have been considered default, but---due to their role and popularity in the past---still relevant today. This category especially includes cases where a spice was attested under a different name from what it is known by now. For example, \textit{badian} is now a chiefly historical term and was attested before the now standard \textit{star anise}.

``Archaic'' refers to historic words that are rare and not relevant today, but still recognizable, such as \textit{Guinea pepper}, anno an early name for Cayenne pepper (a name for chile, \textit{Capsicum annuum}), but referring to one of three African spices today unrelated to chile.

``Obsolete'' refers to names that are essentially dead, cf. \textit{amomum}, which was last used to denote a specific spice in the \nth{19} century. Most of the above categorizations were made by following dictionaries. If a dictionary uses these remarks, (e.g., obsolete), I comply with the dictionary. 
%Examples?
I have identified a couple of cases that could be best characterized as ``speculative'', this refers to spice names that are not attested anywhere, and I assume them to be the author's invention/translation. 
One example for this would be the term \textit{English spice} for `allspice', found in \textcite[64]{raghavan_handbook_2007} but nowhere else, where I think the author decided to translate this name from some other language which does use this name (e.g.,Polish). The motivation behind the name is that Jamaica, where allspice is sourced from, was a British colony, and it was the English, who disseminated allspice in Europe during that time.

% For example, 
% title: king of spices, queen of spice, black gold, red gold

I have highlighted the so-called default items in bold throughout the tables in \cref{ch:data}, as they also act as a keys or identifier (ID) to the rest of the alternative names corresponding to the same spice.

\subsubsection{Borrowed Terms}
\label{sec:borrowed}

In my analysis, I have marked spice terms according to their borrowed status. Based on data from dictionaries, etymological dictionaries, primary and secondary literature and my own judgment, I have indicated if the name is a borrowing or not, or whether it needs further checking. I have annotated spice names with `yes', `not', and `maybe'. Whenever available, I relied on word origins from general and etymological dictionaries for this information, but for a number of words I could not find existing entries or published research, and I introduced my own theories. 

% On a deeper level, I have also annotated the nature of the borrowing: whether it is a phonetic loan, calque (loan translation/semantic translation), learned loan, or phono-semantic matching, and marked folk etymologizations.


% 5.3. Borrowed Most importantly, of course, contributors were asked to indicate whether, to the best of their knowledge, the word was a loanword, i.e., had been borrowed from another language at some point in the language’s history. Protolanguages were also considered stages of the same language, so that a word borrowed into Proto-Uralic, for example, would count as a loanword in Saami. Five degrees of certainty were distinguished

% I. The Loanword Typology project and the World Loanword Database 13 0. No evidence for borrowing 1. Very little evidence for borrowing 2. Perhaps borrowed 3. Probably borrowed 4. Clearly borrowed A value such as “Clearly not borrowed” or “Clearly inherited” was not used, because any word could have been borrowed at some prehistoric time, so we can never be sure that a word is not an old loanword. And even loanwords can be inherited, e.g.,a word borrowed into Proto-Uralic can be inherited by Saami. We define a loanword as a lexeme that has been transferred from one lect into another and is used as a word (rather than as an affix, for example) in the recipient language. Words from a substrate language, too, were considered to be loanwords for the purposes of the LWT project, so we include both adopted and imposed words (see chapter II, §2, §7.4). Lexemes transferred from one regional dialect to another and between an acrolect and a basilect were also treated as loanwords in principle, although in practice they play a minor role. Excluded from the class of loanwords are neologisms (= productively created lexemes) which consist partly or entirely of foreign material, because they are created in the recipient language, and not transferred from a donor language (cf. §5.5.4; but see “calqued” in §5.5.3). 5.4. Age For each word, contributors gave the earliest time at which it was attested or could be reconstructed in the language. For loanwords, this meant the time when the word was borrowed. For nonloanwords, it meant the time of earliest attestation or reconstruction. Dates could be indicated by years (or centuries) or by period name, e.g.,“Middle High German”, or “Tang dynasty”, in which case contributors were asked to provide approximate dates for the periods, e.g.,1050–1350 for “Middle High German”, 618–907 for “Tang Dynasty”, and 5000–3000 BCE for “ProtoIndo-European”. Knowing the age of a word is important in this context for several reasons. For nonloanwords which have only existed in the language for a relatively short period, it is not possible to draw conclusions regarding borrowability: they may be replaced by a loanword given sufficient time. On the other hand, a word that has been present in a language for a thousand years without being replaced by a loanword provides good evidence that its meaning is less borrowable. For older loanwords, tracing their origin can be more rewarding since we are less likely to know the history of their borrowing situation compared to newer loanwords. Studying old loanwords can thus help fill important gaps in historical and archeological knowledge. Finally, in a diverse sample such as ours, the history of some languages is relatively well documented for thousands of years, while others have only been re

% 14 Martin Haspelmath and Uri Tadmor corded for less than a century. Naturally, it would be much easier to identify and trace loanwords in languages with well-documented histories. Knowing the age of loanwords thus enables us to make much more nuanced cross-linguistic comparisons.


% Refer to loanword section in Theory

\subsubsection{Meanings, Literal Meanings, Glosses}

For every term in Arabic or Chinese, I added a gloss, so the literal meanings could be decoded, and most names also have written notes and comments on their logic, formation, origin, or any other remarkable aspect. Sometimes a short explanation is needed to understand the emergence of a term, or the grounds for its existence. The dataset of spice names was populated with terms corresponding to the botanically informed binomial names and the materials they represent, and based on the information from stage one, the names were also annotated with the macro-areas of their native geographic origin.

% Before moving on to analyze this list of spice terminology...

\subsubsection{Attestation}

I have recorded the details concerning attestation where available, noting a date, approximate date, century, and period (i.e., early Old English, Tang dynasty, etc.). For this information I used dates from the \gls{OED}, in English, and historical corpora for Arabic and Chinese where available. The source of the attestation dates are noted in the dataset. Whenever this was not available, I resorted to estimation based on circumstantial historical sources. These are all marked in the relevant dataset.

I have also tried to gather the pre-modern documents where each name was recorded, with the title and author of the historical works for future reference. % Topic? Medicine, religion, history, blabla.

% In English, I had a relatively easy job, as the \gls{OED} is very rich in etymological information, full of quotations. 

%Estimation and guesses?

\subsection{Collecting Etymologies}
\label{sec:collecting_etymologies}

In the third stage, I have collected detailed etymological information on selected names: the terms that were marked as default, and a few historic and highly relevant alternative names. Doing so, we now have a parallel set of spice nomenclature of the three languages for 24 spices, and we can compare them in terms of borrowed status, and their etymological development and origins. The etymologies will be discussed in the next chapter in detail, under every spice, and I also highlighted them using dedicated environments called \textit{Etymology boxes} (see for example Etymology \ref{ety:allspice}).

In terms of representation and storage, I deviated from the usual text format, and I have recorded etymological data in a way that it is machine-readable, but still easy to grasp and edit for the human eye as well. I have separated etymological stages, and types of information for each word, creating large spreadsheets that is relatively easily accessible and modifiable for both man and machine. Doing so, I enabled a way to extract only specific information when needed (sources, attestation dates, donor languages, etc.). I also facilitated for geospatial plotting that can be found in \cref{ch:language}, which gives a visual representation of the etymological stages the words have embarked on.

% \subsubsection{The Representation of Etymological Data and its Problems}

% While collecting etymologies, I have spent an obscene amount of time trying to find the best way to store etymological data. I have constantly changed the way I organize the information I have collected and reviewed current efforts on the topic. To my surprise, 

\subsection{Collecting Additional Data}

To facilitate for geospatial mapping, I needed language data that supplies coordinates. For this purpose, I used the \gls{WALS} \autocite{dryer_wals_2013} and chiefly the Glottolog 4.6 \autocite{hammarstrom_glottolog_2022} datasets published by the Max Planck Institute for Evolutionary Anthropology under Creative Commons Attribution 4.0 International License. I have altered my data to conform to the above datasets in two ways. One, variation in language names were mapped to Glottolog \textit{languoids}\footnote{\url{https://glottolog.org/glottolog/glottologinformation}}, for example, Middle Persian is mapped to Pahlavi. Two, I have added coordinates to some (dead) languages that did not have a location, for the sole purpose of putting them onto a map. For example, Medieval Latin lacked coordinates, and I have added the approximate geographical midpoint of Western Europe, where it was primarily used.

\section{Sources}

\subsection{Primary Sources}

One core component of this study is philological research. Philology is the meticulous study of literary texts, primarily of historical documents, to study language, history, philosophy, literature, culture, religion, or any traditional knowledge of exceptional importance strongly connected to a society, primarily through the analysis of historic texts (sometimes written in now dead languages). Modern philological research relies on two types of sources: primary and secondary literature. Primary literature denotes historical texts, the so-called classics, for example, the already mentioned \textit{De Materia Medica} of the Greek physician Dioscorides (c. 40--90 \textsc{ad}) \autocite{dioscorides_materia_2005}, books of Roman historians, such as Pliny the Elder (23/24--79 \textsc{ad}) and his \textit{Naturalis Historia} \autocite{pliny_the_elder_natural_1855} are good examples, not to mention the or \nth{1}-century cookery book by Apicius \autocite{apicius_apicius_1977}. There also available \glspl{materia medica} from the Islamic scientific golden-age, such as the \gls{Qanun} \textit{[Canon of Medicine]} of Ibn Sīnā/Avicenna (980--1037) \autocite{ibn_sina_-qanun_1329} and fantastic miscellanies from the Tang dynasty era, such as the \gls{Youyang} \textit{[Miscellaneous Morsels from Youyang]} from the \nth{9} century \autocite{yyzz}. Indeed, we must not forget the Bible or Quran, as they are also rich historical and linguistic sources for our topic. A number of these primary texts are available in their original form through museums' and libraries' online databases, as transcribed editions in historical corpora, and of course published English translations. A vast number of classical texts (Greek and Latin) can be accessed through the Perseus Digital Library \autocite{crane_perseus_nodate}. Critical editions of a classical text, such as that of the famous \gls{Periplus} by \textcite{casson_periplus_1989}, or \textcite{de_goeje_bibliotheca_1870}'s \textit{Bibliotheca Geographorum Arabicorum} series are also considered primary. Ancient and Classical dictionaries, such as the \gls{Shuowen}, or the \gls{Lisan} are also an integral part of philology. Secondary literature is everything else building on these works, monographs, histories reviewing a multitude of authentic texts, published in recent times.

% Also, research involving specialists from other field of science, such as the new archaeogenetics, pushing forward progress in biology and evolutionary linguistics can be related.



\subsection{Etymological and General Dictionaries}

Besides the literature itself discussed earlier, a core part of the philology component in this research are etymological dictionaries. Etymological thirst, the seeking of word origins was one of the cardinal thrills for early thinkers ever since Plato, and we will make use of the advances made in the past centuries. The \gls{OED} has detailed etymological information based on previous works on English and for other languages, a couple of works to be mentioned are for Greek \textcite{beekes_etymological_2010}, Hebrew, \textcite{klein_comprehensive_1987}, Old Chinese \textcite{schuessler_abc_2007} and Chinese \textcite{liu_hanyu_1985}. Unfortunately, Arabic lacks an authoritative etymological dictionary for many reasons\footnote{For a brief overview on the matter, see \textcite{blazek_etymology_2006}}, but we can still turn to essential reference works such as the \textit{Encyclopedia of Islam}\footnote{Limited access online at \url{https://referenceworks.brillonline.com/browse/encyclopaedia-of-islam-2}} \autocite{ei2} or the \textit{Encyclopedia Iranica}\footnote{Accessible online at \url{https://iranicaonline.org/}} \autocite{eir}. 




% Key dictionaries were consulted throughout the data collection process, the following is an enumeration of the general or historical dictionaries I used: 

% A full list of general and etymological dictionaries are listed here with complete references.

% \subsubsection{English:}

% \textcite{editors_of_the_american_heritage_dictionaries_american_2022} \glsxtrfull {AHD}\footnote{Accessible online at: \url{https://www.ahdictionary.com}}

% \noindent \textcite{bosworth_anglo-saxon_2014} \glsxtrfull{BT}\footnote{Accessible online at: \url{https://bosworthtoller.com}}

% \noindent \textcite{hoad_concise_2003} \glsxtrfull{EE}

% \noindent \textcite{lewis_middle_1952} \glsxtrfull{MED}\footnote{Accessible online at: \url{https://quod.lib.umich.edu/m/middle-english-dictionary/}} 

% \noindent \textcite{merriam-webster_merriam_nodate} \glsxtrfull{MW}\footnote{\url{https://unabridged.merriam-webster.com/}}

% \noindent \textcite{harper_online_nodate} \glsxtrfull{OE}\footnote{\url{https://www.etymonline.com/}}

% \noindent \textcite{oed} \glsxtrfull{OED}

% \noindent \textcite{cresswell_oxford_2021} \glsxtrfull{WO}

% \subsubsection{French:}

% TLFi Trésor de la langue Française informatisé, http://www.atilf.fr/tlfi, ATILF - CNRS \& Université de Lorraine
% https://www.cnrtl.fr/etymologie/safran

% Sanskrit
% Monier-Williams Sanskrit-English Dictionary (Monier-Williams)

% Arabic:
% (Hans-Wehr)
% (Lane's Lexicon) 

% \gls{CAD}

% \subsubsection{Greek:}

% \glsxtrfull{LSJ}\footnote{Accessible via the Perseus Digital Library. Ed. Gregory R. Crane, Tufts University \url{http://www.perseus.tufts.edu/hopper/}}---\autocite{liddell_greek-english_1940}

% % For Chinese, classic dictionaries, such as the Shuowen Jiezi or the Kangxi Zidian (1716) can be accessed online via the Chinese Text Project (https://ctext.org/), among other online dictionaries such as Handian (www.zdic.net/). Dictionaries usually give scientific name for spices and plants, it is up to us, which lexicographer we deem competent in botanical questions.

% such as Monier-Williams (1899) for Sanskrit, Wehr (1976) and Lane's lexicon (Lane \& Lane-Poole, 1863-1893) for Arabic, Steingass (1892) for Persian25, and the American Heritage Dictionary (\url{www.ahdictionary.com/}) are mostly searchable electronically. 

% \subsubsection{Latin:} 

% % \glsxtrfull{LS}\footnote{Accessible via the Perseus Digital Library. Ed. Gregory R. Crane, Tufts University \url{http://www.perseus.tufts.edu/hopper/}}---\autocite{lewis_latin_1879}

% Post-Classical and Medieval Latin:

% Du Cange et al., Glossarium mediæ et infimæ latinitatis. Niort : L. Favre, 1883-1887
% http://ducange.enc.sorbonne.fr
% http://ducange.enc.sorbonne.fr/PIGMENTUM

% Slovakian
% https://fran.si/

% https://coptic-dictionary.org/

% Turkish:
% https://www.nisanyansozluk.com/

% An Anglo-Saxon dictionary, based on the manuscript collections of the late Joseph Bosworth (the so called Main Volume, first edition 1898) and its Supplement (first edition 1921), edited by Joseph Bosworth and T. Northcote Toller

% Spanish
% DLE
% The Royal Spanish Academy: Diccionario de la lengua española https://dle.rae.es/

% https://dle.rae.es/

% http://www.jergasdehablahispana.org/ ``A Spanish dictionary specializing in dialectal and colloquial variants of Spanish, featuring all Spanish-language countries including Mexico.''

% https://www.spanish-translator-services.com/articles/latin-american-spanish.htm
% ``This is the universal and somewhat arbitrary name that is given to idiomatic and native expressions and to the specific vocabulary of the Spanish language in Latin America.''

% OND Nahuatl Dictionary
% Online Nahuatl Dictionary, Stephanie Wood, ed. (Eugene, Ore.: Wired Humanities Projects, College of Education, University of Oregon, ©2000–present)
% https://nahuatl.uoregon.edu/
% \autocite{wood_online_2000}

% CAL Comprehensive Aramaic Lexicon

% \url{https://cal.huc.edu/}

% Malagasy
% \url{https://en.mondemalgache.org/bins/homePage}

% https://stedt.berkeley.edu/ Professor James A. MATISOFF 

% Linguistic datababes/corpora

% Chinese Text Project edited by Dr. Donald Sturgeon \url{https://ctext.org}


% Old Chinese phonology based on Zhengzhang 2003

% PHILOLOGY

% http://www.filaha.org/

% https://www.fihrist.org.uk/
% % \section{Etymologies}

% Armenian:

% http://www.nayiri.com/

\begin{note}
    References to dictionary entries are made very frequently in this dissertation, and so I made the decision to use a compact way of citing dictionaries. Instead of following the standard APA \nth{7} guideline and referencing every entry separately, I will indicate the entry as a page number or headword and reference every dictionary just once. This would save us from the pain of reading (Oxford University Press, n.d.-a) (Oxford University Press, n.d.-b) (Oxford University Press, n.d.-c) and its endless permutations. This minor deviation from the APA style will make the number of dictionary entries in the bibliography less oppressive, and the running citations more reader-friendly. I will also use footnote citations whenever I reference a dictionary, and I stick to this practice throughout the dissertation to make reading more comfortable.
\end{note}

\section{Corpora}

The second major component of this study is corpus linguistics, and I will use corpora from three major languages: English, Arabic, and Chinese. I chose these languages for two reasons. One, they represent three influential civilizations in the history of spices, as well as powers actively participating in trade throughout history, each having its zenith at slightly different historical periods, as I described previously. 
Two, these languages have historical corpora available.

% The corpora available from these pre-modern times is a corpus of Classical Arabic literature, a Chinese corpus of Tang era poetry, and other corpora containing textual data from relevant times. All corpora and databases are presented in greater detail in section 3 Methods.

% The data from the available corpus is largely from the 1600s, which conveniently falls into the historical period known as Age of Exploration/Age of Discovery, when Europeans sailed the globe far and wide, competing in the search for new spices and new worlds. To sum up, the project will focus on Arabic (Classical-Modern Standard), Chinese (Classical/Middle-Modern), and English (Early Modern-Modern).

% , comparable web corpus data, chosen from the TenTen corpus family (Jakubíček et al., 2013) for all three languages: Arabic (Modern Standard Arabic; MSA), Chinese (Mandarin, Simplified and Traditional), and English. The selected corpora are all available on the Sketch Engine at https://app.sketchengine.eu (Kilgarriff et al., 2014). We are interested in the contemporary `language of spices', the words of the spice domain and related terminology will be explored from linguistic-cognitive perspectives.

% However, we would also like to accommodate for the change of meaning on what spices once were, and how humans use them now vs. earlier periods. Traditional folk medicine and herbal remedies are residues of the past, apparent to different degrees in various cultures. To look for evidence for this shift in language use, we choose to explore historical corpora as well. 




For modern corpora, I will use the English Web 2020 (enTenTen20, circa 36.5 billion words),\footnote{Accessible at \url{https://www.sketchengine.eu/ententen-english-corpus/}} the Arabic Web 2012, preprocessed with the Stanford tagger (arTenTen12, ca. 7.5 billion words),\footnote{Accessible at \url{https://www.sketchengine.eu/artenten-arabic-corpus/}} and the Chinese Web 2017, Simplified version(zhTenTen17, ca. 13.5 billion words),\footnote{Accessible at \url{https://www.sketchengine.eu/zhtenten-chinese-corpus/}} all hosted on the \gls{SkE} (\url{https://www.sketchengine.eu/}) \autocite{kilgarriff_sketch_2004,kilgarriff_sketch_2014}. Enormous web corpora such as the above contains billions of words, therefore I will certainly have enough instances even for spices more rare.

\begin{table}[ht]
    \begin{tabularx}{\textwidth}{@{}lllLl@{}l@{}}
    \toprule
    \textbf{language} & \textbf{type} & \textbf{period} &  \textbf{corpus} & \textbf{size} & \\ \midrule
    English           & web      & modern                       & enTenTen20 & 36,5 & billion words    \\
    Arabic            & web      & modern                       & arTenTen12 & 7,5 & billion words    \\
    Chinese           & web      & modern                       & zhTenTen17 & 13,5 & billion words    \\ \midrule
    English           & books    & historic (15--\nth{19} c.)   & EHBC      & 826   & million words    \\
    Arabic            & books    & historic (~7--\nth{12} c.)    & KSUCCA    & 47    & million words    \\
    Chinese           & books    & historic (~~~--\nth{20} c.)     & Chinese Text Project    & 25    & million characters \\ 
    Chinese           & books    & historic (~~~--\nth{20} c.)     & Scripta Sinica        & 797   & million characters \\ 
    Chinese           & books    & historic (~~~--\nth{20} c.)     & CBETA        & ?   & million characters \\ 
    \bottomrule
    \end{tabularx}
    \caption[The list of corpora consulted in the thesis.]{The list of corpora consulted in the thesis.}
    \label{table:corpora}
\end{table}

In terms of historical corpora, I have consulted a few collections. For English, I relied on the \gls{EHBC} (EEBO, ECCO, Evans) hosted on the Sketch Engine, that is around 826 million words and contains books published between 1473--1820, with a vast majority written around 1600, but the \gls{OED} itself is full of historical quotations and attestation dates.
% English also has a good coverage on the Google Books project ??. 
For Arabic, I have settled on using the \gls{KSUCCA}, which is around 47 million words containing literature on various genres between the 7-\nth{11} centuries, ranging from books on medicine, geography, law, history, and religious texts \autocite{alrabiah_design_2013,alrabiah_empirical_2014}. As for Chinese, I have frequented the \gls{CTP} \autocite{sturgeon_chinese_nodate,sturgeon_chinese_2021} which has base of 25 million characters pre-modern Chinese documents, not including the community edited texts. I also used the \gls{SS}\footnote{Accessible at http://hanchi.ihp.sinica.edu.tw/ihp/hanji.htm}, around 754 million words containing classics ranging from ancient times up until 1949 \autocite{academia_sinica_scripta_1993}; the \gls{QTS}\footnote{Accessible via the \gls{CTP}: \url{https://ctext.org/quantangshi}} [Tang poetry collection], which contains around 48,900 poems; and the \gls{CBETA}\footnote{Accessible at \url{https://cbetaonline.dila.edu.tw/en/}} project, which contains the Chinese Buddhist Canon, also known as the Chinese Tripitaka \autocite[365-386]{chen_buddhism_1964}. Thus, accommodating textual heritage from ancient times up until the 20th century.

% groundbreaking

%  3. Methods
%  The methods of the present study can be divided into two main components, along the lines of the two main objectives. Objective A, exploring the “diffusion of spices” will involve classical philological research, augmented with the help of historical corpora, while Objective B, exploring the “language of spices” will use corpus linguistic methods.
%  3.1. Exploration
%  Before setting out on a philological journey or diving into corpora for massive data collecting adventures, we must clarify what we want to collect data about, and how we are going to compare spices and spice-words. This dissertation being a linguistic study, the words/names and their meanings pertaining to spices, the “semantics of spices” seems to be an appropriate point of commencement. Spicy word senses
%  We decided to choose word senses as the basis of comparison, and we will collect semantic data from lexical databases, such as the WordNet 3.1, also known as the Princeton WordNet of English (PWN)16 (Fellbaum, 1998; Miller, 1995), the Open Multilingual Wordnet (OMW)17 (Bond & Paik, 2012), and the Extended Open Multilingual Wordnet 1.2 (EOMW)18 (Bond & Foster, 2013). Wordnets map semantic relations between words. These semantic networks clarify multiple senses of a word and help us identify meanings useful for our investigation.
%  For example, the word `clove' has seven meanings, four as a noun and three as a verb. Three of the nominal meanings refer to Syzygium aromaticum---the fragrant tree native to the Moluccas, once also known as the Spice Islands---and the remaining one represents the cloves of a garlic (#3, not a cognate word). See Figure 3 below for an overview of the relations between the clove tree (#2), its fresh flower bud (#1), and the dried form of the bud used as a spice, whole or ground (#4).

\section{Illustrations}

% Botanical illustrations are taken from \textcite{kohler_kohlers_1887}, a legendary work on medicinal plants among those interested in anatomical drawings of \textit{Plantae}. Due to the book's age, the pages reproduced here are in the public domain. %Further sources are...

All photographs of spices displayed (except where stated otherwise) are by courtesy of Christine Latour at Aromatiques Tropicales, a spice vendor in Dégagnac, France (\url{https://www.aromatiques.com/en/}). Credit is due to the photographers Felix Farmer\footnote{\url{http://www.felixfarmer.com/}} and Philippe Janina\footnote{\url{https://philippejanina.wixsite.com/photographe}}. For some images of incense, photo curtesy by Glorian (\url{https://glorian.org/}).

% Plots.

% \section{Note on Scholarly Conventions}

% "Commonly used spellings, even if at the sacrifice of consistency"





% \section{Analysis}

% Prototype