% 6 Methodology
% Who what why when and how?
% What did you do to achieve the research aims?
% Why did you choose this particular approach over others?
% How does it relate to your epistemological and ontological positions?
% What tools did you use to collect data and why? What are the implications?
% When did you collect data, and from whom?
% What tools have you used to analyze the data and why? What are the implications?
% Are there ethical considerations to take into account?

% Bottom-up

\chapter{Methodology}
\label{ch:methodology}

% Who what why when and how?

\lettrine[lines=\iniciale]{\textcolor{\accentcolor}{R}}{ealizing} that there is little work done on building a spice name database, or on analyzing spice nomenclature from historical and linguistic-cognitive perspectives, I have set out to assemble one that would facilitate this kind of analysis. 
To introduce very briefly, I have built a database of spices and spice terminology by combing through secondary and primary literature, botanical databases, encyclopedias and dictionaries, and searching for the spices in contemporary and historical corpora. I then used a few selected features of these materials (region of origin, spreadability, etc.) and the corresponding terms (borrowed status, word formation, etc.) and looked at the set of spices as a whole, trying to find patterns and make some interesting observations about the diffusion, naming, and other aspects of spices.

% What did you do to achieve the research aims?

\section{Research Design Principles}

To achieve these aims, I first needed think of an ideal way to compile this complex sets of data, that are highly interdisciplinary in nature. From the very beginning of the design of this study, the following principles were kept in mind regarding the database of spice names: 

(1) The database must be grounded in the close study of the materials---the plants and their products---especially from a historical and botanical standpoint. Awareness of the material's physical journey will help us to contextualize some of the ways the associated names spread. Take for example the Sanskrit term referring to asafoetida (the dried oleoresin gum from \taxon{Ferula assa-foetida} et al.): \sa{हिङ्गु} \textit{hiṅgu}, which is the etymon of both Chinese \zh{興蕖} \textit{xīngqú} (\gls{MC} /hɨŋ ɡɨʌ/),
% \zh{形虞} \textit{xíngyú} (\gls{MC} /ɦeŋ  ŋɨo/)
and English \textit{hing}, but they took very different paths: while the Chinese term is a learned loan from during the spread of Buddhist scriptures on the overland Silk Road, the English word is a late \nth{16}-century borrowing via the sea trade with Mughal India. And if we study the source of the materials and learn about the plants, we will also learn that all the hing that was exported from India in the early modern Period was in fact imported from Persia and Afghanistan.

(2) The database must be thoroughly cited; every word, statement, date, or other piece of information should be carefully referenced. I have already explained the motivation and necessity behind this practice in \cref{ch:introduction}, it is enough to say that currently no one is citing sources for the names they give (except philologists), and sometimes it hard to find the motivation and inspiration behind a term. It is always a good scholarly practice to record where we found certain pieces of information, and when it comes to spice etymologies, this should help experts to verify or refute the findings on a specific item or stage.
% Example?

(3) The database should be easily expandable. Because of the limited time, it is impossible for me to include \emph{every} spice. Therefore, I try to create a pipeline, where a new material and its names can be easily added to the fold, and quickly analyzed. This in principle can also accommodate for the future inclusion of incense, perfume, and herbs, which I will mention in \cref{sec:future_study} when discussing future plans to expand on this research.

\subsection{Identification, Confusion, Adulteration, Clarification}

The ideal first step of all types of research related to spices, herbs, incense, and other aromatics is to identify the product exactly. In the case of spices and incense, this is overwhelmingly a botanical question, while in the case other exotic aromatics, such as musk or ambergris, we must involve the animal kingdom. Medical, pharmaceutical, and food industry studies are heavy on the hard sciences---chemistry, biology---but they sometimes also contain valuable information about both common and scientific names. All medical studies must start with the proper identification of the substances, in fact, there is a range of studies about various techniques on identification and differentiation \autocite[cf.]{ford_cinnamon_2019}. The reasons for this are twofold. 

Firstly, in many cases it is not a straightforward task to tell the substances apart, different spices can have very similar physical qualities. E.g., Chinese star anise \taxon{Illicium verum} and Japanese star anise \taxon{Illicium anisatum} look basically the same, but the latter is toxic, see the excellent points made by \textcite{small_confusion_1996} on the confusion of their common names. Confusions in nomenclature and identity poses a further challenge to clinical trials if the origins of a substance is not properly identified. Take for example \textcite{oketch-rabah_cinnamon_2018}, who writes on the confusion of cinnamon and cassia nomenclature and its implications in pharmaceutical research. Consider first cinnamon (\taxon{Cinnamomum verum} syn. \taxon{C. zeylanicum}). Common names include \textit{true cinnamon} and \textit{Ceylon cinnamon}. However, the cinnamon sold in the US and in the UK markets are generally not the same spice: most of the product labelled as cinnamon on American shelves is in fact cassia (\taxon{Cinnamomum cassia} syn. \taxon{C. aromaticum}) \autocite{oketch-rabah_cinnamon_2018}, which is sometimes called `fake cinnamon' or `bastard cinnamon'. In retrospects the Latin scientific name of the former makes bit more sense now: \textit{verum} means `real, true, genuine'. But why is cassia fake cinnamon? This is due to historical reasons, from when the introduction of the much cheaper cassia pushed down the cinnamon prices drastically in the \nth{19} century \autocite{wijesekera_chemistry_1978}. Most scholars consistently refer to \taxon{C. zeylanicum} as cinnamon, and to \taxon{C. cassia} as cassia but it is not uncommon in everyday language use to confound the two, especially in referring to cassia as cinnamon, out of innocent ignorance. For more detail and on the identity of cinnamon and cassia please see \cref{sec:identity_cinnamon}. Uncovering confusions from under heaps of synonyms lead us to interesting historical events that sometimes explain the vernacular names of a particular product, such as the case of cinnamon and cassia shows. 

Secondly, adulteration and contamination are rampant in the industry. Saffron (\taxon{Crocus sativus})---the most expensive spice by weight nearing the price of gold---is famous for being knocked up (and substituted) with the much cheaper flowers of safflower (\taxon{Carthamus tinctorius}). Even their names reflect these practices: although the two are very different and unrelated plants, their similar dyeing properties and constant confusion have left its mark. \textit{Safflower} the word have been influenced by the word for saffron, even if they have different origins (both ultimately from Arabic). And, on account of the adulteration, safflower have also came to be known as \textit{bastard saffron}, attested in 1548. 

We do not need to lurk modern pharmacological studies to find examples of confusion, the identity of saffron was also elusive in ancient China, where at its introduction in the early Middle Ages, it was confused with safflower, and both were casually called \zh{紅花} \textit{honghua}. It is said that Buddhist monks picked up saffron in Kashmir on their way from India to China, but the knowledge about it was not clear until the Yuan dynasty, when it was actually used and imported \autocite{laufer_sino-iranica_1919}. During Tang times, it was also connected with the---also strongly yellow---turmeric. Turmeric came first, and got the name \zh{鬱金} \textit{yujin} [yü-gold], and later saffron was named \zh{鬱金香} \textit{yujinxiang} [yü-gold-aromatic] \autocite{schafer_golden_1985}. The confusion of saffron and turmeric (and truthfully every other yellow spice used as a dye) is also observable in Classical Arabic, \textit{kurkum} `turmeric', historically also `saffron' (etymon of the word \textit{curcuma}), and the perceived ``similarity'' of Sanskrit \sa{कुङ्कुम} \textit{kuṅkuma} `saffron' did not help to clear the waters either \autocite[see]{guthrie_trade-language_2009}.

Keeping all this in mind, I feel I must lean on rudimentary botanical identification in the investigation, linking the plants and plant parts to the products and their vernacular names. This is important, as it can clear up some of the confusion when two or more products are used interchangeably, and it will highlight problematic cases from the start.

\section{Data Collection}
\label{sec:data_collection}

\subsection{Collecting the Spices}

The data collection for this project was conducted in three stages. In the first stage, I had select a set of spices that will constitute the basis for this project.

After I have assembled the set of spices, I collected information about them from encyclopedic handbooks written by experts in the plant sciences and spice industry professionals. I have made great use of \textcites{van_wyk_culinary_2014}{peter_handbook_2012}{hu_food_2005} at the start, especially when matching plant products to plants. At this stage, I have focused on the identity and characteristics of spices including geographical distribution and native habitats, especially where I saw any room for confusion. As I collected scientific names, I also recorded the common/vernacular names for 
each spice as an initial exploration, and I linked them to a botanical database that can supply further information. I have also collected information regarding their basic uses especially in traditional medicine.

Surprisingly, the abundance of synonyms is also palpable in the scientific nomenclature, sometimes one plant species has dozens of binomial \glspl{taxon}. In an attempt towards standardization of taxonomic data, collaborative efforts have sprung across numerous authoritative institutions to assemble and link their respective databases and sources. These online projects are usually run by a consortium of leading botanical institutions worldwide, among the key entities are the Royal Botanic Gardens at Kew and Edinburgh, the Missouri Botanical Garden, the Harvard University Herbaria \& Libraries, Geneva Conservatory and Botanical Garden, the Muséum National d'Histoire Naturelle in Paris, the South African Biodiversity Institute, the Australian National Botanic Gardens, and the Kunming Institute of Botany, just to name a few.

When it comes to botanical information, navigation in the ocean of scientific binomial names hiding the identity of a plant can be overwhelming. To alleviate this, I turned to a range of botanical databases for the purposes of correct identification, and information gathering. I used databases such as \gls{TPL} (\url{http://www.theplantlist.org}), which was recently superseded by the \gls{WFO} (\url{http://www.worldfloraonline.org}); the \gls{IPNI} (\url{http://www.ipni.org}); \gls{POWO} (\url{http://www.plantsoftheworldonline.org}); the \gls{GBIF} (\url{https://www.gbif.org}); the \gls{FOC} hosted on eFloras (\url{http://www.efloras.org/index.aspx}) and the \gls{BHL} (\url{ https://www.biodiversitylibrary.org/}). \gls{TPL} for instance claimed to be ``a working list of all known plant species'', now under \gls{WFO} it is ``an online flora of all known plants'' , and as such also connects different plant checklists and biodiversity databases using the nomenclatural and publishing information. In my dissertation I will frequently refer to \gls{POWO}, which contains botanical descriptions and geographic data (native and introduced habitat), besides the usual taxonomic and botanical information. 

In addition to online databases, I will occasionally also turn to reference books from the field of food technology and nutritional science, such as the \textit{Handbook of Herbs and Spices} \autocite{peter_handbook_2012,peter_handbook_2006}, and \textit{The Encyclopedia of Herbs \& Spices} \autocite{ravindran_encyclopedia_2017}. These encyclopedias, although aimed at chemistry-focused food industry professionals, also contain holistic information on the plant-based products and discuss the origins and vernacular names, besides the usual particulars on usage and medicinal qualities. It is also worth noting that various dictionaries usually mention the scientific names of plants.

Regarding traditional medicine systems, I frequently consulted modern inventories of \gls{TCM} to identify materials and extract Chinese names, including the the connecting databases of Hong Kong Baptist University: the HKBU Medicinal Plant Images Database (\url{https://library.hkbu.edu.hk/electronic/libdbs/mpd/index.html}), the HKBU Chinese Medicinal Material Images Database (\url{https://library.hkbu.edu.hk/electronic/libdbs/mmd/index.html}) HKBU Chinese Medicine Specimen Database (\url{https://libproject.hkbu.edu.hk/was40/search?channelid=44273}); and the PolyU Chinese Herbal Medicine Database (\url{https://herbaltcm.sn.polyu.edu.hk/}). Armed with the botanical knowledge, we shall have an ideally clear picture on the spices, and a firm base to connect linguistic data to.

% Hong Kong Herbarium
% % https://www.herbarium.gov.hk/en/home/index.html

\subsection{Collecting the Names}

I in the second stage, I have collected names of spices by combing through the literature; whether botanical as described above, historical, or culinary. Always, prioritizing the existing linguistic and philological treatises, of course. I have linked the collected spice names to the respective spices and the result of this is an inventory of nearly 400 spice names that link to the initial set of 24 spices.

For each spice, I tried to collect their names in three languages, and it was also my goal to records where I have found these names. Therefore, thorough citations are available in the dataset pointing to books, articles, databases, dictionaries and even Wikipedia.

furthermore I have checked every spice name I found in the published literature or online I'll check them against dictionary's to see if their use is conventionalized or not   I have also recorded the details about these dictionary entries where ever available are you collected the dates of attestation for spice name with a year century or. As in Tung dynasty molar. I have also tried to Mark where each word was it tested the specifically the title and author of the often historical books  a that the spice things are first recorded 

 B Beyond dates of a test station I have tried to attach a gloss a literal meaning to each spice name and they have also written remarks and notes on there meanings or origins and sometimes short explanation is needed to understand an existence of a night based on the botanical Lee informed spy status it in stage one populated the names with the corresponding   binomial name for the known species and indicated their native macro areas before moving on to analyze this list to spice names as a first step I have highlighted so cold default items these are the names that the spice is most commonly known as and also act as a key ID   to the rest of the alternative names besides these default names and aliases I have recorded historic and obsolete names names everything I could find that has been used or recorded somewhere 
 
 

\subsection{Annotating the names}

default/present
alias
historic
archaic
obsolete

title


status?







%  3. Methods
%  The methods of the present study can be divided into two main components, along the lines of the two main objectives. Objective A, exploring the “diffusion of spices” will involve classical philological research, augmented with the help of historical corpora, while Objective B, exploring the “language of spices” will use corpus linguistic methods.
%  3.1. Exploration
%  Before setting out on a philological journey or diving into corpora for massive data collecting adventures, we must clarify what we want to collect data about, and how we are going to compare spices and spice-words. This dissertation being a linguistic study, the words/names and their meanings pertaining to spices, the “semantics of spices” seems to be an appropriate point of commencement. Spicy word senses
%  We decided to choose word senses as the basis of comparison, and we will collect semantic data from lexical databases, such as the WordNet 3.1, also known as the Princeton WordNet of English (PWN)16 (Fellbaum, 1998; Miller, 1995), the Open Multilingual Wordnet (OMW)17 (Bond & Paik, 2012), and the Extended Open Multilingual Wordnet 1.2 (EOMW)18 (Bond & Foster, 2013). Wordnets map semantic relations between words. These semantic networks clarify multiple senses of a word and help us identify meanings useful for our investigation.
%  For example, the word `clove' has seven meanings, four as a noun and three as a verb. Three of the nominal meanings refer to Syzygium aromaticum---the fragrant tree native to the Moluccas, once also known as the Spice Islands---and the remaining one represents the cloves of a garlic (#3, not a cognate word). See Figure 3 below for an overview of the relations between the clove tree (#2), its fresh flower bud (#1), and the dried form of the bud used as a spice, whole or ground (#4).











% % Etymology box explanation: 
% % words are given in italic. 
% % meanings are given in single quotation marks only when different from the original head.
% % Variations in language names are mapped to Glottolog languoids (https://glottolog.org/glottolog/glottologinformation), for example, Middle Persian is mapped to Pahlavi.

% \section{Sources}

\subsection{Dictionaries}

References to dictionary entries are made very frequently in this dissertation, and so I made the decision to use a more compact way of citing dictionaries and encyclopedias. This can be familiar from the philological tradition, and so the reader will see (OED, ``Pepper'') or (OED, ``Saffron'') instead of (Oxford University Press, n.d.-a) (Oxford University Press, n.d.-b) and its endless permutations. This minor deviation from the APA 7th style will make the number of dictionary entries in the bibliography less oppressive, and the inline citations more reader-friendly. I will stick to this practice throughout the dissertation whenever it is deemed to be more informative, especially in case of online dictionaries. A full list of general and etymological dictionaries are listed here with complete references.

reference works

\subsubsection{English:}

\textcite{editors_of_the_american_heritage_dictionaries_american_2022} \glsxtrfull {AHD}\footnote{Accessible online at: \url{https://www.ahdictionary.com}}

\noindent \textcite{bosworth_anglo-saxon_2014} \glsxtrfull{BT}\footnote{Accessible online at: \url{https://bosworthtoller.com}}

\noindent \textcite{hoad_concise_2003} \glsxtrfull{EE}

\noindent \textcite{lewis_middle_1952} \glsxtrfull{MED}\footnote{Accessible online at: \url{https://quod.lib.umich.edu/m/middle-english-dictionary/}} 

\noindent \textcite{merriam-webster_merriam_nodate} \glsxtrfull{MW}\footnote{\url{https://unabridged.merriam-webster.com/}}

\noindent \textcite{harper_online_nodate} \glsxtrfull{OE}\footnote{\url{https://www.etymonline.com/}}

\noindent \textcite{oed} \glsxtrfull{OED}

\noindent \textcite{cresswell_oxford_2021} \glsxtrfull{WO}

\subsubsection{French:}

TLFi Trésor de la langue Française informatisé, http://www.atilf.fr/tlfi, ATILF - CNRS \& Université de Lorraine
https://www.cnrtl.fr/etymologie/safran

Sanskrit
Monier-Williams Sanskrit-English Dictionary (Monier-Williams)

Arabic:
(Hans-Wehr)
(Lane's Lexicon) 

\gls{CAD}

\subsubsection{Greek:}

\glsxtrfull{LSJ}\footnote{Accessible via the Perseus Digital Library. Ed. Gregory R. Crane, Tufts University \url{http://www.perseus.tufts.edu/hopper/}} --- \parencite{liddell_greek-english_1940}

\subsubsection{Latin:} 

% \glsxtrfull{LS}\footnote{Accessible via the Perseus Digital Library. Ed. Gregory R. Crane, Tufts University \url{http://www.perseus.tufts.edu/hopper/}} --- \parencite{lewis_latin_1879}

Post-Classical and Medieval Latin:

Du Cange et al., Glossarium mediæ et infimæ latinitatis. Niort : L. Favre, 1883-1887
http://ducange.enc.sorbonne.fr
http://ducange.enc.sorbonne.fr/PIGMENTUM

Slovakian
https://fran.si/

https://coptic-dictionary.org/

Turkish:
https://www.nisanyansozluk.com/

An Anglo-Saxon dictionary, based on the manuscript collections of the late Joseph Bosworth (the so called Main Volume, first edition 1898) and its Supplement (first edition 1921), edited by Joseph Bosworth and T. Northcote Toller

Spanish
DLE
The Royal Spanish Academy: Diccionario de la lengua española https://dle.rae.es/

https://dle.rae.es/

http://www.jergasdehablahispana.org/ ``A Spanish dictionary specializing in dialectal and colloquial variants of Spanish, featuring all Spanish-language countries including Mexico.''

https://www.spanish-translator-services.com/articles/latin-american-spanish.htm
``This is the universal and somewhat arbitrary name that is given to idiomatic and native expressions and to the specific vocabulary of the Spanish language in Latin America.''

OND Nahuatl Dictionary
Online Nahuatl Dictionary, Stephanie Wood, ed. (Eugene, Ore.: Wired Humanities Projects, College of Education, University of Oregon, ©2000–present)
https://nahuatl.uoregon.edu/
\parencite{wood_online_2000}

CAL Comprehensive Aramaic Lexicon

\url{https://cal.huc.edu/}

Malagasy
\url{https://en.mondemalgache.org/bins/homePage}


Linguistic datababes/corpora

Chinese Text Project edited by Dr. Donald Sturgeon \url{https://ctext.org}


Old Chinese phonology based on Zhengzhang 2003

PHILOLOGY

http://www.filaha.org/

https://www.fihrist.org.uk/
% \section{Etymologies}

Armenian:

http://www.nayiri.com/

% \clearpage

% \section{Black pepper}

% \begin{spice}
% \textit{Piper nigrum} L. \hfill
% \href{https://www.ipni.org/n/682369-1}{IPNI},
% \href{https://powo.science.kew.org/taxon/682369-1}{POWO},
% \href{http://legacy.tropicos.org/Name/25000013}{TROP},
% \href{https://www.gbif.org/species/3086357}{GBIF}
% \smallskip \\
% \textbf{English}: \textit{black pepper}.
% \textbf{Chinese}: 黑胡椒 \textit{hēihújiāo}.
% \textbf{Arabic}: فلفل أسود \textit{filfil/fulful aswad}.
% \textbf{Hungarian}: \textit{fekete bors}.
% \end{spice}



% \begin{etymology}
% English \textit{pepper}
% < Middle English \textit{peper} \textss{Et Mw Ah Wk} 
% < Old English \textss{Wo} \textit{pipor}, \textss{Ee Et Mw Ah} \textit{piper} \textss{Ee Wk}
% < West Germanic \textss{Ee Wo Et Mw Ah} \textit{*piper} \textss{Wk}
% < Latin \textss{Wo} \textit{piper} \textss{Ee Et Mw Wk} `long pepper, black pepper' \textss{Ah}
% < Ancient Greek \textit{péperi}, \textss{Ee Wo Mw Ah Wk} \textit{piperi} \textss{Et Mw}
% < Pahlavi \textss{Wk Et}
% < Middle Indo-Aryan \textss{Wk} \textit{pippari}, \textss{Et} \textit{pipparī} `long pepper' \textss{Ah}
% < Sanskrit \textit{pippalī} \textss{Ah} `berry, peppercorn', \textss{Ee Wo} \textit{pippali} `long pepper' \textss{Et Mw Wk}
% < \textit{pippalam} `berry, fruit of the pipal tree' \textss{Ah}
% \end{etymology}


% \begin{figure}[!hbt]
%     \centering
%     \includegraphics[width=\linewidth]{imgs/pepper.pdf}
%     \caption{Etymology of the English word \textit{pepper}, and an illustration of its approximate route.}
% \end{figure}

% \blindtext

\section{Illustrations}

Botanical illustrations are almost all taken from \textcite{kohler_kohlers_1887}, a legendary work on medicinal plants among those interested in anatomical drawings of \taxon{Plantae}. Due to the book's age, the pages reproduced here are in the public domain. %Further sources are...

All photographs of spices displayed (except where stated otherwise) are by courtesy of Christine Latour at Aromatiques Tropicales, a spice vendor in Dégagnac, France (\url{https://www.aromatiques.com/en/}). Credit is due to the photographers Felix Farmer\footnote{\url{http://www.felixfarmer.com/}} and Philippe Janina\footnote{\url{https://philippejanina.wixsite.com/photographe}}.

Plots.

\section{Note on Scholarly Conventions}

"Commonly used spellings, even if at the sacrifice of consistency"