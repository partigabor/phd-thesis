\chapter{Background}
\label{ch:background}

% 1. What is the case for the research?
% 2. What are the most important authors, findings, concepts, schools, debates and hypotheses?
% 3. What gaps exist in the literature?
% 4. Are these gaps methodological, conceptual or epistemological?
% 5. How does your thesis fill these gaps?

Knowledge and familiarity about spices varies greatly from person to person. One can live a life of actively pursuing, disseminating, and creating knowledge about spices, while others die without caring or knowing a thing about them. However, presumably both of these types of people would use and consume similar amounts of these substances, depending on which culinary or medicinal tradition they were born into. Spices are various, mainly plant-based substances that have played essential roles in human civilization for millennia. As I mentioned earlier, the assumed roles can be numerous: culinary, medicinal, cosmetic, and ritualistic, and different cultures display varying degrees of importance to different products. 
% This project involves spices, such as pepper, cinnamon, and ginger, and cloves; incenses, such as frankincense, myrrh, and sandalwood; and other aromatics, such as musk and ambergris. 
In this chapter, I will explore the different fields that have generated information about spices, review and evaluate the existing literature, and present the available and appropriate sources for investigating these materials relevant to this project. I will also introduce the theoretical framework that this thesis and its methodology builds on.

\section{Literature Review}

\subsection{On Spices}

When we hear the word \textit{spice}, our imagination rushes through far-flung tropical islands, busy seaports, lush jungles, and arid deserts; it invokes the sight of massive ocean-going ships, oriental traders, and camel caravans. A quick internet image search on the ``spice trade'' shows us antique maps in sepia and neatly arranged Moroccan spice markets in eclectic colors. We can almost smell the word \textit{spice}. These envisioned, heavily stereotypical landscapes go hand in hand with stories of exotic peoples, fantastic creatures, prized commodities, tales of exploration, and much less glorious accounts of colonization. What I described here is an exclusively Westernized viewpoint. While most of the images in our minds are distorted under the influence of romantic orientalist paintings, and tales of triumphant discoveries retold over generations, the essence of the image is very true, and much more gruesome. Arguably, the peoples living in the native habitats of a once overvalued plant species have different experiences etched in their collective memories. One could argue that Europeans imported spice, but often exported horrors. The spice trade and its romantic imagination gave birth to many books, from historical non-fiction on influential characters, such as \textit{Nathaniel's Nutmeg} \autocite{milton_nathaniels_1999}, to popular histories, such as \textit{The Spice Route} \autocite{keay_spice_2006}, and more popular science accounts, such as \textit{Fruit From the Sands: The Silk Road Origins of the Foods we Eat} from paleo-ethnobotanist \textcite{spengler_fruit_2019}. 

Today, spices are mostly discussed from a culinary point of view. The volumes of cookbooks and spice \& herb companions are almost infinite. Gastronomy professionals, chefs, food writers, and hobbyists all participate in an endeavor to introduce spices to us in a fashionable manner, creating references for home cooks and health enthusiasts. Many authors tend to attempt an overarching collection, presented in encyclopedic directories \autocite{farrell_spices_1985,craze_spice_1997,herman_herb_2015,norman_herbs_2015,lakshmi_encyclopedia_2016,oconnell_book_2016,opara_culinary_2021}. 

On a more scientific note, we find authors from the plant sciences, such as plant taxonomist and ethnobotanist \textcite{van_wyk_culinary_2014} who delivers an excellent compendium titled \textit{Culinary Herbs and Spices of the World} where he introduces dozens of aromatic plants, with a clear explanation on their uses and categorization. In her \textit{Food Plants of China} \textcite{hu_food_2005} describes hundreds of edible plants relevant to Chinese eating habits, with the precision of a true botanist. Agricultural ecologist and ethno-biologist \textcite{nabhan_cumin_2014} takes the reader on a ``spice odyssey'', with his illuminating \textit{Cumin, Camels, and Caravans}, discussing the materials in chronological steps of global trade---on the Incense trail, the Silk Road, and the spice trade.

Beyond general and introductory histories of spices, such as those offered by \textcite{turner_spice_2004}'s \textit{Spice: The History of a Temptation}, or \textcite{czarra_spices_2009}'s \textit{Spices: A Global History}, most historians and philologists approach the topic in depth, from their own areas of expertise. Culinary historian \textcite{krondl_taste_2007} compartmentalized the story of spices, and writes about Venice, Lisbon, and Amsterdam, ``the three great cities of spice'' in his \textit{The Taste of Conquest} and presents the story of spices through the vying eyes of European powers. Spices in Greek mythology are explored in \textit{The Gardens of Adonis} by an expert of Ancient Greece, \textcite{detienne_gardens_1994}, while \textcite{schivelbusch_tastes_1992}, a cultural historian discovers the social history of spices, stimulants and intoxicants in his \textit{Tastes of Paradise}. \textcite{freedman_out_2008}, a historian and expert on medieval cuisine, in his book \textit{Out of the East: Spices and the Medieval Imagination} explores how the European fascination with spices fueled the quest for new lands and colonial expansion. The initial voyages to America by Columbus, Pizarro, and others were motivated by the search for spices, and the mirage of \textit{La Canela}, a legendary valley abundant in cinnamon, equally promising to that of gold in El Dorado \autocite{dalby_christopher_2001}. One of the most valuable works for us is \textcite{dalby_dangerous_2000}'s \textit{Dangerous Tastes: The Story of Spices}. Andrew Dalby is a linguist and historian, and besides Latin and Greek he has command of other languages, such as Sanskrit and Burmese, which allows him to present the topic of spices with the pen of a truly versatile philologist and convey authentic scholarly information on spice names bridging East and West. A thought-provoking volume titled \textit{The Poetics of Spice} by \textcite{morton_poetics_2006} is a literary critical study that discusses how spices were represented in Romantic and Victorian era English literature, and how the topic connects to romantic tropes; ideologies, such as consumerism, capitalism; and ideas, such as abstinence and luxury. ``Spice is a complex and contradictory marker: of figure and ground, sign and referent, species and genus; of love and death, epithalamium and epitaph, sacred and profane, medicine and poison, Orient and Occident; and of the traffic between these terms.'' \autocite[9]{morton_poetics_2006}.

Looking beyond holistic, comprehensive tomes on the history of spices attempting to gather all of them in a single book, some commodities have already been explored thoroughly in a more concentrated approach. The history of salt \autocite{kurlansky_salt_2002}, tea \autocite{mair_true_2009}, pepper \autocite{shaffer_pepper_2013}, and vanilla \autocite{rain_vanilla_2004} are worth mentioning, and treatises on other popular substances of trade (chocolate, sugar, tobacco, etc.) are abundantly available. Even more outstanding are the works that focus their investigation on a specific area, whether it is the ``cultural biography'' of the chile pepper in China \autocite{dott_chile_2020}---retelling an unquestionably influential incorporation of a new item to a diet---or the allure of musk and perfume in the Islamic tradition \autocite{king_musk_2007}. These works contain valuable linguistic information as well, regarding the origins and spread of the names of spices, and they help us to investigate their spread and diffusion. 

Studies on specific spices are one of the most important sources for this thesis, and highly related to the project for example are the studies on the loanword status of ginger \autocite{ross_ginger_1952}, on the diffusion of chile \autocite{wright_medieval_2007}, on the identity and etymology of Sichuan pepper \autocite{austin_sichuan_2008}, and on the ``trade-language origin'' of turmeric \autocite{guthrie_trade-language_2009}. Recent advances regarding the name and identity of cinnamon and cassia in ancient vs. modern times published by \textcite{haw_cinnamon_2017}, and the Eurasian itinerary of asafoetida \autocite{leung_itinerary_2019} are crucial pieces of research in order to trace the products accurately. These and similar types of research were highly influential in the preparation of this thesis.

For an overview about the concept, function, and uses of spices in the classical Islamic periods, please refer to \textcite{dietrich_afawih_2004}; for the same in a Chinese cultural and historical context, please see \textcite[147-153]{hu_food_2005}, and \textcite{yan__2006}.

% Hansen ``aromatics (the term xiang refers broadly to spice, incense, or medicine)''


% \subsection{On Plants and Plant Names}
% \label{sec:plant_vs_spice}

% A brief section on botanical nomenclature.

% Plant names are very important 

% botanical nomenclature

% The International Code of Nomenclature for Cultivated Plants

% \autocite{gledhill_names_2008}

% \autocite{debowiak_semantic_2019}

% https://www.drugs.com/npp/allspice.html

% https://www.drugs.com/npp/asafetida.html

% Appendix 1. Technical Aspects of Plant Names in Small 2009 569

% Appendix 1. Technical Aspects of Plant Names in Small 2009 569

% pharmacognostic

\subsection{On Food and Foodways}

Literature on gastronomy and the culinary sciences is plenty, but high-quality scholarly works are fewer. Reference works include \textcite{davidson_oxford_2014}'s \textit{The Oxford Companion to Food} and \textcite{katz_encyclopedia_2003}'s \textit{Encyclopedia of Food and Culture}, and other key publications include \textcite{toussaint-samat_history_2009}'s \textit{A History of Food}. Remarkable contributions to ancient and medieval culinary history were made by Dalby, focusing on the cuisines of Rome, Greece, and Byzantium \autocite{dalby_siren_1996,dalby_tastes_2010,dalby_classical_1996,dalby_food_2003} 

In a Chinese context the definitive work is still \textcite{chang_food_1977}'s \textit{Food in Chinese Culture}, while in the Islamic tradition, medieval cuisine and recipes are explored in details by \textcite{zaouali_medieval_2007}. A few works on the culinary history of the Middle East are also results of great scholarship, including the translation of Nawal Nasrallah, who made a tenth-century Baghdadi cookbook accessible for us in the \textit{Annals of the Caliphs' Kitchens} \autocite{ibn_sayyar_al-warraq_annals_2007}, and \textcite{lewicka_food_2011} who introduced us the \textit{Food and Foodways of Medieval Cairenes}.

\subsection{On Trade}

I must point out that spices are mostly explored through their trade. And, as most spices originate in tropical Asia, our center of attention will be directed towards the continent. The term \textit{spice trade} loosely refers to the cross-cultural, economic, and diplomatic ventures of historic kingdoms, empires, and companies, agglomerating around the Indian Ocean, and other regions such as the Mediterranean, East Africa, Maritime Southeast Asia, and by land Central Asia. The history of the spice trade is one of its own, covers hundreds of years, and it is fundamentally connected with the history of globalization. Naturally, the story of spices is intertwined with trade routes and geopolitical events, involving contact between peoples, cultures, religions, ideas, and languages. See general works on economic history, such as the \textit{Spice Islands} \autocite{burnet_spice_2011}, or on political history \textit{The Scents of Eden: A Narrative of the Spice Trade} \autocite{corn_scents_1998}. Specific eras and regions related to our topic include the ancient Indo-Roman trade \autocite{sidebotham_berenike_2011,cobb_indian_2019}, the medieval Indian Ocean sea trade spanning from the Persian Gulf to China championed by Arab seafarers \autocite{hourani_arab_1975, pearson_spices_1996}, and the Southeast Asian maritime trade \autocite{donkin_between_2003,hall_history_2010,reid_southeast_1988,chen_toward_2019}. Young scholars are also doing incredible work, I would like to highlight the thesis of \textcite{hoogervorst_southeast_2012}, who combines historical linguistic and archaeological approaches in the research of Southeast Asia in the ancient Indian Ocean world. Southeast Asia was the source for Chinese spices as well, in their paper \textcite{xu__2021} detail the history of the spice trade between China and Srivijaya (and other kingdoms of maritime Southeast Asia), its rise during the Han and Tang, peak during the Song and Yuan, and decline during the Ming and Qing dynasties. They consult historical records regarding the volume of the trade and the nature of the tribute-trade relationships. For more on the spice trade from a Chinese perspective (in Chinese), refer to \textcite{yan__2007,yan__2012,yan_1619_2016}.

Besides a degree of domestication and long-standing cultivation practices, the abundance of spices today are a result of long-distance trade and cultural exchange. New advances in the field of archaeobotany concerning Roman and Islamic times for example, helps us to map the routes of the materials and trade-connections better \autocites[see][]{van_der_veen_roman_2015}{ van_der_veen_archaeobotany_2018}.

One surprising fact that I have learned from my reading, is that the Silk Road, the trade network of roads and desert pathways connecting Central Asia with China did not really feature spices; or at least not to the extent I previously believed. Valerie \textcite{hansen_silk_2012}'s well-informed book based on unearthed documents of the region show a trade that is small in volume, and much less lavish in terms of luxury goods than previously thought. Most of the trade covered short distances and whirled around everyday goods and just a minute amounts of exotic perfumes and aromatics, especially musk. Silk often acted as a currency. The word \textit{spice} only occurs two times in her book as she highlights the most traded products in each period. This is not to say that spices did not exist here at all---we know that many spices were introduced to China on the silk roads, and that traveling merchants carried pepper and propagated other spices via their journeys---but that the bulk of the spice trade between East and West did not happen overland.

% or the Europeans seeking dominance over the Spice Islands shall be discussed in the thesis where necessary.

Many of the contemporary works I mentioned that trace the initial steps of certain spices and other foodstuff relay accounts from primary sources. For example \textcite{spengler_spices_2019} writes that the black pepper of tropical India is first mentioned by Chinese sources in during the Han dynasty (202 \BC{}--9 \AD{}; 25--220 \AD{}), in the \gls{Hou Hanshu}, quoting \textcite[374]{laufer_sino-iranica_1919}. I have noticed that in a lot of cases, we can thank a few giants for the reports, legendary scholars whose research we still use and reference. These are the people who laid down the groundwork for future studies by their hard work and language skills, including Berthold \textcite{laufer_sino-iranica_1919} and his invaluable \textit{Sino-Iranica}, which catalogues ``Chinese contributions to the history of civilization in ancient Iran, with special reference to the history of cultivated plants and products''; and Edward H. \textcite{schafer_golden_1985}, and \textit{The Golden Peaches of Samarkand}, which lists luxury exotica that reached the Tang court, exploring cultural interactions with other regions. I would also like to mention Isaac \textcite{burkill_dictionary_1935}, who recorded every economically important plant and mineral under the sun of the Malay Peninsula, annotated with local names, traditional knowledge, and regional historical significance in his monumental \textit{A Dictionary of the Economic Products of the Malay Peninsula}. Their command and knowledge in history, sinology, and botany is immeasurable.

In the study on spices, incense, and aromatics through the tools of Semitic philology, I should mention the recent addition of \textcite{amar_arabian_2017}'s \textit{Arabian Drugs in Early Medieval Mediterranean Medicine}, and \textcite{lev_practical_2008}'s \textit{Practical Materia Medica of the Medieval Eastern Mediterranean According to the Cairo Genizah}, but we cannot leave out \textit{Domestication of Plants in the Old World} by plant geneticist \textcite{zohary_domestication_2012}, which supplies a great overview of the agronomic development of the region, or \textit{Duke's Handbook of Medicinal Plants of the Bible} \autocite{duke_dukes_2008}.

\subsection{On Chemistry, Medicine, and Healing}

Besides history, archaeology, and botany, progress in spice related research in recent times are predominantly from the field of medicine. There are uncountable pharmacological---clinical and in vitro---studies on the effects of various medicinal plants \autocite{boy_recommended_2018}, and many of them are motivated by food and nutritional science research, such as \textcite{baker_impact_2013}'s survey on the effects of cooking with and ingesting cinnamon, nutmeg and cloves. In the dissertation I will try keep away from deep deliberations of scientific treatises from medical, biochemical, and pharmacological journals as much as possible, however, I might  comment on issues related to folk uses and traditional knowledge if it is relevant for the greater cause.

% Further moving away from history, 
I must briefly mention the fields closely knit with the food industry: chemistry and pharmacology. The authoritative \textit{Handbook of Herbs and Spices} \autocite{peter_handbook_2012} and \textit{The Encyclopedia of Herbs \& Spices} \autocite{ravindran_encyclopedia_2017} are for industry professionals. These works detail the physical and chemical properties of the materials, and the plants and their products are described in detail. Besides botanical information, the plants' chemical compounds and volatile oils are in focus, but general knowledge about the origins, names, uses, and functions are also presented. The chemistry of spices is an interesting topic, scientific and popular books were both published on it.\footnote{For a highly visual and novel take on a book introducing the chemistry of spices, see \textcite{farrimond_science_2018}} The science behind how spices work is a fascinating one, there are two questions we should pose, one: ``Why are spices spicy?'', and two: ``Why humans like spices?''. The answer to the first question is that the pungency we feel---a mild rush of heat or minutes of tingling lips---is a in fact a toxic shield, it is the plant's evolutionary response to herbivores, bugs and pests \autocite[21]{turner_spice_2004}. However, this is not a crucial component in the organism's life cycle; these substances (the volatile oils causing flavor and pungency) are so-called secondary metabolites, they are insignificant to the plant's biology \autocite[18]{parthasarathy_chemistry_2008}. The heat to the chili is effectively the same as thorns to the rose. The spiciness of a spice is a weapon, and while bugs and insects would run amok trying to have a taste of the fruit of \textit{Myristica fragrans} (the tree of nutmeg and mace), it made humans---quite ironically---sail to the end of the world to find it. No obstacle was great enough to stop mankind's appetite for fragrant, pungent, and spicy flavors. In answering the second question, we can expect that if the spiciness of spices has a Darwinian explanation, the human desire for them should also sound like one. \textcite{sherman_darwinian_1999} in their influential, and aptly titled article \textit{Darwinian gastronomy} claimed that spices taste good because they help us fight hostile bacteria and microorganisms responsible for digestive issues such as food poisoning; they are beneficial for our health. The authors also compared cuisines of the world based on how much spice they use in their everyday cooking. The piquancy of some capsicums is essentially an irritation, \textcite{spence_why_2018} explores, why do so many people find the ``oral burn'' so appealing, \textcite{carstens_it_2002} investigates the neural mechanism of oral irritation from spices and carbonated drinks, and we can learn about pungency and personal preference from \textcite{prescott_pungency_1995}. The antibacterial and antioxidant effects of spices are known for millennia, and recent research \autocite{billing_antimicrobial_1998,nilius_spices_2013,yashin_antioxidant_2017} shows that the old wise ones were not at all wrong compiling their \glspl{materia medica} and \glspl{bencao} to guide future generations on herbal healing. Of course, there were plenty of exaggerated claims on the potential healing effects of some products, from them being an antidote for snake venom to the cure for death itself.

\subsubsection{Materia Medicas, Pharmacopeias, Bencaos}

\Gls{materia medica} (Latin for `medical material') refers to a descriptive collection of knowledge about substances---plant-based, mineral, or from an animal source---with therapeutic properties, usually in the form of a book, often illustrated. It is a term from the history of medicine, named after the highly influential book of Dioscorides, a Greek physician and pharmacologist from the first century \AD{}. The term \gls{pharmacopeia} is closely related to this, but this refers to a more technical book that contains directions on how to combine different materials for effective healing remedies. Basically, it is a drug making manual.

\Gls{bencao} \tc{(本草)} [measure word for books-herb] is essentially the Chinese equivalent of materia medica. It refers to compilations of classical Chinese medicinal literature. The \gls{Shennong} from circa 200 \AD{}, although lost, is generally considered the first \autocites[see][]{nugent-head_first_2014}{yang_divine_1998}. A great explanation of the \gls{bencao} tradition can be found in the introduction of \textcite{wu_illustrated_2005}'s \textit{An Illustrated Chinese Materia Medica}, and \textcite{zhao_concise_2018} offers a brief overview on the classification of \gls{bencao} literature, and how it connects to traditional Chinese medicine. The most famous \gls{bencao} however, is the \gls{Bencao}, sixteenth-century Chinese encyclopedia of materia medica and natural history written by Li Shizhen. It is probably the most important book of \gls{TCM}, building on the knowledge of earlier Chinese pharmacological works. It is often translated to English as the \textit{Compendium of Materia Medica}, and the first complete English translation project is currently under way headed by Paul \textcite{unschuld_first_2022}. 
A modern, scientific example for a materia medica style compilation would be \textcite{duke_crc_2002}'s \textit{CRC Handbook of Medicinal Spices}.

In the Arabic context on the other hand, we must acknowledge the advances of Islamic medicine, and the fruitful decades of the Islamic Golden age that saw many scholars publish extensively, forwarding the tradition of the Greeks, building on the works of Dioscorides, Galen, and Hippocrates. The writings of philosopher and polymath Ibn Rushd (Averroes), physician and pharmacologist Ibn Juljul, botanist Ibn al-Bayṭār, and alchemist Abū Bakr al-Rāzī were all influential in the history of Western medicine and pharmacology. Maybe the most prominent of all was Ibn Sīnā (c. 980--1037; latinized as Avicenna) inspiring many future scholars for over centuries, such as Thomas Aquinas (1225-1274) \autocite{smith_avicenna_1980}. His book \gls{Qanun} completed in 1025 was used as a standard textbook at universities up to the seventeenth century \autocite{musallam_avicenna_1987}. Scholars still discuss him and his contributions \autocite{sajadi_ibn_2009}, and compare his findings with recent pharmacological studies. For example, on the traditional uses and health benefits of saffron \autocite[see][]{hosseinzadeh_avicennas_2013}.

\subsection{On the Role of Spices Through Time}
\label{shift}

I must also touch on the change in meaning on what spices once were, and what they are now. It can be now clear that in the past spices were more valued for their ceremonial or medicinal use, but I would like to make the shift in usage explicit. 

For example, the ancient Romans imported and used cinnamon in large quantities, but they did not eat it or cook with it. They treasured it as incense and medicine instead. It is often repeated that emperor Nero have burned (as incense and offering) a year's supply of Rome's cinnamon on his wife's funeral (whom probably he himself have killed) in 65 \AD{} \autocite[437-438]{toussaint-samat_history_2009}. Even if we stopped burning cinnamon, is not because of these practices disappeared---the Catholic Church still uses 50 tons of frankincense a year \autocite{ash_why_2020}---it rather seems that most materials in question gradually gained more favour for their culinary appeal. 

In the notion that the role of spices changed over time, there is a universally observable pattern: the gradual shift from their relevance in medicine towards gastronomy. \textcite{freedman_health_2015} writes on social and cultural implications of the role in spices and their importance in health and wealth during the Middle Ages. The shift is mainly due to the emergence of modern medicine and the marginalization of traditional folk medicine, especially in developed, western societies. What can be more telling than the term ``alternative medicine'', clearly indicating the switch: what was the ``only'' medicine once, is now a secondary (and sometimes frowned upon) option, as opposed to just ``medicine'' or in some places ``Western medicine''. In many cultures with strong roots in folk healing, the widespread use of medicinal plants, herbs, and spices are thriving and in recent years these practices are even gaining international popularity. We could think of \glsentrylong{TCM}, the Indian Ayurveda, or the Indonesian practice of \textit{jamu}.\footnote{\textit{Jamu} is the name for the traditional medicine of Indonesia, encompassing practices or herbal healing with Javanese origins, usually in the form of mixing ingredients in drinks and potions. For more, see \textcite{beers_jamu_2012}} Besides this well-known shift regarding spices and the healing factor, it is important to point out that in the past the line between food and medicine were much more blurred, this can still be observed for example in modern Chinese food therapy, \tc{食療} \textit{shiliao}, rooted in ancient dietetic traditions \autocite{engelhardt_dietetics_2001}.

\subsection{On Food and Language}

One of the best examples for a linguistic study related to gastronomy is \textit{The Language of Food: A Linguist Reads the Menu} by \textcite{jurafsky_language_2014}. Dan Jurafsky, a computational linguist and authority in the field of Natural Language Processing (NLP), explores our connection to food and eating in a series of interesting studies. From tracing the historic and linguistic origins of ketchup, macaroni, or salami, to what the wording of a restaurant menu can tell us about prices. From a Chinese perspective, food and menus are explored by \textcite{yao_chinese_2019}, while the topic of fruit-words is presented by \textcite{depner_chinese_2019}. 

This thesis will involve sensory words---nouns, verbs, and adjectives of gustation, olfaction---surrounding spices and other aromatics, and in this aspect, previous studies of linguistic synesthesia will definitely prove useful \autocites[see][]{huang_linguistic_2019}{zhao_directionality_2019}. Some cognitive studies on sensory information have been conducted involving spices, most interesting are the ones that explore cross-modality relations. For example, and fMRI experiment concluded that reading words with strong olfactory associations, such as `garlic', `jasmine', or `cinnamon' activates the olfactory regions of the brain \autocite{gonzalez_reading_2006}. Another unique study looked at the possible corresponding sound attributes to spiciness/piquancy, and a series of experiments found that fast tempo, high pitch, and distortion are indeed linked to the sensation \autocite{wang_sounds_2017}. On a more linguistic note,  \textcite{zhong_sweetness_2020} explored taste, examining the sensory lexicon around the realm of desserts. They showed that ``mouthfeel'', a multi-sensory concept plays more important role than the quality of ``sweetness''. \textcite{bagli_tastes_2021}'s \textit{Tastes We Live by} is a very recent publication that deals with the linguistic conceptualization of taste in the English language.

\section{Research Gap}
\label{sec:research_gap}

I have started this chapter with discussing the literature on spices through the eyes of different disciplines. I mentioned gastronomy, botany, history, trade and economics, and after a brief touch of classical medicine I have circled back to philology, and finally landed on research combining language and food, and the sensory modalities. So far, we saw that studies on spices---specific or in general---are available, most notably in the form of historical works focusing on some aspect of the spice trade or tracking the story of the material itself. Besides history, the availability of literature from food and nutritional science, biology and medicine is satisfactory, quenching the need of industry professionals. In this field we see a more rapid development, new studies and findings are relatively frequent, especially about popular spices. 

What we also have seen is the obvious lack of linguistic studies themed around spice. A handful of scholars have investigated questions related to language, almost exclusively from a historical linguistic point of view---trying to unearth etymologies. The few available findings however are not collected, knowledge on spice names and related terminology is found sporadically in many disciplines. In the face of such scarcity of linguistic studies on spice terminology it is not surprising that the \textit{Handbook of Herbs and Spices} of \textcite{peter_handbook_2012}---a standard reference work for chemistry and food industry professionals---often relies on an online blog to list spice names! This online blog created in the early 2000s is a personal website of one Gernot Katzer, who currently rules over the internet with his exhaustive collection of spice information, also including spice names in numerous languages. \textcite{katzer_gernot_2006} supplies a massive amount of valuable information to the public, but his lists on spice names are often inaccurate, and---since he is an individual writing about his own travels and empirical experiences and not aiming at academia---sparsely cited.

In the problem statement of the previous chapter, I have briefly mentioned that in my opinion, the lack comprehensive publications regarding spice names causes a deficit of understanding among authors who write about spices. Take for example the very recent \textit{Culinary Herbs and Spices: A Global Guide} by \textcite[11]{opara_culinary_2021}, where the authors in an attempt to give the Hindi name for allspice, write \textit{``Kebab Chini''}, which is the Hindi name for cubeb pepper (\textit{Piper cubeba}), \hi{कबाबचीनी} \textit{kabābcīnī}, a completely different spice. The problem cannot be better illustrated than the examples below.

\subsection{Faulty Claims}

The following is a collection of erroneous, incorrect, inaccurate, or misleading quotes where historians, botanists, and food writers failed to look up the names of the very subjects of their treatises, supplied with my comments.

\begin{itemize}
    \item ``Cinnamon is derived from the Greek word for spice and the prefix `Chinese'.'' \autocite[10]{czarra_spices_2009}---a popular folk etymologization, with no proof. 
    %The Greeks in turn got the word from the Phoenicians, who were most likely involved in sea trade with Eastern caravan routes controlled by the Arabs.
    \item ``The name saffron is derived from \textit{zafarán} or \textit{za'fran}, the Arabian word for yellow.'' \autocite[124]{van_wyk_culinary_2014}---it does not mean yellow, but it was conflated with the word for yellow by Europeans (also, ``Arabian'' is not a language).
    \item ``The name [of saffron] comes from the Arabic for `thread'.'' \autocite[422]{mcgee_food_2004}---it does not, the Arabic word is a loanword from Persian, and it only means `saffron'. 
    \item ``Pliny named the plant \textit{coriandrum}, from the Latin for bug, \textit{coris}. But it has other names---\textit{cilantro} (in the United States and Latin America), \textit{dhana} [sic!] (in India), \textit{Chinese parsley} (in China, presumably).'' \autocite[87]{oconnell_book_2016}---an ill-informed presumption and goes against common sense. \textit{Chinese parsley} is a name in English, not in Chinese.
    \item ``In Sanskrit, black pepper is known as \textit{maricha} or \textit{marica}, meaning an ability to dispel poison, and it is taken to aid digestion, improve appetite, ease pain, and to cure colds [...]'' \autocite[3]{shaffer_pepper_2013}---the fact that the author wrote ``\textit{maricha} or \textit{marica}'' shows that she does not know that these are two different ways to transliterate one Sanskrit word (which has no such meaning she described).
    \item ``The name `paprika' came from the Greek term for black pepper, \textit{peperi}.'' 
    Other name changes occurred as the spice moved through regional languages such as Greek, in which it is called \textit{piperia}.'' 
    \autocite[103]{czarra_spices_2009}---while not entirely wrong, many steps were skipped in this explanation, making it difficult to evaluate.
\end{itemize}

There are fallacies not only regarding the names but the circumstantial information regarding the nature and history of the spices and the spice trade as well, ranging from failure to distinguish between spices and spice mixtures, to confusing cities and travelers important in the spice trade.

\begin{itemize}
    \item ``The main distinction between the Indian and Japanese curries is that the Indian version uses a combination of spices, while Japanese \textit{karē} is made with curry powder.'' (https://www.tasteatlas.com/kare)---absolute nonsense, curry powder is a combination of spices as well.
    \item ``And Malacca (Singapore today), a port poised at the gateway to the oceanic routes to Europe, [...]'' \autocite[vi]{hill_contemporary_2004}---this statement is horribly wrong, Malacca is not Singapore.
    \item ``Chinese traveller Sulaiman visited Kerala coast—recorded the black pepper cultivation and trade with China.'' \autocite[3]{ravindran_black_2000}---Sulaymān was not a Chinese traveler, the excerpt talks about Sulaymān at-Tājir, a ninth-century Muslim merchant from the Sassanid port of Siraf.
    % \item The Molucca Islands, the original source of nutmeg and cloves, the name of these islands, known as Moluku to the Arabs, meant `land of many kings', an apt description, since there are over 17000 Molucca Islands \autocite[63]{czarra_spices_2009}---Arabic \textit{mul\={u}k does indeed mean `kings', but the 17,000 is a number for whole Indonesia}
\end{itemize}

% % https://en.wikipedia.org/wiki/Quatre_%C3%A9pices
% Quatre épices is a spice mix used mainly in French cuisine, but can also be found in some Middle Eastern kitchens. Its name is French for "four spices"; it is considered the French allspice.[1] 

% Czarra 31
% Cloves were praised early in the Sanskrit literature of
% India. They were called katukaphalah, `the strong scented'.

% Czarra 32
% The Greeks used the word `miser' to apply to someone who counted cumin seeds.


These few selected lines can make it feel like we are doomed if we look for accurate spice names and origins in the spice literature, but the situations is not that bleak. I merely wanted to show that in many instances, there is simply no awareness or effort to supply the reader with the correct information, especially when it comes to names and etymologies.

Up to date, a comprehensive study on spices from a linguistics perspective is lacking. The information already out there is sporadic and unorganized, and as I have introduced above it was botanists, historians, chefs, and historical linguists who contributed to the research on aromatic products, their origins, and their place in the human culture and lexicon. In a few cases, findings happen to be misinformed, thanks to some authors making presumptions along erroneous lines, which only adds to the confusion. This is bound to happen when botanists attempt venture into the lands of etymology, or when food writers choose to sail the high seas of historiography. For a good illustration of this problem, see the criticism of \textcite{haw_cinnamon_2017} on \textcite{austin_sichuan_2008}'s attempt to trace the etymology of \textit{fagara} (Sichuan pepper) where the authors with a background in botany have made questionable assumptions related to Classical Arabic phonology and morphology. I must be careful and not make similar mistakes, never give in to the temptation of baseless speculations, especially outside the realm of linguistics and philology. With that being said this dissertation would fill the gap that exists regarding research on spice terminology.

% and about the language of spices? sensory stuff?

Beyond my proposal to fill this gap with a study attempting to group and categorize aromatic materials of the spice domain, I also aim to analyze the diffusion of spices informed by tracing the journeys of loanwords and wandering words of three languages: English, Arabic, and Chinese. The quest for exploring patterns of spice diffusion and spice terminology could yield new insights and open possibilities for future research. Furthermore, and analysis of spice nomenclature based on linguistic-cognitive features has not yet been made and constitutes an original approach. 

\subsection{Research Questions}
\label{sec:research_questions}

I will now try to formulate the questions I aim to answer. The first two questions arise from the investigation on the ``diffusion of spices'' and are more related to the philology component of the thesis. The third questions is more related to the corpus linguistic and cognitive component of the study, investigating the ``language of spices''.

\begin{itemize}
    \item \textbf{Q1} Does the propagation of \glspl{wanderwort} within the domain of the spice trade follow the diffusion of the materials?

    \item \textbf{Q2} Is there any underlying pattern behind the mechanisms of spice diffusion, considering both the materials and the nomenclature?
    
    \item \textbf{Q3} Is there any influence on the naming spices, in terms of sensory words and synesthetic properties?
    
    % \item \textbf{Q4} Do the presence or absence of various spice related lexical categories in a language show their level of embeddedness in a culture?
    
    % \item \textbf{Q5} Would the different patterns of spice name propagation and linguistic-cognitive characteristics correlate or show differences in any way?

\end{itemize}

% The fifth research question is about comparing the findings of Objectives A \& B, looking for patterns and correlations. For each question, our expectations, assumptions, and hypotheses are discussed under section.

% Aiming at a similar direction, we would position this thesis to manifest itself on the crossroads of language, culture, and cognition, using corpus linguistic and philological methods, and explore spices from previously neglected perspectives.

%%%%%%%%%%%%%

% 1. The scaffolding upon which your thesis is built. Think of the theoretical framework like a toolbox that you build from a whole range of available tools.
% 2. What theoretical concepts are used in the research?
% 3. What hypotheses, if any, are you using?
% 4. Why have you chosen this theory?
% 5. What are the implications of using this theory?
% 6. How does the theory relate to the existing literature, your problem statement and your epistemological and ontological positions?
% 7. How has this theory has been applied by others in similar contexts? What can you learn from them and how do you differ?
% 8. How do you apply the theory and measure the concepts (with reference to the literature review/problem statement)?
% 9. What is the relationship between the various elements and concepts within the model? Can you depict this visually?

\section{Theoretical Framework}


% At the start of our investigation on the words of the spice domain, an outline of the theoretical background will be presented. First, we will approach the topic by introducing the concept of \glspl{wanderwort} and complement it by briefly discussing phenomena from the field of language contact: lexical borrowing, loanwords, and calques, as the core of the theoretical framework. Secondly, we will look at patterns of \gls{wanderwort} propagation as lexical innovations, and discuss the adoptability of established theories, such as Everett M. Rogers (2003)'s diffusion of innovations---focusing on its connection to underlying mechanisms of language change (Labov, 2001), and existing models for diffusion of variation and lexical innovation. Thirdly, we will look at how the adoption of Anna Wierzbicka (1996)'s semantic primes can be beneficial in our attempt to explore the language use surrounding spices and aromatics, considering cognition and the sensory domains. Lastly, we will connect these ideas and present a novel approach for researching cultural and linguistic contact, based on patterns of Wanderwort propagation and linguistic-cognitive perspectives of spice words.
% Our position is that \glspl{wanderwort} of the spice domain are an ideal topic of choice for this research, due to the fact that parallel to the names and words, we have annexed materials as well. The diffusion of the physical products as exotic, novel substances constitute a non-linguistic, and the diffusion of names and words constitute a linguistic type of innovation---both observable in historical and corpus data. This will be our main approach in the present study. 

% \begin{itemize}
% \item wanderwort
% \item loanword haspelmath
% \item prototype theory
% \item sensory modalities (higher senses-lower senses)
% \end{itemize}

\subsection{On Wandering Loanwords: \textit{Wanderwörter}}

Terms of the spice domain are often loanwords, likely to be \glspl{wanderwort} meaning wandering words. A Wanderwort (also known as Kulturwort ``culture word''), itself a loan from German, is ``a word borrowed from one language to another across a broad geographical area often as a result of trade or adoption of newly introduced items or cultural practices'' (Merriam-Webster, n.d.). We can observe this linguistic curiosity typically on names of foodstuff, plants, animals, metals, and other artifacts, such as copper, tobacco, potato, tomato, lemon (Trask, 2000, p. 366), materials which spread significantly due to trade. Further examples are numerous: cumin, ginger, orange, pepper, silver, soap, sugar, wine, the most famous example being tea.
The case of tea is well-known, for its names have multiple origins in different Sinitic languages. Mair and Hoh (2009, pp. 262-268) identifies three groups of names for tea: te, cha, and chai. Mandarin and Cantonese use cha, while te is from Hokkien, a Southern Min language variant.
Tea trade was prevalent at the port of Xiamen (Amoy), especially with Europeans after the mid-1500s, while land routes such as the Tea Horse Road and the Silk Road already exported tea for centuries from Yunnan and Sichuan, the homeland of Camellia sinensis. Eventually both te and cha entered the English lexicon, but more important is that almost all other languages adopted either of these words for tea (Mair \& Hoh, 2009) in the wake of merchant ships and trading caravans. Depending on geopolitical circumstances, the maritime or continental name variants---\textit{te} and \textit{cha}---spread so, that in a drastic oversimplification we might say: it is \textit{te} if transmitted by sea, \textit{cha} if transmitted by land.

This linguistic distribution is not unique to tea, genealogies of names for other consumables with similar global stories are observable in many other \glspl{wanderwort}, such as \textit{chili}, and `pepper'. Both the far-reaching journey of the Nahuatl word \textit{chīlli}\footcite[chilli \link{https://nahuatl.uoregon.edu/content/chilli}]{ond} from the Aztecs of Mexico, and the ancient Sanskrit term for the Indian long pepper, \textit{pippalī}\footcite[pippala \link{https://dsal.uchicago.edu/cgi-bin/app/macdonell_query.py?page=163}]{macdonell_practical_1929}, have intriguing stories to tell for linguists, historians, or anybody else; see Dott (2020) for the chili pepper's ``cultural biography'' in China, and Shaffer (2013) for the story of black pepper. This becomes especially apparent when we think of the compelling fusion in the term `chili pepper'. One of the objectives of this study is to address peculiarities arising from complex histories of spice names just mentioned above, and while trying to provide answers, a holistic view of the spice domain should be kept in mind. An overarching linguistic study of substances with similar features has not been conducted and might reveal new knowledge and patterns yet to be found.

The instance of tea as a Wanderwort is relatively recent, and thus we are able to reconstruct the steps of diffusion. Nonetheless, the origins of \glspl{wanderwort} are often obscure, and to some extent enforcing this connotation some scholars (chiefly Indo-Europeanists) still ``borrow'' this term to refer to a specific group of loanwords, where transmission through one or more unknown languages is suspected, and/or the donor language is uncertain, as Michiel de Vaan writes (Mooijaart \& Wal, 2008, pp. 199-201). In his short definition, Trask (2000, p. 366) fails to mention this latter issue with \glspl{wanderwort}. Another concern we must also remember is that `wandering words' are not always easily distinguishable from loanwords. In fact, in the mid-19th century, the beginnings of the Indo-European tradition of reconstructing a proto-language was briefly misguided by the failure to recognize and separate inherited lexical items from introduced ones (Polomé, 1990, p. 137). In hindsight, \glspl{wanderwort} should be considered a subgroup in the general category of loanwords. Appealing to this idea, Polomé (1990) renders it in English `wandering loanwords'. One could argue that the main problem with Wanderwort research is that sometimes---to put it bluntly---they are just too old to trace with surety. There are hardly any means for uncovering origins and deciphering etymologies beyond a certain time-depth. 

% On innovation
% Everett Rogers' popular 1962 theory, the diffusion of innovations involving constant five groups of people: ``innovators'', ``early adopters'', ``early majority'', ``late majority'', and ``laggards''; illustrated that the rate of adoption in every manifestation of innovation, be it a


% A number scholars worked towards an understanding of the underlying mechanisms of language change, including Labov (1999, 2001, 2010) and a multitude of models were proposed to explain linguistic diffusions of various types, the focus ranging widely from dialect geography (Bailey et al., 1993, p. 77; Trudgill, 1974) to phonological/sound change, in a concept named lexical diffusion by Chen and Wang (1975). These areas of research are filled with forever ongoing debates between linguists, but the ubiquity of S-curves are noticeable (Blythe & Croft, 2012). J.K. Chambers writes that ``the S-curve has been observed in diffusions of all kind including technological advances and epidemics, as well as linguistics changes […]'' (Chambers & Schilling, 2013). Jiang et al. (2021) showed that modern internet neologisms also tend to follow the initial stages of a sigmoid, that is: slow start, rapid rise/spread, followed by a slowing period of conventionalization (before most contemporary neologisms decay). Neologisms and loanwords share some properties, one being that once they reach saturation and replace an existing term or gain acceptance on their own, we consider them part of the lexicon.

% When we treat loanwords as certain type of innovation, we have to remember that contact between speech communities are in most cases constant: languages and dialects ``do not exist in a vacuum'' (Hock & Joseph, 2009). Loanwords can replace existing words, therefore they also compete in the arena of linguistic change, and the integration depends on the circumstances of the contact, as well as the lexical situation prior. Historical and social factors are heavily in play, especially that in our case the products of the spice trade revolve around themes beyond mere discovery and trade. Cultural contact was often ensued by exploitation, colonization, and imperialism. Philip Durkin mentions need and prestige as the two main motivations for lexical borrowing (Hickey, 2020, pp. 172-173). Would we see a pattern for the motivation of borrowing depending on the directionality of the trade? Would the propagation of \glspl{wanderwort} match the trails of the trade networks? And most importantly, would the diffusion of spices and the propagation of their wandering names follow the S-curve? Questions are arising and they shall be strategically composed, in order for us to uncover answers. They will be collected and presented after reviewing the literature.

% \glspl{wanderwort}, by definition, are related to new items, ideas, materials, and practices. In other words, they are related to innovation via trade, cultural contact, and exchange, and they provide a wonderful window to observe linguistic contact and (ex)change. The adoption of new, previously unknown aromatics and spices for medicinal, culinary, and/or religious and ritualistic purposes immediately ushers the speakers of the speech community to adopt vocabulary related to the new phenomena if they do not already possess ``appropriate'' ones in their lexicon. Or to simply put, people need to call things by a name and describe them. In our investigation, we plan to look at the ``speed'' of diffusion, that is, rates of lexical adoption and adaptation in the spice domain, to see if the names of spice items arrived prior to the substance in question (often by accounts of sailors, merchants, scholars, and pilgrims); or else, if the name travels together with the product; or a language adapts a name after the appearance of the product. If the availability of data allows, we will be able to see how patterns of geographic and linguistic diffusion converge or diverge both in space, and time. 

\subsection{On Sensation}

The spread of spice related words is not limited only to the nomenclature of new products---which presumably embraces foreign words, loanwords, and calques (loan translations) in high concentration---but also to the different categories of terminology that describe the substance. New sensations invoke new words, (or give way to new meanings of existing ones) driven by the need to express and describe taste, smell, color, and texture. With the help of corpora, we can collect sensory words surrounding aromatics, and explore how humans discuss flavor, fragrance, heat, and spiciness. We shall see if the function of modifiers and adjectives are there to help to identify spices, and we shall see if patterns of spice diffusion (directionality, quantity of transmission) differ based on sensory properties. With enough data we will be able to examine tendencies that pertain to land-based or maritime trading routes.

Apart from words related to the sensory domains, mainly olfaction and gustation, it might be worth examining the ``language of spices'' from another angle as well. By comparing linguistic behavior of various spices, I expect to see different degrees of adaptation, ranging from just the basic existence of the borrowed/translated name for a newly introduced substance, to highly dynamic and versatile presence in everyday language use. Beyond the (a) name \textit{pepper}---for example---we can observe and effortlessly identify other ``degrees'' of linguistic use: (b) the incorporation of words for the sensations induced by the spice or other characteristics of its nature (\textit{peppery}); (c) cognate verbs of seasoning and cooking (\textit{to pepper}); and (d) denominal metaphors and idioms (\textit{to have pepper in the nose}). I hypothesize that the presence or absence of spice related terminology and derivationally related words in a language correlates with the levels of acceptance and familiarity in a society, i.e., the language use reflects the degree of adaptation of the product, and ultimately its embeddedness in the culture. For example, if a language names a color after \textit{coffee}, \textit{cinnamon}, or \textit{saffron}, it is a good guess that a large portion of the society is familiar with the product. This direction is one of the future goals, as stated in \cref{sec:future_studies}, and maybe this thesis will be a good base for such a study. Furthermore, we can ask questions, such as: Would considering different spices affect the categories of linguistic presence discussed above? Would patterns of spice diffusion make a difference when looking at linguistic ``behavior''?

Spice names and collocates invoke sensations from different sensory domains, and are strong ``carriers'' of synesthesia. Zhao (2018) writes that linguistic synesthesia ``describes a situation where perceptions in different sensory modalities are associated in both perceptual experiences and verbal expressions''. Evidence for cross-sensory conventions occur in the nomenclature of spices as well, they are sometimes obvious, but cumbersome to unearth and explain the reasons.

In Spanish, the word for pepper is \textit{pimienta}, from the plural form of Latin \textit{pigmentum}, which is ``a material for coloring, a color, paint, pigment'', with an additional, transferred meaning in post-classical Latin, ``the juice of plants'' (Lewis \& Short, 1958, p. 1375), and a Spanish etymological dictionary indicates ``plant juice, food seasoning'' (Gómez de Silva, 1985, p. 415). The emergence of this meaning of the word for `pigment' is believed to be a due to the observed Mesoamerican practice of using dried chilies of the genus Capsicum for seasoning dishes, which also worked as an organic food coloring substance. After Christopher Columbus returned to from the New World with \textit{Capsicum annuum} in 1493, the impact of the new sensation was so strong that it technically replaced the existing Catalan word---\textit{pebre}, the derivative of the already mentioned Sanskrit etymon, \textit{pippalī}, via the Latin piper, referring to black pepper (\textit{Piper nigrum})---with \textit{pimienta}. But, if pimienta is now black pepper, then what is the Spanish word for chili pepper, the fruit of capsicum? Well, it seems it is \textit{chile} in Mexico, \textit{pimiento} in South America, but it is not that simple. Further adding to our perplexity, \textit{pimiento} is masculine, \textit{pimienta} is feminine in Spanish. And what is \textit{paprika}, then? We start to see the piquancy of the problems, and the confusion is no less clear in Spanish than in English. And, to raise the level distress a bit more, it is worth mentioning that chilies in Hindi are named \textit{mirch}, which in turn comes from Sanskrit \textit{marica/marīca}, the word for ``pepper-shrub, pepper-corn; black pepper'' (Macdonell, 1929b, p. 219; Monier-Williams, 1899, p. 790).

The goal of these examples on a few spicy \glspl{wanderwort} was to demonstrate the chaos surrounding conventions and common names of occasionally unrelated plants and their fruits, berries, and seeds; their confluence with historical and geopolitical developments. We hope that a systematic overview of the literature and methods combining corpus linguistics and philology will not only help to untangle the threads of vernacular names ungoverned by rules, but also gain insight into the connection between spice names and sensory domains.

The most exciting part of this novel approach is to see the ``attitude'' of different cultures. Are the properties of spices universal? Would speakers belonging to the Chinese cultural sphere describe a certain spice with the same tastes as Arabic speakers? Turmeric is often described as bitter in English; would other languages do this as well? Cross-cultural and cross-linguistic perspectives in research on cognition, pragmatics, and semantics are copious, one of the pioneers of which is Anna Wierzbicka, who has worked on the concept of semantic primes for decades. Finding universal, equivalent core meanings spanning across languages has been a kind of holy grail in linguistics. Following her framework, we will adapt ideas related to physical qualities and linked metaphors (Goddard \& Wierzbicka, 2014, pp. 55-79).







% \section{Terminology}

% spice
% xiang
% tabul






% 1.4. Objectives
% In the previous subsections, we briefly introduced \glspl{wanderwort}, innovation, and sensory language surrounding the realm of spice. Now we shall review our attempt to connect these concepts, establish our approach based on the theoretical framework, and concisely lay out the objectives of this dissertation.
% The goal of this thesis is to explore \glspl{wanderwort} and see what they can tell us about human contact, language, and cognition. To achieve this, our objectives are:
% A. Explore the `diffusion of spices', examining the geographical spread via trade---looking at directionality, speed, quantities, means of transmission (by land or sea)---and trace the propagation of accompanying \glspl{wanderwort} to the best we can with the available corpus data and existing philological literature.
% B. Explore the `language of spices', examining surrounding terminology related to the sensory domains (gustatory, olfactory, visual, tactile), and looking at the presence or absence of different linguistic categories---spice names acting as nouns, and derivationally related nouns, adjectives, verbs, and idiomatic and metaphoric expressions with the help of selected corpora from different languages.
% C. Compare the findings of the above two perspectives, juxtapose the mechanisms of material spread and Wanderwort propagation (`diffusion of spices') with patterns of linguistic behavior (`language of spices'), and look for patterns and correlations.
% In pursuit of Objective A, we make use of existing theories related to diffusion and innovation, while in tackling Objective B, we adapt concepts arising from the theories of semantic primes and linguistic synesthesia. In Objective C, we propose an original approach to examine spices which hopefully yields results that can prove or refute our hypotheses and answer the research questions. We should emphasize again that products of the spice domain are fortunate subjects because they embody the physical and the linguistic novelty as well; they are archetypal \glspl{wanderwort}, and we aim to `measure' their embeddedness in a culture by the degree of prevalence of more diverse terminology and collocations in the relevant language.
% During the course of this project, we plan to collate a linguistics infused account of the spice trade, with the help of sporadic but valuable information gathered from the literature, and the etymological knowledge amassed while working on the dissertation. A comprehensive, curated database of spice names could be, without doubt, a significant addition to many types of future research on spices.
% The thesis will focus on three major languages: English, Arabic, and Chinese---all representing areas and civilizations that had important shares in the spice trade throughout history. By the use of historical corpora, we will be able to look at snapshots from the distant past, and glance at the linguistic and historic situation surrounding spice use.
% In investigating the realm of spices, we expect to observe trends that are typical of innovation, both in lexical and practical adoption. If the existence of diachronic data allows us to see that the diffusion of the spice products resembles an S-shaped curve, we could possibly propose a model for Wanderwort propagation, and present a thesis rich in practical, empirical, and theoretical contributions. One that adds to the understanding of the nature of wandering words, as evidence for the exchange of new ideas and contact between civilizations.
% The following section deals with relevant literature and sources, after which we shall define the research questions. Then we arrive to the second half of this proposal and determine a Research plan, where we reveal the methods to be used in this study, followed by a discussion of expected results.
