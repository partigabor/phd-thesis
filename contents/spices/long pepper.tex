\section{Long Pepper}
\label{sec:long_pepper}

Long pepper is a relative of the black pepper, and as a commercial product it comes from two sources: the Indian long pepper \taxon{Piper longum}, and Javanese long pepper \taxonn{Piper retrofactum}{Vahl}. The latter is sometimes also called Balinese long pepper or Indonesian long pepper \tvolcite[551]{2}{peter_handbook_2012}.

% 27.11 Long pepper The long pepper of commerce comes from two species of Piper, both from the Piperaceae family. Indian long pepper comes from Piper longum L. and the Javan or Balinese or Indonesian long pepper comes from P. chaba Hunter (syn. P. retrofractum Vahl). The two are not clearly distinguished in the market. Indian long pepper grows widely all over India and Sri Lanka while the Java long pepper occurs naturally in the Java region of Indonesia, and in the Indo-Malayan region. A third species that is also marketed as Indian long pepper is P. peepuloides Wall; this is found in the north-east regions of India. Indian long pepper is a trailing plant, climbing on small shrubs, boulders, etc., while the Java long pepper is a climber climbing on support trees with the aid of clinging roots. The venation pattern is very distinctly different in the two species. The fruit of the Java long pepper is longer, larger and more pungent than that of the Indian long pepper. Both long peppers are similar to black pepper in chemical composition. The fruits contain volatile oil and alkaloids; the flavour is contributed by the essential oil which occurs at levels of 7–8 %. The pungency of the long pepper is contributed by alkaloids, mainly piperine but also piperlonguimine and pipalartine. The volatile oil of long pepper consists mainly of sesquiterpenes, hydrocarbons and ethers (α-bisabolene, β-caryophyllene, β-caryophyllene oxide, α-zingiberene, etc.), and saturated aliphatic hydrocarbons (pentadecane, tridecane, heptadecane, etc.) (Ravindran, 2000; Nirmal Babu et al., 2006).

\subsection{The Botany, Origins, and Cultivation of Long Pepper}

\subsection{The History of Long Pepper}

