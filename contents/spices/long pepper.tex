\section{Long Pepper}
\label{sec:long_pepper}




Long pepper is a relative of the black pepper, and as a commercial product it comes from two sources: the Indian long pepper \taxon{Piper longum}, and Javanese long pepper \taxonn{Piper retrofactum}{Vahl} Vahl. The latter is sometimes also called Balinese long pepper or Indonesian long pepper. According to \tvolcite[551]{2}{peter_handbook_2012} vol 2

% 27.11 Long pepper The long pepper of commerce comes from two species of Piper, both from the Piperaceae family. Indian long pepper comes from Piper longum L. and the Javan or Balinese or Indonesian long pepper comes from P. chaba Hunter (syn. P. retrofractum Vahl). The two are not clearly distinguished in the market. Indian long pepper grows widely all over India and Sri Lanka while the Java long pepper occurs naturally in the Java region of Indonesia, and in the Indo-Malayan region. A third species that is also marketed as Indian long pepper is P. peepuloides Wall; this is found in the north-east regions of India. Indian long pepper is a trailing plant, climbing on small shrubs, boulders, etc., while the Java long pepper is a climber climbing on support trees with the aid of clinging roots. The venation pattern is very distinctly different in the two species. The fruit of the Java long pepper is longer, larger and more pungent than that of the Indian long pepper. Both long peppers are similar to black pepper in chemical composition. The fruits contain volatile oil and alkaloids; the flavour is contributed by the essential oil which occurs at levels of 7–8 %. The pungency of the long pepper is contributed by alkaloids, mainly piperine but also piperlonguimine and pipalartine. The volatile oil of long pepper consists mainly of sesquiterpenes, hydrocarbons and ethers (α-bisabolene, β-caryophyllene, β-caryophyllene oxide, α-zingiberene, etc.), and saturated aliphatic hydrocarbons (pentadecane, tridecane, heptadecane, etc.) (Ravindran, 2000; Nirmal Babu et al., 2006).

\subsection{The Botany, Origins, and Cultivation of Long Pepper}

\subsection{The History of Long Pepper}

\subsection{The Names of Long Pepper}

\subsubsection{English}

In English, \textit{long pepper} is a calque after the modelling the Latin \textit{piper longus}, and first appear in the early Old English Medicinal text known as \textit{Bald's Leechbook}\footcite[longpepper]{oed}. The plant's binomial name was also derived from this term, using the neuter form \taxon{Piper longum}. The \gls{OED} points out that it was supposed to refer to flowers or unripe fruits of the (black) pepper plant in earlier times. This notion must arise from the fact that the long pepper fruits do somewhat look resemble the unripe black pepper clusters looking like catkins, and some Romans must have assumed that long pepper is just the unripe version of small black pepper clumps. Nevertheless, I am certain that the Romans did not see young unripe black peppers still on the vine very often, so we can forgive them this time. The gloss of \textit{long pepper} from Latin is not unique to English, many European languages went down the same route.\footnote{Compare Anglo-Norman \textit{poivre lonc} (13th cent.; Middle French, French \textit{poivre long}), Middle Dutch \textit{lanc peper} (Dutch \textit{lange peper}), Middle Low German \textit{lanc pēper}, \textit{lancpēper}, Old High German \textit{langpfeffar} (Middle High German \textit{langer pheffer}, German \textit{langer Pfeffer}), Old Swedish \textit{langa pipar} (Swedish \textit{långpeppar}), according to the OED; as well as Italian \textit{pepe lungo}, Spanish \textit{pimienta larga}, Portuguese \textit{pimenta-longa}, Finnish \textit{pitkäpippuri}, Polish \textit{pieprz długi}, etc.} 

In the East, however, where there was no Latin to distinguish between black (\textit{nigrum}) and long (\textit{longum}), simply the Sanskrit name \textit{pippali} was borrowed by the languages whose speakers got familiar with long pepper and its sisters directly from speakers of Indic languages, compare Malayalam \textit{tippali}, Telugu \textit{pippali}, or Tibetan \textit{pi pi ling}. Modern Hindi \textit{pippali} is most probably a \textit{tatsama}\footnote{\textit{Tatsama} refers to a group of vocabulary consisting of learned loanwords from Sanskrit into modern languages of India, including both Indo-Aryan and Dravidian languages. It is comparable to the usage of Greek and Latin words in modern European languages, as they belong to a higher register. E.g.: the choice to use \textit{curriculum} over \textit{courses}. It is accompanied with  \textit{tadbhava}, which is the class of words that evolved.} word, a learned loan from Sanskrit. The name of the sacred fig (\textit{Ficus religiosa}) -- otherwise known as the bodhi tree, under which the Buddha gained enlightenment and rendered \textit{peepul} in English from Hindustani \textit{p\={i}pal} -- has Sanskrit \textit{pippala} `berry, especially the fruit of the sacred fig' as an etymon. The sacred fig was a kind of ``spiritual import'', we know about two instances when the Indian king gifted bodhi trees to the Chinese emperor in 641 and 647 from Magadha, the homeland of these trees \parencite[122]{schafer_golden_1985}.

Long pepper in Chinese is 蓽茇 \textit{bìbō}, as it appears on TCM databases\footnote{}, or 蓽拔 \textit{bìbá}, with some other historical character variations. A local Hong Kong spice vendor is marketing it as 長胡椒/蓽撥 \textit{zhǎng hújiāo/bìbō}, the first of which is a obvious rendering of the English \textit{long [black] pepper}, while the second is using the second character 撥 \textit{bō}, the same that is used the first time in historical documents. The first mention is in 通典 \textit{Tongdian}\footnote{\url{http://www.chinaknowledge.de/Literature/Science/tongdian.html}} ``Comprehensive statutes'' written by Du You, a late \nth{8}-century encyclopedia and administrative history covering ancient times up to 756, including the Battle of Talas and other important events in Tang history. Long pepper appears in the last part of the book about ``Frontier defense'', under the section 波斯 \textit{Bosi} [Persia], in a listing all the products that are supposed to be found there.\footnote{\url{https://ctext.org/dictionary.pl?if=en&id=565096}} Long pepper also appears in the 酉陽雜俎 \textit{Youyang Zazu}\footnote{\url{http://www.chinaknowledge.de/Literature/Novels/youyangzazu.html}} ``Miscellaneous Morsels from Youyang'', a \nth{9} century Tang miscellany on various topics by Duan Chengsi. It contains fantastic stories from ghosts to strange animals, ``legends and hearsay, reports on natural phenomena, short anecdotes, and tales of the wondrous and mundane, as well as notes on such topics as medicinal herbs, perfume, tattoo and language'' -- to quote \textcite[1]{reed_youyang_1995}. Book\footnote{The original term is 卷 \textit{juan}, menaing `scroll, book', or `volume, chapter'.} eighteen contains 24 entries of exotic plants that have been imported to China or brought as tribute from places such as Syria, Persia, Malaysia, and Silla [Korea]. The author usually gives the foreign names of these products and tries to compare them to a plant more familiar to the Chinese readership. The plants featured here include cardamom, galbanum, acacia, jackfruit, Balm of Gilead, Narcissus, and jasmine \parencite[68]{reed_youyang_1995}. Entry 56 is on long pepper (蓽撥 \textit{bìbō}), where Duan tells us that it comes from Magadha, and pronounced as 蓽撥梨 *bit-bat-li\footnote{Reconstructed Tang pronunciation}. Magadha refers to a culturally important historic region of India roughly on the eastern Ganges-plain. He also tells us the purported Fulin [Roman] name for it, and then proceeds to describes the appearance of the plant, likening the fruit to mulberries, which bear a close enough similarity of long pepper fruits. This is clear evidence that the Chinese used the Sanskrit word referring to long pepper, and \textcite[151]{schafer_golden_1985} mentions that it was commonly shortened to \textit{pippal} and mispronounced as \textit{pitpat} or \textit{pippat}. 

\begin{quote}
    蓽撥,出摩伽陀國,呼為蓽撥梨,拂林國呼為阿梨訶他。苗長三四尺,莖細如箸。葉似戢葉。子似桑椹,八月採。 (YYZZ 18:56)\footnote{The same page also has an entry on black pepper. \url{https://ctext.org/library.pl?if=en&file=85088&page=282}}
\end{quote}

https://www.zdic.net/hans/%E9%98%BF%E6%A2%A8%E8%AF%83%E5%92%83
阿梨訶咃 / 阿梨訶陀 arihata?
a li he tuo
qa li ho? tuo?
https://cidian.qianp.com/ci/%E9%98%BF%E6%A2%A8%E8%AF%83%E5%92%83

It is now a good time to remind the reader that it is this long pepper that gave us the word \textit{pepper} in English and many other languages around the world, as it was shown in \ref{ety:pepper}. 

I mentioned ``sisters'' earlier, because long pepper is not alone here, there are other species, such as \taxon{Piper retrofractum}, also known as \textit{Javanese long pepper} or sometimes as \textit{Balinese long pepper}. At this point it will make sense to use the name \textit{Indian long pepper} when referring to \taxon{Piper longum} to avoid confusion. These two plants and their fruits are very similar, and they are often lumped together in discussions. It is enough to remember that Indian long pepper is important in India and mainland Southeast Asia, while Javanese long pepper is more relevant to insular Southeast Asia, but both were exported to medieval China and most likely there was no distinction made between the two. Javanese long pepper is more pungent than both black and long pepper, and is used in medicine, pickling, and curries, and much is exported to China -- wrote \textcite{burkill_dictionary_1935}. Long pepper also spread through southern Asia before black pepper \parencite[1746-1751]{burkill_dictionary_1935}. 

We know that long pepper was popular in Rome during Pliny's time, and that it was more expensive than black pepper. And if we look at the fact that the name borrowed to Greek from Sanskrit was \textit{pippali} and not \textit{marica}, we can readily assume that it was introduced to Europe before black pepper.

These plants hold the key to one of the questions I asked at the beginning of this project, that is: Why was the Indonesian word \textit{cabai} so resistant, and why Indonesian did not loan words of 'pepper' or 'chili'?  


They bear very similar fruits, turning bright read when ripe, reaching upwards.  

% long pepper  n. the dried immature fruit-spikes of either of two vines of South East Asia, Piper longum and P. retrofractum (family Piperaceae), used as a condiment; (also) the plants themselves; cf. pepper n. 1b.Long pepper was formerly supposed to be the flowers or unripe fruits of the pepper plant, Piper nigrum.  


Chinese
