% \subsection{Pink Pepper}
% \label{sec:pinkpepper}




% Wyk 254:
% Description Pink pepper is the ripe fruits of Schinus molle or S. terebinthifolius, about 6 mm (¼ in.) in diameter, with a small hard seed surrounded by a brittle, bright red or pink outer wall. They have a pungent, resinous and spicy taste. the plant The pepper tree or Peruvian pepper tree (Schinus molle) is a hardy, long-lived tree of up to 15 m (ca. 50 ft) high with characteristic pendulous branches and leaves. The Brazilian pepper tree (S. terebinthifolius) is a smaller tree (10 m or ca. 32 ft) with non-drooping branches and broad leaflets. Both species are popular garden trees and have become invasive in warm regions. origin S. molle is indigenous to the Peruvian Andes in South America, while S. terebinthifolius occurs naturally in Brazil.1 Pink peppercorns are derived from both S. molle and S. terebinthifolius and their names seem to be used interchangeably, so that it is almost impossible to determine their botanical origin. cultivation Both species are easily grown from seeds and thrive under almost any conditions. harvesting The fruits occur in large clusters and are hand-picked when fully ripe and dry. culinary uses Pink pepper has become fashionable as a colourful additive to the pepper grinder, often in combination with black and white pepper. It adds to the complexity of the aroma and flavour of the ground pepper but is rarely used on its own. It may not be safe to ingest large amounts. The outer sweet part of the fruits can be fermented to produce a drink or can be boiled to make syrup. The fruits of Schinus molle (molli in the local Quechua language) were once very popular in the Central Andean region for making a drink called chicha.2 Flavour compounDs The main compound in the essential oil of the fruits of both species is often α-phellandrene, with smaller quantities of β-phellandrene, α-terpineol, α-pinene, β-pinene and p-cymene.3,4 Cardanol, a known skin irritant, occurs in the essential oil of S. terebinthifolius.5 The oil of this species appears to be variable but α- and β-pinene, δ-3-carene, limonene, α- and β-phellandrene, p-cymene and terpinolene are often reported as the main compounds.3 notes Most regions of the world have their own pungent “pepper” plants.1 These are botanically and chemically mostly unrelated and include African Guinea pepper (Xylopia aethiopica), African pepper tree (Warburgia salutaris), West African Melegueta pepper (Aframomum melegueta), East Asian sansho or Chinese pepper (Zanthoxylum piperitum, Z. simulans and several others), South American mountain pepper (Drimys piperita), Tasmanian mountain pepper (D. lanceolata) and, of course, the well-known black pepper and its relatives, as well as the popular chilli peppers.

% 1. Mabberley, D.J. 2008. Mabberley’s plant-book (3rd ed.). Cambridge University Press, Cambridge. 
% 2. Goldstein, D.J., Coleman, R.C. 2004. Schinus molle L. (Anacardiaceae): Chicha production in the Central Andes. Economic Botany 58: 523–529. 
% 3. Bendaoud, H., Romdhane, M., Souchard, J.P., Cazaux, S., Bouajila, J. 2010. Chemical composition and anticancer and antioxidant activities of Schinus molle L. and Schinus terebinthifolius Raddi berries essential oils. Journal of Food Science 75: 466−472. 
% 4. Maffei, M., Chialva, F. 1990. Essential oils from Schinus molle L. berries and leaves. Flavour and Fragrance Journal 5: 49–52. 
% 5. Stahl, E., Keller, K., Blinn, C. 1983. Cardanol, a skin irritant in pink pepper. Planta Medica 48: 5−9.