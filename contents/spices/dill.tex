% The word dill and its close relatives are found in most of the Germanic languages; its ultimate origin is unknown.[3] The generic name Anethum is the Latin form of Greek ἄνῑσον / ἄνησον / ἄνηθον / ἄνητον, which meant both 'dill' and 'anise'. The form anīsum came to be used for anise, and anēthum for dill. The Latin word is the origin of dill's names in the Western Romance languages (anet, aneldo, etc.), and also of the obsolete English anet.[4] Most Slavic language names come from Proto-Slavic *koprъ,[5] which developed from the PIE root *ku̯ə1po- 'aroma, odor'.[6]
