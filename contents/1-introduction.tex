%% This is introduction.tex of the Ph.D. thesis of Gábor Parti.
%% It follows the requirements of The Hong Kong Polytechnic University.
%% LaTeX template originally created by Tom M. Ragonneau.



% Aim: 
% Scope: 
% Research questions: 
% Gap: 
% Main argument: 
% Contribution:

\chapter{Introduction}
\label{ch:introduction}

% 1. What is your original contribution? 
% 2. Why should the examiner care about your research?
% 3. What is the thesis problem statement?
% 4. What do you (not) hope to achieve?
% 5. What are the research questions and hypotheses?
% 6. What are your epistemological and ontological positions?
% 7. What is your contribution to the field?
% 8. How is the thesis laid out?



% 0. There is an ingredient... spin a story?

\lettrine[lines=\iniciale]{\textcolor{\accentcolor}{U}}{nderstanding} the language of spices is a key to open a door to this world. A door that leads to the realization that our cultures---and our foods---are deeply interconnected, and that they have have been so for thousands of years. I will try to demonstrate this by introducing these fascinating substances from a new perspective, the perspective of language. It is trendy nowadays to talk about \textit{foodways}, a term that refers to ``the eating habits and culinary practices of people, regions, or historical periods'' \pvolcite[]{2}[289]{allen_encyclopedia_2013}, and food history, a relatively young interdisciplinary academic field is starting to gain traction as well, but the connections between language and food are one of the most interesting examples of contemporary humanities research I have come across \textcite[see][]{jurafsky_language_2014}. There is a segment of this topic---the spice domain---which encompasses products that have had profound effects of human imagination, culture, and history. Although overshadowed by the serious and heavy questions of nutrition, scientific research on spices was never a fringe field; it is enough to look at the many pharmacological studies that dive into the chemistry of these materials to see that people are still interested in their health effects, as much as their taste and aroma.

As spices are not a necessity to human survival, but rather constitute an enthralling phenomenon that can be studied from many angles, research on spices has been embraced by many historians, a few botanists and literary critics, and countless culinary enthusiasts. It may be so that spices are not vital for humans, but sustenance itself is just enough to maintain life, not to enjoy it or live it to the fullest. Spices today represent the excitement, the vigor, as it is so clear from expressions in our language: to \emph{spice up} your life is to enliven it!



% There is a small aromatic tree that grows on the islands of the Caribbean, bearing little berries. The indigenous people used this tree in every step of their food preparation, and it is still an important crop/ingredient today. The wood is used to smoke meat, the leaves are added for aroma similarly to the bay-leaf in Europe, and the ground up dried unripe berries are used as a spice. Outside the Caribbean, it is not a primary ingredient but many places use it in various ways since its diffusion starting from the \nth{16} century. In English, you can call it \textit{Jamaica pepper} (it is not a true pepper), in Arabic, it is \textit{fulful ifranjī} [Frankish pepper] or \textit{fulful ḥulw} [sweet pepper], and in Chinese, it is mostly known as \textit{duoxiangguo} [many-spice/fragrant-fruit]. Such different names! Furthermore

\subsubsection{Original Contribution}

% 1. What is your original contribution? 
This thesis aims to do a systematic investigation on spice names and related terminology, including products that were used (or still being used) medicinally, as incense, or as perfume. Aromatics that were at some point considered spices, have been traded and transported across long distances since antiquity and before, and the most coveted ones have slowly dispersed throughout the globe. 
Spices and the spice domain as a topic are usually discussed within the broad areas of history, botany, chemistry, and gastronomy, all concerned by very different aspects of these materials. To the best of my knowledge, there is no academic work that puts the field of linguistics in focus when discussing spices as a whole, and so this project is a unique contribution to our linguistic knowledge about the spice domain.

% 2. Why should the examiner care about your research?
But why should anyone care about spices and their names? Because exploring the names of the products of the spice trade---traveling on vast networks of historic trade routes such as the Silk Road (small volume of trade), and its nautical counterpart the Maritime Silk Road (large volume of trade)---helps us to map and better understand linguistic contact and cultural exchange. These ever-expanding trade networks, first regional, later connecting East and West were a precursor to today's globalized, interconnected world, and one of their most lucrative products was dried plant-matter. These aromatic substances were lightweight, easy to transport, and resistant to spoilage. And, of course, they were highly valued for their fragrant and pungent properties, and their reputed---both putative and real---benefits for the human body and soul. Exotic and rare spices and their role in rituals, medicine, and later cuisine made them sought after. The spice business inspired people to trade, travel, explore, and wage wars. Spices are important in world history as they are directly responsible for discoveries, colonization, and the birth of capitalism. We know a great deal about the nature of spices thanks to botanists and naturalist, their medicinal effects thanks to pharmacists and chemists, and their uses and culinary values thanks to experts of gastronomy. There is also a vast literature on the story and spread of spices thanks to researchers of history, but the careful study of their names is often neglected. This work was born due to a fascination with the global journey and etymology of spice words, in the ``true sense'' of the word etymology. 

\subsubsection{Problem Statement}

% 3. What is the thesis problem statement?
Soon, my attention slightly shifted towards a problem that could be best described by a lack of consistent and comprehensive understanding of spice names. The absence of proper research regarding spice terminology results in a lack of a standard, and decline of trust in secondary literature. Authors often give misguided and contradicting information regarding the origins of a name, or speculate on their meaning. There are no two authors that use the same set of names when discussing a spice, which in itself is not a problem in most cases, but it leads to problems in case of lesser known or exotic items. There is a great deal of confusion on names and identities in the spice literature, especially in lay areas aimed at the general public, such as popular histories or guidebooks. The reasons for this are several.
Firstly, the experts of herbs, spices and other aromatics are chiefly botanists, food industry professionals, chemists, chefs and food writers, merchants and historians. Most people in research related to spices focus on aspects of the products other than their names: from plant morphology, chemical composition, and pharmacological effects, to social and cultural histories, their symbolism in literature, not to mention the myriad of ways on how to buy, store, mix, and use spices in creative recipes discussed by the handy spice encyclopedias tailored for gastro-enthusiasts. Relatively few linguists devoted their time and attention to trace spice origins. In short, the topic of spices requires a highly multidisciplinary expertise, and when a plant taxonomist writes about linguistics, or a culinary writer approaches history, some mistakes are due. 

Secondly, there is no agreed upon inventory or reference work of spice names to cover the multitude of spices that exist, and their many names in various languages; least of all a complete list of \emph{all the spices}. Truthfully that seems rather impossible, or at least quite a daunting task to embark on. Although the internet nowadays is full of compact guides and indices of spices assembled by people who are fascinated with spices and their colorful uses listing their names in many languages, these are not always trustworthy, and often cite no sources. Similarly, blogs and articles are most often than not dubious, and almost always require fact checking, as many are just permutations of historically inaccurate anecdotes and origin stories.
% Most recently, I have found spice lists that are clearly just a dump of computer generated information floating around on the internet, using web scraping and machine translation: these lists are highly inaccurate. 
Until today there is no comprehensive treatise on spice terms within academia, and no database that focuses on, clarifies, explains, or compares their names.

This is not to say that there is no work done on spice terminology, there are a number of high quality writings from philologists, linguists, and historians well versed in one or more relevant linguistic and cultural area. The problem is that this kind of research requires a highly specialized knowledge, and in result the information already out there is sporadic, less accessible, and grossly unorganized. Key pieces of information are often hidden between the pages of books on traditional philology, literary critique, botany, medicine, economic history, and archaeology of a given region. Not to mention the many old works that are the primary sources for the aforementioned publications. Consequently, since little effort have been made to collate the data, there is a chasm between the critically researched reliable information and what the end user---whether it is a fellow researcher or a spice zealot---can easily access.

\subsubsection{Goals}

% 4. What do you (not) hope to achieve?
The original goal in the beginning of this work was to gather and augment the existing information about spice names, their origins, and track their diffusion on spatial and temporal trajectories. This still constitutes the core of this thesis, and I hope to achieve this by combing through the existing literature and collecting the names of spices, amending the gaps, and correcting possible errors on the way. Doing so, the result should be a carefully researched compendium of spice names, grounded in philology and linguistics, but with the awareness of what spices are to botany, and what their role was in history. \Cref{ch:data} presents this process and displays the data seriatim, in a linear manner. 

This procedure shall manifest in a dataset of spice names, with complete lexicographical annotation including etymological information and attestation dates. This in turn, would allow me to trace the words and track the linguistic diffusion of spices through space and time, which is then can be discussed hand-in-hand with the physical diffusion of the materials. Eventually, the mapping of the spices will be the basis for a discussion on the implications of linguistic and cultural contact, and exchange, and it makes up \cref{ch:diffusion}
This chapter ties well together with the concept of \glspl{wanderwort}, `wandering loanwords', a phenomenon known in the field of historical linguistics related to the topic of borrowing and material culture. 

In addition to this, the data of spice names will also be the basis of a linguistic analysis, focusing on the characteristics of terms themselves presented in \cref{ch:names}. This part will include a deep dive into how spice names are created or borrowed, how prototype items beget prototype words to generate new names for novel items of trade, and into the mechanisms and motivations of linguistic acculturation and spice name propagation.

% MORE ON SENSORY THINGS SOMEWHERE?

Finally, spice names will be discussed according to their role in daily language, how spice words entered the lexicon and what is their role in metaphors and idiomatic expressions. This is to trace spices' embeddedness in a culture, and to see how significant they are in the everyday human experience, as can be seen in \cref{ch:language}.

% 5. What are the research questions and hypotheses?

\section{Definitions}
\label{sec:definitions}

The first step is to clarify what is meant by \textit{spices}. According to the \gls{OED}, the definition of \textit{spice} is as follows: ``One or other of various strongly flavoured or aromatic substances of vegetable origin, obtained from tropical plants, commonly used as condiments or employment for other purposes on account of their fragrance and preservative qualities.'' Similarly, the first meaning for \textit{spice} as a noun in \gls{MW} is ``any of various aromatic vegetable products (such as pepper or nutmeg) used to season or flavour foods''. The Wikipedia entry on \textit{Spice} gives slightly more information, hinting on which plant parts are frequently used as spices and mentions their food-coloring properties, while also---very appropriately---ventures beyond the culinary stance of usual dictionary definitions, stating that ``spices are sometimes used in medicine, religious rituals, cosmetics or perfume production.'' This notion is much more important than expressing it with a mere `sometimes' could imply as we will see; before modern times, spices were much more important for the medicinal properties.

There is no universal definition on what a spice is; botany, pharmacology, gastronomy, and history all have different perspectives. However the idea about ``spices'' that the reader currently has in mind, is bound to be a culinary one. Some authors try to give a definition according to plant morphology, \textcite[9]{czarra_spices_2009} writes about ``an aromatic part of a tropical plant'', and goes on to mention bark, flower, root, and seed. \textcite[xix]{turner_spice_2004} adds gum and resin, fruit, and stigma to this listing. For a full picture, we must complement it further, as spices can come in many forms: dried tree barks (cinnamon, cassia); twigs
(cassia twigs); flower buds (cloves); stigmas or styles (saffron); fruits (pepper, chili); fruit walls or pericarps (star anise); berries (allspice, juniper); seeds (nutmeg, coriander); seed coverings or arils (mace); seed pods (cardamom, vanilla); and roots and rhizomes (ginger, turmeric). Technically, every dried part of a plant can be referred to as spice, except the leaves. The green leaves---fresh or dried, but mostly used fresh---are considered herbs, and they are used for similar purposes to spices nowadays: flavouring, seasoning, garnishing. Dried leaves of herbs can be categorized as ``spice herbs'' \parencite[see][]{van_wyk_culinary_2014}. The category of herbs can be problematic, because there is a botanical definition, and also a culinary definition, and the literature often confuses the two. Botanically, a herb is an annual?? plant that has a soft stem, while a culinary herb is is where the leaves are used in food preparation, similar to a medicinal herb...

O'Connell (2016, p. 9) backs this view in his informative compendium, but also cites Rosengarten (1975, p. 16, as cited in O'Connell, 2016), who maintained that it is `extremely difficult to determine where a spice ends and a herb begins'. According to him, culinary herbs are just one group of spices. Along these lines, Britannica (n.d.) for example treats herbs and spices in a single entry.
The above distinction---that herbs are the greens and spices are every other (dried) parts of a plant---is essentially nonsense to a botanist since it echoes the needs of a chef. We can give examples for both spice and herb from the same species: coriander seeds and coriander leaves (also called cilantro or Chinese parsley in the US) are both from the plant \taxon{Coriandrum sativum}. Another often mentioned difference is that herbs are soft stemmed, annual plants that die each year, in contrast to woody, spice yielding trees or bushes. This, on the other hand, is a botanical definition, and not very useful for somebody active in the culinary arts (Allen, 2012, p. 10). Moreover, most plants we consider herbs grow in temperate climates, while spices tend to grow in tropical regions (Turner, 2004), a further classification on botanical basis. Herbs can also be categorized into culinary and medicinal herbs, and in both cases, the leafy, green parts of foliage are used for their aroma and flavour, and supposed health benefits, respectively.
Defining spices and herbs is difficult because the definitions vary by discipline, depending on the needs of the expert: the gardener, the herbalist, the chef. In the present study, we focus on dried---mostly plant-based---aromatic commodities that traveled long distances due to trade and were at certain points in history considered a desired commodity or even a luxury. This is basically the definition of the historian, where the implications of climate and remoteness translated as value; spices were a produce difficult to obtain, and thus obviously expensive. Long distance transportation was possible when the plant products were hauled across deserts and oceans in a dried form, making them lightweight and less susceptible to rot. Culinary and medicinal herbs had their value in their freshness, and thus were not ideal products of trade; they spread through naturalization and were generally available locally. Historically, anything rare and aromatic can be considered a spice, including incense for burning, coffee in the early days, fragrant perfumes, or even exotic fruits; anything `special' (even if today nobody would agree so). This is well observed in the origins of the English name: the word \textit{`spice'}, via Old French \textit{espice}, comes from Late Latin \textit{speciēs} (plural) `spices, goods, wares' with the original meaning in Classical Latin being `kind, sort'. English `species' and `special' are obvious cognates of the same Latin etymon: speciō, which referred to anything observable: a sight, `spectacle' (cf. `inspect'), and also anything extraordinary, `specific' kind of item (Glare, 2012, pp. 1983-1984)
This implies that in different periods, the meaning of the term `spice' covered different substances, based on what products were considered special, desirable, and difficult to obtain; the definition constantly changed. From this point on, whenever spices are mentioned, we refer to this broad definition, using `spice' as an umbrella term for any historic exotica. These definitions, and the differences between the terms spice, incense, herb, condiment, etc. will be explored in detail in the dissertation, as well as a shift in meaning considering spices.



% 1.2. Scope
% The scope of this study can be delimited by three factors: the subjects under study (the spices); the languages involved; and the timeframe covered. Spices
% Firstly, the subjects under scrutiny shall adhere to the following criteria: a material, (1) once of significant value, (2) traded over great distances, (3) since pre-modern times, for its (4) aromatic or pungent properties. This means that besides plant-based spices, we can include biotic materials not intended for oral consumption: incense (usually from tree resin and oil) such as myrrh, frankincense, or sandalwood, and substances of animal origin, such as musk or ambergris.
% Due to the lack of a single definition, the lists of herbs and spices are as many as books written on them. Every compiler---cook, linguist, or historian---comes up with his or her own list, there is no standard approach. Czarra (2009) discusses only 5 premier spices, `the foremost five' as he calls it. Nabhan (2014) works with 26 `spice boxes', to use his terminology. (Rosengarten, 1975) mentions 41 `of the world's most popular spices'. Petruzzello (n.d.) simply lists 70; Van Wyk (2014) in his reference guide introduces 120, and the glossary of spice names in Dalby (2002) contains more than 200 entries. The attempt to collect and describe spices in their entirety is not a modern phenomenon, people tried to collect and list the knowledge of aromatic and medicinal plants and products ever since the day we use them, proof being the countless herbariums1 and pharmacopeias2 of the last two millennia. Besides focusing on medicinal properties, spices were listed as commodities of trade. We know for example that for reasons motivated by customs tariff and taxation, Roman ports, such as Berenike3 had inventories of speciēs that required special attention (Parthasarathi, 2015), and as a later example we can mention an Italian merchant's book on trade from the 14th century contains 188 spices (Balducci Pegolotti & Evans, 1936, pp. 411-435). A comprehensive list of the world's spices is likely to be extremely long and collecting them all in this short period of time seems to be an unattainable goal. In the dissertation, we will follow our criteria laid out above, and start with the most prominent of spices: black pepper (Piper nigrum). Pepper is the world's most traded and consumed spice, followed by cinnamon in the Western markets (Senaratne & Pathirana, 2020, p. 16). We will work downwards, considering the most popular, most consumed, most prominent spices in every language/culture we propose to look at.
% The Harmonized System (HS)4 tariff codes used in international trade differentiates around 20 different spices, mostly driven by practical reasons related to storage and transportation requirements. These are the most traded spices around the world today.
% 1 A collection of preserved plants with descriptions and botanical analysis.
% 2 Pharmacopeias are historic manuals for making medicines and drugs.
% 3 For the history of the ancient Indo-Mediterranean trade through archeological research in the port of Berenike, see Sidebotham, S. E. (2011). Berenike and the ancient maritime spice route. University of California Press.
% 4 The Harmonized Commodity Description and Coding System is a standardized international system to classify globally traded products. (https://www.freightos.com/freight-resources/harmonized-system-code-finder-hs-code-lookup/)
% Figure 1. HS04 codes for Chapter 09: Coffee, tea, mate, and spices. Additional digits (HS06) would signify more fine-grained distinctions between commodities, whole or ground for example (FREIGHTOS).5
% We plan to include a number of spices, as much as needed for our attempt to find patterns in diffusion and language use, keeping in mind expected variations per cuisine and cultural sphere. Our initial base 20 candidates are as follows:
% 1. black pepper
% 2. cinnamon & cassia
% 3. cloves
% 4. nutmeg & mace
% 5. ginger
% 6. cardamom
% 7. turmeric
% 8. chili & paprika
% 9. vanilla
% 10. Sichuan pepper
% 11. saffron
% 12. star anise
% 13. cumin
% 14. coriander
% 15. camphor
% 16. frankincense
% 17. myrrh
% 18. musk
% 19. ambergris
% 20. asafoetida
% The list is not final, it will probably change due to availability of data or other considerations. However, if possible, we will include more items to broaden the scope, further considered items are anise, fennel, dill, mustard, sandalwood, sumac, etc.
% The literature usually categorizes spices by their taxonomy governed by rules from the discipline of biology. Although above we discussed spices from a historical perspective, and mentioned standards of international trade, we should emphasize that this is a linguistic study and therefore our basis of comparison will be spice names, more precisely the word senses we collect and identify with the help of corpora and lexical-semantic databases, such as the WordNet6 (Miller, 1995), botanical or other categorizations would be only secondary. Now that we identified the subjects in our scope, let us move on to the languages to be included and the timeframes to be covered.
% 5 Accessed at https://www.freightos.com/freight-resources/harmonized-system-code-finder-hs-code-lookup/
% 6 Accessible at https://wordnet.princeton.edu/
% Languages
% Secondly, we plan to look at three languages, namely Arabic, Chinese, and English---all three from distant language families, with very distinct features. These languages represent three very different cultural spheres that had all participated in the spice trade in different historical periods and were at times highly influential players. Investigating spices and spice names through the lens of these languages offers us an opportunity to take step back from the usual Eurocentric viewpoint, and observe the spice domain from a more global, inter-cultural approach, investigating 3 pillars of human high culture as equals. These civilizations had interactions not only with each other, but their respective cultures and languages---rich in literary traditions---have left a mark on many of the world's languages through emigration, trade, cultural and religious influence, colonization, and imperialism. Eras
% Thirdly, a few words on defining the time periods we wish to cover. We will start our linguistic investigation using a set of contemporary, comparable web corpus data, chosen from the TenTen corpus family (Jakubíček et al., 2013) for all three languages: Arabic (Modern Standard Arabic; MSA), Chinese (Mandarin, Simplified and Traditional), and English. The selected corpora are all available on the Sketch Engine at https://app.sketchengine.eu (Kilgarriff et al., 2014). We are interested in the contemporary `language of spices', the words of the spice domain and related terminology will be explored from linguistic-cognitive perspectives.
% However, we would also like to accommodate for the change of meaning on what spices once were, and how humans use them now vs. earlier periods. Traditional folk medicine and herbal remedies are residues of the past, apparent to different degrees in various cultures. To look for evidence for this shift in language use, we choose to explore historical corpora as well. The era we decided over is a distant one, and it roughly covers the 7-11th centuries, a period significant from both Arabic and Chinese points of view. In the Middle East we have seen the rapid expansion of Islamic empires in the form of the Umayyad (661-750) and Abbasid caliphates (750-1258), stretching over vast swathes of land from North Africa on the west, to the easternmost extremities of Persia. On the Far East, the Tang dynasty (618-907) illustrated the peak of classical Chinese civilization---usually regarded open and cosmopolitan---controlling the regions on the eastern terminus of the Silk Road. In both cases, this time is regarded as an important cultural and economic golden age: poetry, literature, science, and trade flourished. Important materia medicas are extant from the Tang era (Wu, 2005), and Islamic authors were occupied producing heavy tomes on geography, alchemy, medicine, philosophy, and other topics7. During the 8th century both powers reached each other's spheres of influence in central Asia, and at 751 the battle of Talas---ending with the victory of the Arab forces and their allies---affected the fate of the region for centuries. The corpora available from these pre-modern times is a corpus of Classical Arabic literature, a Chinese corpus of Tang era poetry, and other corpora containing textual data from relevant times. All corpora and databases are presented in greater detail in section 3 Methods.
% 7 For more, see Meri, J. W. (2006). Medieval Islamic civilization: An encyclopedia. Routledge.
% In the case of English, the appropriate, matching historical period would be that of Old English, however, this earliest recorded period is not abundant in spice related terms (although attested wandering loanwords include `pepper', `copper', `gem', `mint') (OED, n.d.; Wollman, 1993). In the case of English, we propose to do a comparison between Modern English and Early Modern English (late 15th to late 17th c.), where the dichotomy in the domain of spices between contemporary and historic times is still observable. The data from the available corpus is largely from the 1600s, which conveniently falls into the historical period known as Age of Exploration/Age of Discovery, when Europeans sailed the globe far and wide, competing in the search for new spices and new worlds. To sum up, the project will focus on Arabic (Classical-Modern Standard), Chinese (Classical/Middle-Modern), and English (Early Modern-Modern).









% Salad herbs
%     Potherbs
%     Microgreens
% Culinary herbs
% Spice herbs
% Spices
% Spice mixtures
% Seasonings \& Condiments
%     Seasonings
%     Condiments
%     Sauces
%     Dips
%     Pickles and preservatives
%     Essences
%     Vinegar
%     Herbal extracts and liqueurs
%     Food coloring
%     Garnish

% Main uses: spice/herb/coloring/flavouring

% herb (botanical)
% herb (culinary)
% spice (culinary)
% spice (historical)

% 1. Spices as pharmacia
% 2. Spices as aromata
% 3. Spices as pigmenta
% 4. Spices as condimenta
% \parencite{halikowski_smith_portugal_2001}

% guide
% checklist
% index
% inventory







% 5. What are the research questions and hypotheses?





\setlength{\tabcolsep}{3pt}

\begin{table}[ht]
% \begin{adjustbox}{max width=\textwidth}
\begin{tabularx}{\textwidth}{@{}r>{\footnotesize}llll@{}rl@{}}
\toprule
\textbf{\#} & \textbf{Species}             & \textbf{English} & \textbf{Chinese} & \textbf{Translit.} & \textbf{Arabic} & \textbf{Translit.}     \\ \midrule
1           & \textit{Pimenta dioica}            & allspice         & 多香果              & \textit{duōxiāngguǒ}     & فلفل إفرنجي     & \textit{filfil ifranjī}      \\
2           & \textit{Pimpinella anisum}         & anise            & 茴芹               & \textit{huíqín}          & ينسون           & \textit{yansūn}              \\
3           & \textit{Ferula assa-foetida}       & asafoetida       & 阿魏               & \textit{āwèi}            & حلتیت           & \textit{ḥiltīt}              \\
4           & \textit{Carum carvi}               & caraway          & 葛縷子              & \textit{gělǚzi}          & كراويا          & \textit{karāwiyā}            \\
5           & \textit{Elettaria cardamomum}      & cardamom         & \tradchinesefont{荳蔻}            & \textit{dòukòu}          & هال             & \textit{hāl}                 \\
6           & \textit{Cinnamomum cassia}         & cassia           & 肉桂               & \textit{ròuguì}          & سليخة           & \textit{salīkha}             \\
7           & \textit{Capsicum annuum}           & chile            & 辣椒               & \textit{làjiāo}          & فلفل حار        & \textit{fulful hārr}         \\
8           & \textit{Cinnamomum verum}          & cinnamon         & 錫蘭肉桂             & \textit{xīlánròuguì}     & قرفة            & \textit{qirfa}               \\
9           & \textit{Syzygium aromaticum}       & clove            & 丁香               & \textit{dīngxiāng}       & قرنفل           & \textit{qaranful}            \\
10          & \textit{Coriandrum sativum}        & coriander        & \tradchinesefont{芫荽}               & \textit{yánsui}          & كزبرة           & \textit{kuzbara}             \\
11          & \textit{Cuminum cyminum}           & cumin            & 孜然               & \textit{zīrán}           & كمون            & \textit{kammūn}              \\
12          & \textit{Anethum graveolens}        & dill             & 蒔蘿               & \textit{shíluó}          & شبت             & \textit{shibitt}             \\
13          & \textit{Foeniculum vulgare}        & fennel           & 茴香               & \textit{huíxiāng}        & شمر             & \textit{shamar}              \\
14          & \textit{Trigonella foenum-graecum} & fenugreek        & 胡蘆巴              & \textit{húlúbā}          & حلبة            & \textit{ḥulba}               \\
15          & \textit{Zingiber officinale}       & ginger           & 薑                & \textit{jiāng}           & زنجبيل          & \textit{zanjabīl}            \\
16          & \textit{Piper longum}              & long pepper      & 蓽撥               & \textit{bìbō}            & دار فلفل        & \textit{dār filfil}          \\
17          & \textit{Myristica fragrans}        & mace             & \tradchinesefont{肉荳蔻皮}             & \textit{ròudòukòupí}    & بسباسة	& \textit{basbāsa} \\
18          & \textit{Myristica fragrans}        & nutmeg           & \tradchinesefont{肉荳蔻}          & \textit{ròudòukòu}       & جوز الطيب       & \textit{jawz al-ṭīb}         \\
19          & \textit{Piper nigrum}              & pepper           & 胡椒               & \textit{hújiāo}          & فلفل            & \textit{filfil, fulful}      \\
20          & \textit{Crocus sativus}            & saffron          & 番紅花              & \textit{fānhónghuā}      & زعفران          & \textit{zaʿfarān}            \\
21          & \textit{Zanthoxylum bungeanum}     & Sichuan pepper   & 花椒               & \textit{huājiāo}         & فلفل سيتشوان    & \textit{filfil sītshuwān}    \\
22          & \textit{Illicium verum}            & star anise       & 八角               & \textit{bājiǎo}          & ينسون نجمي      & \textit{yansūn najmī}        \\
23          & \textit{Curcuma longa}             & turmeric         & 薑黃               & \textit{jiānghuáng}      & كركم            & \textit{kurkum}              \\
24          & \textit{Vanilla planifolia}        & vanilla          & 香草               & \textit{xiāngcǎo}        & فانيليا         & \textit{fānīliyā}\\ 
% \midrule
%  & & & & & & \\ \midrule
% 25          & \textit{Physeter macrocephalus*} & ambergris        & 龍涎香              & \textit{lóngxiánxiāng}   & عنبر            & \textit{ʿambar}          \\
% 26          & \textit{Cinnamomum camphora}     & camphor          & 樟                & \textit{zhāng}           & كافور           & \textit{kāfūr}           \\
% 27          & \textit{Moschus moschiferus*}    & musk             & 麝香               & \textit{shèxiāng}        & مسك             & \textit{misk}            \\
% 28          & \textit{Boswellia sacra}         & frankincense     & 乳香               & \textit{rǔxiāng}         & لبان            & \textit{lubān}           \\
% 29          & \textit{Commiphora myrrha}       & myrrh            & 沒藥               & \textit{mòyào}           & مر              & \textit{murr}            \\
% 30          & \textit{Santalum album}          & santalwood       & 旃檀               & \textit{zhāntán}         & الصندل          & \textit{ṣandal}          \\ 
\bottomrule
\end{tabularx}
% \end{adjustbox}
\caption[The set of 24 spices included in this project.]{The set of 24 spices included in this project.}
\end{table}

\setlength{\tabcolsep}{6pt}



% 8. How is the thesis laid out?

% 6. What are your epistemological and ontological positions?

% 7. What is your contribution to the field?
The main contribution of this thesis would be a working database of spice names that can serve as a basis for further study. Spices an aromatic products with varying importance and relevance in different places and in different times are essentially endless, so there is always a room (and need) to expand. This dataset is to be grounded in the following principles: correct botanical identification of a plant and the obtained substance; awareness of the substance's physical and botanical properties, origin, spread, history, uses and cultural/religious significance; collection of names denoting the substance in the literature, including pre-modern periods; reviewable by marking sources and references. The fundamental idea is that these information can tell us a story from a new angle: by tracing the diffusion of spices and their names we can potentially find patterns in trade, contact, and blabla. 

Besides this, a linguistic analysis on the names attributed to a spice product will shed light on blabla




An attempt to group and categorize the aromatic materials of the spice domain based on linguistic-cognitive features has not yet been made and constitutes an original approach. The quest for exploring patterns of spice diffusion and spice terminology could yield new insights and open possibilities for future research. 






















% 1.4. Objectives
% In the previous subsections, we briefly introduced Wanderwörter, innovation, and sensory language surrounding the realm of spice. Now we shall review our attempt to connect these concepts, establish our approach based on the theoretical framework, and concisely lay out the objectives of this dissertation.
% The goal of this thesis is to explore Wanderwörter and see what they can tell us about human contact, language, and cognition. To achieve this, our objectives are:
% A. Explore the `diffusion of spices', examining the geographical spread via trade---looking at directionality, speed, quantities, means of transmission (by land or sea)---and trace the propagation of accompanying Wanderwörter to the best we can with the available corpus data and existing philological literature.
% B. Explore the `language of spices', examining surrounding terminology related to the sensory domains (gustatory, olfactory, visual, tactile), and looking at the presence or absence of different linguistic categories---spice names acting as nouns, and derivationally related nouns, adjectives, verbs, and idiomatic and metaphoric expressions with the help of selected corpora from different languages.
% C. Compare the findings of the above two perspectives, juxtapose the mechanisms of material spread and Wanderwort propagation (`diffusion of spices') with patterns of linguistic behaviour (`language of spices'), and look for patterns and correlations.
% In pursuit of Objective A, we make use of existing theories related to diffusion and innovation, while in tackling Objective B, we adapt concepts arising from the theories of semantic primes and linguistic synaesthesia. In Objective C, we propose an original approach to examine spices which hopefully yields results that can prove or refute our hypotheses and answer the research questions. We should emphasize again that products of the spice domain are fortunate subjects because they embody the physical and the linguistic novelty as well; they are archetypal Wanderwörter, and we aim to `measure' their embeddedness in a culture by the degree of prevalence of more diverse terminology and collocations in the relevant language.
% During the course of this project, we plan to collate a linguistics infused account of the spice trade, with the help of sporadic but valuable information gathered from the literature, and the etymological knowledge amassed while working on the dissertation. A comprehensive, curated database of spice names could be, without doubt, a significant addition to many types of future research on spices.
% The thesis will focus on three major languages: English, Arabic, and Chinese---all representing areas and civilizations that had important shares in the spice trade throughout history. By the use of historical corpora, we will be able to look at snapshots from the distant past, and glance at the linguistic and historic situation surrounding spice use.
% In investigating the realm of spices, we expect to observe trends that are typical of innovation, both in lexical and practical adoption. If the existence of diachronic data allows us to see that the diffusion of the spice products resembles an S-shaped curve, we could possibly propose a model for Wanderwort propagation, and present a thesis rich in practical, empirical, and theoretical contributions. One that adds to the understanding of the nature of wandering words, as evidence for the exchange of new ideas and contact between civilizations.
% The following section deals with relevant literature and sources, after which we shall define the research questions. Then we arrive to the second half of this proposal and determine a Research plan, where we reveal the methods to be used in this study, followed by a discussion of expected results.