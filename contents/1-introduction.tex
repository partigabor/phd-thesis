\chapter{Introduction}
\label{ch:introduction}

Upon reading about the different spices that nature has gifted us with, I have come across a unique one. There is a small aromatic tree native to the islands of the Greater Antilles on the Caribbean, bearing little berries. The indigenous people used this tree and its fruit in various ways in the stages of food preparation, as a medicine and flavoring agent. It is still an important crop today, only growing in Central America. The wood is used to smoke meat, the leaves are added to stews and rum for their aroma (similarly to the bay-leaf in Europe), but most importantly the dried berries are used as a spice. Outside this region it is not a particularly well-known ingredient, still, many cuisines use it in various ways since its diffusion starting from the seventeenth century. In English, we can call it \textit{Jamaica pepper} or \textit{pimento}, but it is mostly known today as \textit{allspice}. In Arabic, it is \textit{fulful ifranjī} literally [Frankish pepper] meaning `European pepper' or \textit{bah\={a}r ḥulw} [sweet spice], and in Chinese, it is mostly known as \textit{duoxiangguo} [many-spice/fragrant-fruit]. Furthermore, Hungarians call it \textit{szegfűbors}[clove-pepper], in Turkish it is \textit{yenibahar} [new-spice], in Iceland it is known as the `handy spice', \textit{allrahanda}, and the Polish call it \textit{ziele angielskie} [English herb]. What a variety! The tree itself is also called Caribbean laurel. However, this plant is not a laurel, it is not a pepper, nor chili, and it is not an herb. What is this versatile spice exactly? How come that this material has so many so different names? And what is the explanation and motivation behind these names; what is their story? This dissertation is about answering such questions and telling the story of spices through examining their nomenclature.

\section{Significance}

Understanding the language of spices is a key to open a door to this world. A door that leads to the realization that our words---and material culture---are deeply interconnected, and that they have been so for thousands of years. I will try to demonstrate this by introducing these fascinating substances from a new perspective, the perspective of language. It is trendy nowadays to talk about \textit{foodways}, a term that refers to ``the eating habits and culinary practices of people, regions, or historical periods'' \pvolcite[]{2}[289]{allen_encyclopedia_2013}, and food history, a relatively young interdisciplinary academic field is starting to gain traction as well. However, the connections between language and food are one of the most interesting examples of contemporary humanities research I have come across \autocite[see][]{jurafsky_language_2014}. There is a segment of this topic---the spice domain---which encompasses products that have had profound effects of human imagination, culture, and history. Although somewhat overshadowed by more ``serious'' questions of nutrition, modern scientific research on spices was never a fringe field. It is enough to look at the many pharmacological studies that dive into the chemistry of these materials to realize that people are still interested in their health effects---just as they were thousands of years ago---as much as their taste and aroma.

Spices were never a necessity to human survival, but commanded intense desire and efforts, and as such they  constitute an enthralling phenomenon throughout human history, which can be studied from many angles. Therefore, research on spices has been embraced by a few historians, many botanists, and countless culinary enthusiasts. Scholars also realized the cultural significance of these products and their trade early on and following the path of these materials tried to uncover the stories they can tell us about cultural contact and exchange. Spices in the past conveyed the mystery and riches of faraway lands, they were considered remedies for sickness, and they were the ultimate gifts of paradise. It may be so that spices are not vital for humans, but sustenance itself is just enough to maintain life, not to live it to the fullest. Spices today represent the excitement, the joy and vigor, as it is so clear from expressions in our language: to \emph{spice up} your life is to enliven it!



\section{Originality}

% 1. What is your original contribution? 
This thesis aims to do a systematic investigation on spice names and related terminology, including products that were used (or still being used) medicinally, as incense, or as perfume. Aromatics that were at some point considered spices have been traded and transported across long distances since antiquity and before, and the most coveted ones have slowly dispersed throughout the globe. Spices and the spice domain as a topic are usually discussed within the broad disciplines of history, botany, chemistry, and gastronomy, all concerned by very different aspects of these materials. To the best of my knowledge, there is no academic work that puts the field of linguistics in focus when discussing spices as a whole, and so this project is a unique contribution to our linguistic knowledge about the spice domain.

% 2. Why should the examiner care about your research?
But why should anyone care about spices and their names? Because exploring the names of the products of the spice trade---traveling on vast networks of historic trade routes such as the Silk Road (small volume of trade), and its nautical counterpart the Maritime Silk Road (large volume of trade)---helps us to map and better understand linguistic contact and cultural exchange. These ever-expanding trade networks, first regional, later connecting East and West were a precursor to today's globalized, interconnected world, and one of their most lucrative products was dried plant-matter. These aromatic substances were lightweight, easy to transport, and resistant to spoilage. And, of course, they were highly valued for their fragrant and pungent properties, and their reputed---both putative and real---benefits for the human body and soul. Exotic and rare spices and their role in rituals, medicine, and later cuisine made them sought after, and the spice business inspired people to trade, travel, explore, and wage wars. Spices are important in world history as they are directly responsible for discoveries, colonization, and the birth of capitalism. We know a great deal about the nature of spices thanks to botanists and naturalist, their medicinal effects thanks to pharmacists and chemists, and their uses and culinary values thanks to experts of gastronomy. There is also a vast literature on the spread and economy of spices thanks to researchers of history, but the careful study of their names is often neglected. The meticulous study of spice terminology is important, because the words can tell us a huge deal about the material's origins and journey, even at a time depth where textual or archeological evidence is not available. This work was born out of fascination with the etymologies and global dispersion of spice terms, and hopefully the attempt to collect them in one study can be original and interesting.

\section{Problem Statement}

% 3. What is the thesis problem statement?
Soon, my attention slightly shifted towards a problem that could be best described by a lack of consistent and comprehensive knowledge regarding spice names. I noticed a gap when it comes to spice terminology, as very few scholars have turned their focus towards the nomenclature. The absence of proper research regarding spice terminology have resulted in a lack of understanding, and a decline of trust in the (secondary) literature. Authors often give misguided and contradicting information regarding the source of a name, or simply erroneously speculate on their meanings and origins. There are no two authors that use the same set of names when discussing a spice, which in itself is not a problem, but it leads to issues in case of lesser known or exotic items. There is also a great deal of confusion on the relationship between names and identities in the literature, especially in lay areas aimed at the general public, such as popular histories or guidebooks. The reasons for this are several.

Firstly, the experts of herbs, spices, and other aromatics are chiefly botanists, food industry professionals, chemists, chefs and food writers, merchants and historians. Most people conducting research related to spices focus on aspects other than the names of the products; from plant morphology, to chemical composition and pharmacological effects, to social and cultural histories, their symbolism in literature, not to mention the myriad of ways on how to buy, store, mix, and use spices in creative recipes as discussed by the handy spice encyclopedias tailored for gastro-enthusiasts. Relatively few linguists devoted their time and attention to trace spice origins and even fewer to compile them. In other words, the topic of spices requires a highly multidisciplinary expertise, and when a plant taxonomist writes about linguistics, or a culinary writer approaches history, some mistakes are due.

Secondly, there is no reference work or an agreed upon inventory of spice names to cover the multitude of spices that exist, and their many names in various languages (and in different time periods), least of all a complete list of \emph{every spice}. Truthfully that seems rather impossible, or at least quite a daunting task to embark on. Although the internet nowadays is full of compact guides and indices assembled by people who are fascinated with spices and their colorful uses listing their names in many languages, these are not always trustworthy, and often cite no sources. Similarly, blogs and articles are most often than not dubious, and almost always require fact checking as many are just permutations of historically inaccurate anecdotes and origin stories. Most recently, I have found spice lists that are clearly just a dump of computer generated information floating around on the internet using web scraping and machine translation: these lists are highly inaccurate and should be avoided. I will further elaborate on the inaccuracies regarding names and etymologies found in the spice literature under the literature review in the next chapter.

There is no comprehensive treatise on spice terms within academia until today, and no database that focuses on, lists, explains, analyzes, clarifies, traces, or compares spice names. Hence, there is an obvious need for a standard work for others to turn to, and I hope that this dissertation can lay the foundation for future efforts to achieve this.

This is not to say that there is no work done on spice terminology, there are a number of high-quality writings from philologists, linguists, and historians well versed in one or more relevant linguistic and cultural area. The problem is that this kind of research requires a highly specialized knowledge, and in result the information already out there is sporadic, less accessible, and grossly unorganized. Key pieces of information are often hidden between the pages of books on traditional philology and material culture, literary critique, economic history, and even botany, medicine, and archaeology of a given region. Not to mention the many old works that are the primary sources for the aforementioned publications. Consequently, since little effort have been made to collate the data, there is a chasm between the critically researched reliable information and what the end user---whether it is a fellow researcher or a spice zealot---can easily access.



\section{Aims}
% Rewrite

% 4. What do you (not) hope to achieve?
The original goal in the beginning of this work was to gather and augment the existing information about spice names and their origins, and track their diffusion on spatial and temporal trajectories. This still constitutes the core of this thesis, and I hope to achieve it by combing through the existing literature, collecting the names of spices, amending the gaps, and correcting possible errors on the way. By doing so, the results should also give birth to a carefully researched compendium of spice nomenclature, grounded in philology and linguistics, but with the awareness of what spices are to botany, and what their role was in history. \Cref{ch:data} \nameref{ch:data} presents you parts of this process and displays the data seriatim, and introduces some of the spices.

The procedure shall manifest a dataset of spice names, with complete lexicographical annotation including etymological information and attestation dates. This in turn would allow me to trace the words and track the linguistic diffusion of spices through space and time, which then can be discussed hand-in-hand with the physical diffusion of the materials. Eventually, the spread of the spices will be the basis for a discussion on the implications of linguistic and cultural contact and exchange, and it makes up \cref{ch:diffusion} \nameref{ch:diffusion}. This chapter ties well together with the concept of \glspl{wanderwort}, ``wandering loanwords'', a phenomenon known in the field of historical linguistics related to the topic of borrowing and material culture. The goal of this chapter is to map the diffusion of the terms of the spice domain, as informed by etymological research.

In addition to this, the data of spice names will also be the basis of a linguistic analysis, focusing on the characteristics of terms, and the methods and strategies languages use to name spices as presented in \cref{ch:language} \nameref{ch:language}. This part will include a deep dive into how spice names are devised or created, how prototype items beget prototype words to generate new names for novel items of trade, and what are the mechanisms and motivations of linguistic acculturation and spice name propagation. The goal of this part is to shed light in how spice names are born and how they operate on linguistic-cognitive levels, what type of information they convey and what sensory modalities they tie to.

% More on this part?

% More on this part? Rewrite

% 5. What are the research questions and hypotheses?

\section{Definitions}
\label{sec:definitions}

The first step is to clarify what is meant under the term \textit{spice}. According to the \gls{OED}, the definition of \textit{spice} is as follows: ``One or other of various strongly flavored or aromatic substances of vegetable origin, obtained from tropical plants, commonly used as condiments or employment for other purposes on account of their fragrance and preservative qualities.''\footcite[spice]{oed} Similarly, the first meaning for \textit{spice} as a noun in \gls{MW} is ``any of various aromatic vegetable products (such as pepper or nutmeg) used to season or flavor foods.''\footcite[spice]{mw} The \gls{AHD} adds examples: ``Any of various pungent, aromatic plant substances, such as cinnamon or nutmeg, used to flavor foods or beverages.''\footcite[spice]{ahd} 
The Wikipedia entry on \textit{Spice} gives slightly more information, hinting on which plant parts are frequently used as spices and mentions their food-coloring properties, while also---very appropriately---ventures beyond the culinary stance of usual dictionary definitions, stating that ``spices are sometimes used in medicine, religious rituals, cosmetics or perfume production.''\footcite{wikipedia_spice_2022} This notion is much more important than expressing it with a mere `'`sometimes'' could imply as we will see; before modern times, spices were much more important for their medicinal properties.

There is no universal definition on what a spice is; botany, pharmacology, gastronomy, and history all have different perspectives. However, the idea about ``spices'' that the reader currently has in mind, is bound to be a culinary one. Some authors try to give a definition according to plant morphology, \textcite[9]{czarra_spices_2009} writes about ``an aromatic part of a tropical plant'', and goes on to mention bark, flower, root, and seed. \textcite[xix]{turner_spice_2004} adds gum and resin, fruit, and stigma to this listing. For a full picture, we must complement it further, as spices can come in many forms: dried tree barks (cinnamon); twigs
(cassia twigs); flower buds (cloves); stigmas or styles (saffron); fruits (pepper, chili); fruit walls or pericarps (star anise, Sichuan pepper); berries (allspice, juniper); seeds (nutmeg, coriander); seed coverings or arils (mace); seed pods (cardamom, vanilla); and roots and rhizomes (ginger, turmeric). Technically, every dried part of a plant can be referred to as spice, except the leaves. The green leaves---fresh or dried, but mostly used fresh---are considered herbs, and they are used for similar purposes to spices nowadays: flavoring, seasoning, garnishing. Dried leaves of herbs, such as basil, oregano, and thyme, can be categorized as ``spice herbs'' \autocite[see][]{van_wyk_culinary_2014}. The category of herbs can be problematic, because there is a botanical definition, and also a culinary definition, and the literature often confuses the two. Botanically, an herb is an annual, biennial, or perennial plant that has a soft stem (never becomes woody), and dies after flowering.\footcite[herb]{oed} A culinary herb is an herb where the fresh leaves are used in food preparation, as opposed to other, dried plant parts. Medicinal herbs are those that are primarily consumed for their medicinal properties. \textcite[9,16]{oconnell_book_2016} backs this view in his informative compendium, but also cites \textcite[16]{rosengarten_book_1973}, who maintained that it is ``extremely difficult to determine where a spice ends and a herb begins''. According to him, culinary herbs are just one group of spices. Along these lines, the Encyclopedia \textcite{britannica_spice_2022} for example treats herbs and spices in a single entry. 

The above distinction---that herbs are the greens and spices are every other (dried) parts of a plant---is essentially nonsense to a botanist since it echoes the needs of a chef. We can give examples for both spice and herb from the same species: coriander seeds and coriander leaves (also called \textit{cilantro} or \textit{Chinese parsley} in the U.S.) are both from the plant \textit{Coriandrum sativum}.
Or see an even more elaborate example for culinary categorization: mustard (\textit{Brassica juncea}, brown mustard). Mustard in Europe is mostly known in the form of a creamy yellow paste, often very pungent, this is called a condiment. This condiment is made from the mustard seeds, which are considered to be the spice. Some regions enjoy mustard greens in their salads for example, here we can categorize it as a (salad) herb, and the roots of the mustard plant are popularly fermented in Chinese cuisine to make pickles, known as \tc{榨菜} \textit{zha cai} [to press, extract juices, salted vegetable-vegetable, dish]. Another often mentioned difference is that herbs are soft stemmed---as I just mentioned---that die at the end of the growing season in contrast to woody, spice yielding trees or shrubs. This, on the other hand, is a botanical definition, and not very useful for somebody active in the culinary arts \autocite[10]{allen_herbs_2012}. Moreover, most plants we consider herbs grow in temperate climates, while spices tend to grow in tropical regions \autocite{turner_spice_2004}, a further classification on ecological basis. Herbs can also be divided into culinary and medicinal herbs, and in both cases, the leafy, green parts of foliage are used for their aroma and flavor, and supposed health benefits, respectively, but this division is a modern afterthought, not applicable to past times. In short, defining spices and herbs is difficult because the definitions vary by discipline, depending on the needs of the expert: the gardener, the herbalist, the chef. 

Just as \textit{herb} has two main definitions---botanical and culinary, the term \textit{spice} also has two definitions: a culinary, and a historical. In the present study, I focus on dried, plant-based aromatic commodities that traveled long distances due to trade and were at certain points in history considered a desired commodity or even a luxury. This is basically the definition of the historian, where the implications of climate and remoteness translated as value. Spices were difficult to obtain, and thus were often instantly expensive. Long distance transportation was possible because the plant products were hauled across deserts and oceans in a dried form, making them lightweight and less susceptible to rot. Culinary and medicinal herbs had their value in their freshness, and thus were not ideal products of trade; they spread through naturalization and transplantation and were generally available locally, commanding much more modest sums of money. Contrary to herbs, spices were more versatile in their applications, they were used as medicine due to their (real, putative, or exaggerated) health benefits, as incense and perfume due to their aroma, as coloring pigments, as flavoring agents, and even as spiritual offerings. In connection with Portugal's role in the fifteenth--sixteenth-century European spice trade, \textcite{halikowski_smith_portugal_2001} distinguishes four categories of spices along the same lines: spices as \textit{pharmacia}; spices as \textit{aromata}; spices as \textit{pigmenta}; spices as \textit{condimenta}. Historically speaking, anything rare and aromatic can be considered a spice, including incense for burning, coffee in the early days, chocolate, perfumes, or even exotic fruits, such as pomegranates; anything `special' (even if today nobody would agree so).

This is well observed in the origins of the English name: the word \textit{spice}, via Old French \textit{espice}, coming from Late Latin \textit{speciēs} (plural) `spices, goods, wares' with the original meaning in Classical Latin being `kind, sort'. English \textit{species} and \textit{special} are obvious cognates of the same Latin etymon: \textit{speciō}, which referred to anything observable: a sight, `spectacle' (cf. \textit{inspect}), and also anything extraordinary, \textit{specific} kind of item \autocite[1983-84]{glare_oxford_2012}. This implies that in different periods, the meaning of the term \textit{spice} covered different substances, based on what products were considered special, desirable, and difficult to obtain; the definition constantly changed. From this point on, whenever spices are mentioned, I refer to this broader definition, using \textit{spice} as an umbrella term for any historic \textit{exotica}. These definitions and the differences between the terms spice, incense, herb, condiment, etc. will be explored in detail in the dissertation, as well as the shift in meaning considering spices (\cref{shift}).

% Spices in Arabia,
% Spices in China

% in Wyk:
% Salad herbs
%     Potherbs
%     Micro greens
% Culinary herbs
% Spice herbs
% Spices
% Spice mixtures
% Seasonings \& Condiments
%     Seasonings
%     Condiments
%     Sauces
%     Dips
%     Pickles and preservatives
%     Essences
%     Vinegar
%     Herbal extracts and liqueurs
%     Food coloring
%     Garnish


\section{Scope}

The scope of this study can be delimited by three factors: the subjects under study (i.e., the spices); the languages involved; and the time frames covered.

\subsection{The Set of Spices}

Due to the lack of a single definition and the vast number of both popular and rare materials, the lists of herbs and spices are as many as the books written on them. Every compiler---cook, linguist, or historian---comes up with his or her own list, there is no standard approach. \textcite{czarra_spices_2009} discusses only 5 premier spices, `the foremost five' as he calls it, \textcite{nabhan_cumin_2014} works with 26 `spice boxes', to use his terminology, and \textcite{rosengarten_book_1973} mentions 41 `of the world's most popular spices'. \textcite{van_wyk_culinary_2014} in his reference guide introduces 120, and the glossary of spice names in \textcite{dalby_dangerous_2000} contains more than 200 entries. The Encyclopedia Britannica simply lists 70 \autocite{petruzzello_list_2021}. 

The attempt to collect and describe spices in their entirety is not a modern phenomenon, people tried to collect and assemble the knowledge on aromatic and medicinal plants and products ever since the day we started to use them; proof being the countless \textit{herbariums},\footnote{A collection of preserved plants with descriptions and botanical analysis.} \glspl{materia medica}, and \glspl{pharmacopeia} of the last two millennia, e.g., the \gls{DMM} of Dioscorides. Besides focusing on medicinal properties, spices were listed as commodities of trade. We know for example that for reasons motivated by customs tariff and taxation, Roman ports, such as Berenike\footnote{For the history of the ancient Indo-Mediterranean trade through archeological research in the port of Berenike, see \textcite{sidebotham_berenike_2011}.} had inventories of \textit{speciēs} that required special attention \autocite{parthasarathi_roman_2015}, and as a later example we can mention an Italian merchant's book on trade from the fourteenth century, which contains 188 ``spices'' \autocite[411-435]{pegolotti_pratica_1936}. Modern considerations and regulations related to commercial shipping can also be consulted, the Harmonized System (HS)4 tariff codes (0901-0910) used in international trade differentiates around 20 different spices, mostly driven by practical reasons related to storage and transportation requirements.\footnote{The Harmonized Commodity Description and Coding System is a standardized international system to classify globally traded products (\url{https://www.freightos.com/freight-resources/harmonized-system-code-finder-hs-code-lookup/}).}

% 0901 Coffee, whether or not roasted or decaffeinated, husks and skins, coffee substitutes
% containing coffee in any proportion
% 0902 Tea
% 0903 Mate
% 0904 Pepper of the genus piper, dried or crushed or ground fruits of the genus capsicum or of
% the genus pimenta
% 0905 Vanilla
% 0906 Cinnamon and cinnamon-tree flowers
% 0907 Cloves (whole fruit, cloves and stems)
% 0908 Nutmeg, mace and cardamoms
% 0909 Seeds of anise, badian, fennel, coriander, cumin, caraway or juniper
% 0910 Ginger, saffron, tumeric (curcuma), thyme, bay leaves, curry and other spices

A comprehensive list of the world's spices is likely to be extremely long and collecting them all in this short period of time seems to be an unattainable aim, so I must limit my scope, I must select a small set of spices from the many. I have studied how others perform this selection, and I concluded that unless we include every known item, there is no scientifically sound way of selecting just a few, it will always reflect the compilers' knowledge, interest, and cultural background. I have planned to include a number of spices, as much as needed for an attempt to find patterns in diffusion and language use, keeping in mind the expected variations per cuisine and cultural sphere. 



\setlength{\tabcolsep}{3pt}

\begin{table}[ht]
\caption[The set of 24 spices included in this thesis.]{The set of 24 spices included in this thesis, with scientific names of the source plant, names in English, Chinese, Arabic, and their transliterations.}
% \begin{adjustbox}{max width=\textwidth}
\begin{tabularx}{\textwidth}{@{}r>{\footnotesize}llll@{}rl@{}}
\toprule
\textbf{\#} & \multicolumn{1}{l}{\textbf{Species}} & \textbf{English} & \textbf{Chinese} & \textbf{Translit.} & \textbf{Arabic} & \textbf{Translit.}     \\ \midrule
1           & \textit{Pimenta dioica}            & allspice         & \tc{多香果}              & \textit{duōxiāngguǒ}     & فلفل إفرنجي     & \textit{filfil ifranjī}      \\
2           & \textit{Pimpinella anisum}         & anise            & \tc{茴芹}               & \textit{huíqín}          & ينسون           & \textit{yansūn}              \\
3           & \textit{Ferula assa-foetida} et al.& asafoetida       & \tc{阿魏}               & \textit{āwèi}            & حلتیت           & \textit{ḥiltīt}              \\
4           & \textit{Carum carvi}               & caraway          & \tc{葛縷子}              & \textit{gělǚzi}          & كراويا          & \textit{karāwiyā}            \\
5           & \textit{Elettaria cardamomum}      & cardamom         & \tc{荳蔻}            & \textit{dòukòu}          & هال             & \textit{hāl}                 \\
6           & \textit{Cinnamomum cassia}         & cassia           & \tc{肉桂}               & \textit{ròuguì}          & سليخة           & \textit{salīkha}             \\
7           & \textit{Capsicum annuum} et al.    & chile            & \tc{辣椒}               & \textit{làjiāo}          & فلفل حار        & \textit{fulful hārr}         \\
8           & \textit{Cinnamomum verum}          & cinnamon         & \tc{錫蘭肉桂}             & \textit{xīlánròuguì}     & قرفة            & \textit{qirfa}               \\
9           & \textit{Syzygium aromaticum}       & clove            & \tc{丁香}               & \textit{dīngxiāng}       & قرنفل           & \textit{qaranful}            \\
10          & \textit{Coriandrum sativum}        & coriander        & \tc{芫荽}               & \textit{yánsui}          & كزبرة           & \textit{kuzbara}             \\
11          & \textit{Cuminum cyminum}           & cumin            & \tc{孜然}               & \textit{zīrán}           & كمون            & \textit{kammūn}              \\
12          & \textit{Anethum graveolens}        & dill             & \tc{蒔蘿}               & \textit{shíluó}          & شبت             & \textit{shibitt}             \\
13          & \textit{Foeniculum vulgare}        & fennel           & \tc{茴香}               & \textit{huíxiāng}        & شمر             & \textit{shamar}              \\
14          & \textit{Trigonella foenum-graecum} & fenugreek        & \tc{胡蘆巴}              & \textit{húlúbā}          & حلبة            & \textit{ḥulba}               \\
15          & \textit{Zingiber officinale}       & ginger           & \tc{薑}                & \textit{jiāng}           & زنجبيل          & \textit{zanjabīl}            \\
16          & \textit{Piper longum}              & long pepper      & \tc{蓽撥}               & \textit{bìbō}            & دار فلفل        & \textit{dār filfil}          \\
17          & \textit{Myristica fragrans}        & mace             & \tc{肉荳蔻皮}             & \textit{ròudòukòupí}    & بسباسة	& \textit{basbāsa} \\
18          & \textit{Myristica fragrans}        & nutmeg           & \tc{肉荳蔻}          & \textit{ròudòukòu}       & جوز الطيب       & \textit{jawz al-ṭīb}         \\
19          & \textit{Piper nigrum}              & pepper           & \tc{胡椒}               & \textit{hújiāo}          & فلفل            & \textit{filfil, fulful}      \\
20          & \textit{Crocus sativus}            & saffron          & \tc{番紅花}              & \textit{fānhónghuā}      & زعفران          & \textit{zaʿfarān}            \\
21          & \textit{Zanthoxylum spp.}          & Sichuan pepper   & \tc{花椒}               & \textit{huājiāo}         & فلفل سيتشوان    & \textit{filfil sītshuwān}    \\
22          & \textit{Illicium verum}            & star anise       & \tc{八角}               & \textit{bājiǎo}          & ينسون نجمي      & \textit{yansūn najmī}        \\
23          & \textit{Curcuma longa}             & turmeric         & \tc{薑黃}               & \textit{jiānghuáng}      & كركم            & \textit{kurkum}              \\
24          & \textit{Vanilla planifolia}        & vanilla          & \tc{香草}               & \textit{xiāngcǎo}        & فانيليا         & \textit{fānīliyā}\\ 
\bottomrule
\end{tabularx}
% \end{adjustbox}
\label{table:set}
\end{table}

\setlength{\tabcolsep}{6pt}


As aromatic plant products are far greater in number than initially expected (in the hundreds), I had to delimit my study by selecting a small set of spices as the subjects of this study. One of the clearly intended future goals of this project is to include all known historically important aromatic products, but for the sake of this dissertation be finished on time, I had to make some choices, and I have set up a list of criteria. The subjects under scrutiny shall adhere to the followings: (1) a plant-based material, with (2) aromatic or pungent properties, (3) traded over great distances, (4) known and used since pre-modern times, and preferably (5) once of significant value. The more criteria the substance has fulfilled, the higher it climbed on my list as I assembled a set of spices while reading through the literature. Most importantly, I tried to include globally popular spices that people can be familiar with. In the end, I have finalized my set including 24 spices for this thesis, which you can see in \cref{table:set}. 

% This means that besides plant-based spices, we can include biotic materials not intended for oral consumption: incense (usually from tree resin and oil) such as myrrh, frankincense, or sandalwood, and substances of animal origin, such as musk or ambergris.

In the dissertation, I have followed the criteria laid out above, and included the most prominent spices, for example black pepper (\textit{Piper nigrum}). Pepper is the world's most traded and consumed spice, followed by cinnamon \autocite[16]{senaratne_cinnamon_2020}. In many parts of the world, pepper is also the \textit{prototype} spice. Considering popularity, I have worked downwards, considering the most consumed, and most traded spices. The current set of 24 in my opinion is enough to grasp the big picture, as it includes a large variety of spices from relatively\footnote{Keeping in mind that almost all spices around the world originate from regions with tropical or subtropical climates.} diverse areas. In the future I hope to extend this project and include more aromatic materials to the fold.

The scientific literature usually categorizes spices by their taxonomy governed by rules from the discipline of botany. Although above I mentioned standards of international trade and shipping that use different categorizations, I discussed spices from a historical perspective, and I should emphasize that this is a linguistic study and therefore the basis of comparison will be the spices as conceptual categories revolving around words, and the spice names themselves. Strict botanical or other categorizations would be only secondary. What I mean under this is that, even if the botanical reality tells us that chile peppers are in fact several different species and varieties on a taxonomic ladder, I will treat them as a unit according to the broad sense the term \textit{chili pepper} conveys. Therefore, the close organizational units are words, meanings, and concepts. I will clear this up more in the methodology chapter. Now that we have identified the subjects in our scope, let us move on to the languages to be included, and the time frames to be covered.

% More precisely the word senses we collect and identify with the help of corpora and lexical-semantic databases, such as the WordNet6 (Miller, 1995) 
% 6 Accessible at https://wordnet.princeton.edu/

\subsection{The Languages}

To make this study more interesting, I will look at three languages, namely English, Arabic, and Chinese. All from distant language families, with very distinct features, these languages represent three very different cultural spheres that had all participated in the spice trade in different historical periods and were at times highly influential players. Investigating spices and spice names through the lens of these languages offers us an opportunity to take a step back from the usual Eurocentric viewpoint, and observe the spice domain from a more global, intercultural perspective, investigating it via three pillars of human high culture as equals. These civilizations not only had interactions with each other, but their representative and languages---rich in literary traditions---have left a mark on many of the world's languages through migration, trade, cultural and religious influence, colonization, and imperialism. By tracing the words' histories up until contemporary English, Arabic, and Chinese, I must touch on the many other languages that have played a role in transmitting the words as source and donor languages in the spice domain, such as Latin, Persian, Sogdian, Sanskrit, and many more. Focusing on the three languages however, will allow for a comparative approach, without overwhelming ourselves with the many related languages that would make for a scope too wide, and beyond my abilities.

% More on this? Language families?

\subsection{The Time Frame}

Lastly, a few words on defining the time periods to be covered. I will start the linguistic investigation using a set of contemporary names in the three languages mentioned above, and I will trace the histories of the words to the periods where the etymologies lead me. Consequently, the historical periods that saw the greatest exchange and knowledge production about spices will be treated with a much greater emphasis than others.

These significant eras for example include the seven to twelfth centuries, a period significant from both Arabic and Chinese points of view. In the Middle East we have seen the rapid expansion of Islamic empires in the form of the Umayyad (661--750) and Abbasid caliphates (750--1258), stretching over vast swathes of land from North Africa on the west, to the easternmost extremities of Persia. On the Far East, the Tang dynasty (618--907) illustrated the peak of classical Chinese civilization---usually regarded open and cosmopolitan---controlling the regions on the eastern terminus of the Silk Road, followed by the similarly important Song dynasty (960--1279). In both cases, this time is regarded as an important cultural and economic golden age: poetry, literature, science, and trade flourished. Important \glspl{materia medica} are extant from the Tang era \autocite{wu_illustrated_2005}, and Islamic authors were occupied producing heavy tomes on geography, alchemy, medicine, philosophy, and other topics \autocite[131]{meri_medieval_2006}. During the eighth century both powers reached each other's spheres of influence in Central Asia, and at 751 the Battle of Talas---ending with the victory of the Arab forces and their Tibetan allies---affected the fate of the region for centuries. 

In the case of English, the appropriate, matching historical period would be that of Old English, however, this earliest recorded period is not abundant in spice related terms, although attested wandering loanwords include \textit{pepper}, \textit{copper}, \textit{gem}, and \textit{mint} \autocite{wollman_early_1993}. In the case of English, the most relevant historical period would correspond to that from Middle English (ca. 1150--1450) to Early Modern English (late fifteenth to late seventeenth c.), where the dichotomy in the domain of spice terminology between contemporary and historic times is still observable. This is the historical period marked by England's slow emergence from the ragged periphery of Europe during the Age of Discovery, into becoming a global superpower thanks to its advances towards maritime supremacy, the zenith\footnote{From Arabic \textit{samt} `astronomical path'.} of which begins with the foundation of the East India Company in 1600, and later culminates in the expansion of the British Empire.

% 5. What are the research questions and hypotheses?

% 6. What are your epistemological and ontological positions?

% 7. What is your contribution to the field?

\section{Contribution}

The main contribution of this thesis would be a working database of spices and their accompanying names that can serve as a basis for further study in historical linguistics, typology, and more. Spices and aromatic products with varying importance and relevance in different places and in different times are essentially endless, so there is always a room (and need) to expand. This thesis treats these materials as unique witnesses of cultural interaction and tracks their physical and linguistic diffusion. Collecting the \glspl{wanderwort} of the spice domain and examining their linguistic history makes an interesting window to look at early human contact, exchange, and the dissemination of plants, materials, words, and knowledge.

Furthermore, a look into the role of spices' highly sensory nature can be an exciting point of study regarding human cognition and language, and especially our attitude towards naming and conceptualizing novel substances.

% language

% The fundamental idea is that the collected information can tell us a story from a new angle: by tracing the diffusion of spices and their names we can potentially find patterns in not only trade, but cultural and linguistic contact.

% Besides this, a linguistic analysis on the names attributed to a spice product will shed light on ...

% 8. How is the thesis laid out?

\section{Layout}

\Cref{ch:background} \nameref{ch:background} deals with the literature review, further explains the research gap, and introduces the theoretical framework, \cref{ch:methodology} \nameref{ch:methodology} introduces the research design principles, its challenges, and the data collection process regarding spices, names, and etymologies. \Cref{ch:data} \nameref{ch:data} introduces a few spices in more detail one by one, including their uses, botany, and history besides their names. The empirical chapters \cref{ch:diffusion} \& \cref{ch:language} will present \nameref{ch:diffusion} and \nameref{ch:language}. In the end, a short conclusion closes the thesis, with the mentioning of limitations and future plans.










