\begin{spice}\label{spice:dill}
\textsc{Dill} \hfill \href{https://powo.science.kew.org/taxon/837530-1}{POWO} \\
\textbf{English:} \textit{dill}. 
\textbf{Arabic:} {\arabicfont{شبت}} \textit{shibitt}. 
\textbf{Chinese:} {\tradchinesefont{蒔蘿}} \textit{shíluó}. 
\textbf{Hungarian:} \textit{kapor}.  \\
\noindent{\color{black}\rule[0.5ex]{\linewidth}{.5pt}}
\begin{tabular}{@{}p{0.25\linewidth}@{}p{0.75\linewidth}@{}}
Plant species: & \taxonn{Anethum graveolens}{L.} \\
Family: & \textit{Apiaceae/Umbelliferae} \\
Plant part used: & fruit; leaf \\
Region of origin: & Nort Africa; West Asia \\
Cultivated in: & India \\
Color: & greyish brown \\
\end{tabular}
\end{spice}