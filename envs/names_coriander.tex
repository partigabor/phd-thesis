\begin{table}[!ht]
\centering
\begin{tabularx}{\textwidth}{@{}ll>{\itshape}lLl>{\small}l@{}}
\toprule
\textbf{\#} & \textbf{Language} & \multicolumn{1}{l}{\textbf{Term}} & \textbf{Gloss} & \textbf{Loan} & \multicolumn{1}{l}{\textbf{Source}} \\
\midrule
1	& English	& Chinese parsley	& 	& no	& \textcite{oed} \\
2	& English	& cilantro	& 	& yes	& \textcite{oed} \\
3	& English	& coliander	& 	& yes	& \textcite{oed} \\
4	& English	& coriander	& 	& yes	& \textcite{oed} \\
5	& English	& coriander-seed	& 	& no	& \textcite{oed} \\
6	& English	& dhania	& 	& yes	& \textcite{oed} \\
\midrule
1	& Arabic	& kuzbara	& 	& yes	& \textcite{wehr_dictionary_1976} \\
\midrule
1	& Chinese	& húsuī	& barbarian-coriander	& yes	& \textcite{defrancis_abc_2003} \\
2	& Chinese	& xiāngcài	& fragrant-vegetable	& no	& \textcite{mdbg} \\
3	& Chinese	& yánsuī	& lilac-coriander	& no	& \textcite{mdbg} \\
\bottomrule
\end{tabularx}
\caption{Conventionalized names for coriander in English, Arabic, and Chinese, found in dictionaries.}
\label{table:names_coriander}
\end{table}

