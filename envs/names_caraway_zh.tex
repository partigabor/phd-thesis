\begin{table}[!ht]
\centering
\begin{tabularx}{\textwidth}{@{}l>{\itshape \small}ll>{\itshape}lL>{\small}l@{}}
\toprule
\textbf{\#} & \multicolumn{1}{l}{\textbf{Species}} & \multicolumn{1}{l}{\textbf{Name}} & \multicolumn{1}{l}{\textbf{Tr.}} & \multicolumn{1}{l}{\textbf{Gloss}} & \multicolumn{1}{l}{\textbf{Source}} \\
\midrule
\textbf{1}	& \textbf{Carum carvi}	& \textbf{\tradchinesefont{葛縷子}}	& \textbf{gělǚzi}	& \textbf{}	& \textbf{\textcite{kleeman_oxford_2010}} \\
2	& Carum carvi	& \tradchinesefont{頁蒿}	& yèhāo	& leaf-wormwood	& \textcite{mdbg} \\
3	& Carum carvi	& \tradchinesefont{藏茴香}	& zànghuíxiāng	& Tibetan-hui-spice	& \textcite{mdbg} \\
\bottomrule
\end{tabularx}
\caption{Various names for caraway in Chinese.}
\label{table:names_caraway_zh}
\end{table}

