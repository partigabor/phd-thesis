\begin{table}[!ht]
\centering
\begin{tabularx}{\textwidth}{@{}ll>{\itshape}lLl>{\small}l@{}}
\toprule
\textbf{\#} & \textbf{Language} & \multicolumn{1}{l}{\textbf{Term}} & \textbf{Gloss} & \textbf{Loan} & \multicolumn{1}{l}{\textbf{Source}} \\
\midrule
1	& English	& bastard cinnamon	& 	& yes	& \textcite{oed} \\
2	& English	& cassia	& 	& yes	& \textcite{oed} \\
3	& English	& cinnamon	& 	& yes	& \textcite{oed} \\
\midrule
1	& Arabic	& salīkha	& peel, strip	& no	& \textcite{wehr_dictionary_1976} \\
2	& Arabic	& dārṣīnī	& Chinese wood	& yes	& \textcite{wehr_dictionary_1976} \\
3	& Arabic	& qirfa	& bark, rind	& no	& \textcite{wehr_dictionary_1976} \\
\midrule
1	& Chinese	& guì	& cassia	& no	& \textcite{defrancis_abc_2003} \\
2	& Chinese	& guìpí	& cassia-skin	& no	& \textcite{defrancis_abc_2003} \\
3	& Chinese	& guìzhī	& cassia-branches	& no	& \textcite{defrancis_abc_2003} \\
4	& Chinese	& guìzǐ	& cassia-seeds	& no	& \textcite{defrancis_abc_2003} \\
5	& Chinese	& ròuguì	& flesh-cassia	& no	& \textcite{defrancis_abc_2003} \\
\bottomrule
\end{tabularx}
\caption{Conventionalized names for cinnamon in English, Arabic, and Chinese, found in dictionaries.}
\label{table:names_cinnamon}
\end{table}

