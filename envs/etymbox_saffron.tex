\begin{etymology}\label{ety:saffron}
English \textit{saffron}, ca. 1200; cf. Middle English saf(f)rǒun
< French \textit{safran} `id.', c. 1150; cf. Middle Low German safferân, Middle Dutch saffraen (Dutch saffraan), Middle High German saffrân (modern German safran)
< Medieval Latin \textit{saffrānum} `id.'
< Arabic {زعفران} \textit{zaʿfarān} `id.', (not connected with \textit{ṣafrā'} feminine of \textit{aṣfar} `yellow'); cf. Turkish, Persian, and Hindi; Jewish Aramaic zaʿperānā; Spanish azafran, Portuguese açafrão; the word without this prefix gives rise to Italian zafferano, zaffrone, Provençal safran, safrá, Catalan safrá, French safran, medieval Latin safranum, medieval Greek ζαϕρᾶς, modern Greek σαϕράνι, Russian šafran. \footnote{\textcite[s.v. saffron]{oed}; \textcite[saf(f)rǒun]{med}; \textcite[s.v. safran]{tlfi}; \textcite{wehr_dictionary_1976}}
\end{etymology}