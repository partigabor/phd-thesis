\begin{spice}\label{spice:anise}
\textsc{Anise} \hfill \href{https://powo.science.kew.org/taxon/846658-1}{POWO} \\
\textbf{English:} \textit{anise}; \textit{aniseed}. 
\textbf{Arabic:} {\arabicfont{أنيسون}} \textit{anīsūn}; {\arabicfont{{يانسون} \textit{yānsūn}}}. 
\textbf{Chinese:} {\tradchinesefont{茴芹}} \textit{huíqín} [anise-celery]. 
\textbf{Hungarian:} \textit{ánizs}.  \\
\noindent{\color{black}\rule[0.5ex]{\linewidth}{.5pt}}
\begin{tabular}{@{}p{0.25\linewidth}@{}p{0.75\linewidth}@{}}
Plant species: & \taxonn{Pimpinella anisum}{L.} \\
Family: & \textit{Apiaceae} \\
part used: & fruit; oil \\
Region of origin: & E. Mediterranean; W. Asia \\
Cultivated in: & Turkey; Egypt; Spain; Russia; Italy; etc. \\
Color: & light brown \\
\end{tabular}
\end{spice}