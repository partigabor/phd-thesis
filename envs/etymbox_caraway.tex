\begin{etymology}\label{ety:caraway}
English \textit{caraway}
< Medieval Latin \textit{carui} `id.', or some allied Romanic form; cf. cognates French carvi, Italian carvi, Spanish carvi (whence Scots carvy, kervie), Old Spanish alcaravea, alcarahueya, Portuguese alcaravia, alcorovia
< Arabic \textit{karawiyā} `id.', (loaned to some Eurropean languages with \textit{al-} definite article; via Andalusian Arabic)
< Aramaic {\he{כַרְוָיָא‎}/\sy{ܟܲܪܘܵܝܵܐ}} \textit{karwāyā} `id.'
< Ancient Greek {καρώ} \textit{karṓ} `id.', a form of the word \textit{káron}, derived from \textit{káre} `head'; -ṓ form seems Pre-Greek (these forms could not immediately give the Arabic, hence possibly via *καρυΐα \textit{*karuḯa} a typical plant derivation form of καρώ \textit{karṓ}, κάρον \textit{káron}); cf. cognates Latin \textit{carum, careum}\footnote{\textcite[s.v. caraway]{oed}; \textcite[s.v. caraway]{ahd}; \textcite[74]{corriente_dictionary_2008}; \textcites[207]{low_aramaeische_1881}[437-438]{low_flora_1924}; \textcites[653]{beekes_etymological_2010}[599]{sokoloff_dictionary_2002}}
\end{etymology}