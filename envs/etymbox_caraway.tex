\begin{etymology}\label{ety:caraway}
English \textit{caraway} `caraway', ca. 1440
< Medieval Latin \textit{carui} `caraway', or some allied Romanic form, ca. 1080; cf. cognates French \textit{carvi}, Italian \textit{carvi}, Spanish \textit{carvi}; Old Spanish \textit{alcaravea}, \textit{alcarahueya}, Portuguese \textit{alcaravia}, \textit{alcorovia}
< Arabic {كراويا} \textit{karāwiyā} `caraway', (loaned to some Eurropean languages with \textit{al-} definite article, via Andalusian Arabic)
< Aramaic {\he{כַרְוָיָא}/\sy{ܟܲܪܘܵܝܵܐ}} \textit{karwāyā} `caraway'
< Ancient Greek {καρώ} \textit{karṓ} `caraway', a form of the word \textit{káron}, derived from \textit{káre} `head'; -ṓ form seems Pre-Greek (these forms could not immediately give the Arabic, hence possibly via *καρυΐα \textit{*karuḯa} a typical plant derivation form of καρώ \textit{karṓ}, κάρον \textit{káron}); cf. cognates Latin \textit{carum, careum}\footnote{\textcite[s.v. caraway]{oed}; \textcite[s.v. caraway]{ahd}; \textcites[74]{corriente_dictionary_2008}[carvi]{tlfi}; \textcites[207]{low_aramaeische_1881}[437-438]{low_flora_1924}; \textcites[653]{beekes_etymological_2010}[599]{sokoloff_dictionary_2002}}
\end{etymology}