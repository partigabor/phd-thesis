\begin{etymology}\label{ety:anisun}
\textbf{Arabic} {أنيسون} \textit{anīsūn} `anise', (later assimilated as \ar{يانسون} \textit{yānsūn}), a. 791
< \textbf{Ancient Greek} {ἄνισον} \textit{ánison} `anise; dill', and other Greek dialectal variants, e.g.: \textit{ánēthon}; included both plants, only later distinguished (probaby of substrate origin)
<\textss{?} \textbf{Egyptian (Ancient)} \textit{jnst} `a medicinal, edible plant (probably anise)', ca. 2030-1650 BC\footnote{\textcite{wehr_dictionary_1976}; \textcite{liddell_greek-english_1940}; \textcites[99]{erman_worterbuch_1926}[240]{hemmerdinger_noms_1968}}
\end{etymology}