\begin{spice}\label{spice:allspice}
\textsc{Allspice} \hfill \href{https://powo.science.kew.org/taxon/196799-2}{POWO} \\
\textbf{English:} \textit{allspice}; \textit{pimento; Jamaica pepper}. 
\textbf{Arabic:} {\arabicfont{فلفل إفرنجي}} \textit{fulful ifranjī} [Frankish pepper]. 
\textbf{Chinese:} {\traditionalchinesefont{多香果}} \textit{duōxiāngguǒ} [many-spice-fruit]. 
\textbf{Hungarian:} \textit{szegfűbors} [clove-pepper]; \textit{jamaicaibors} [Jamaican-pepper]; \textit{amomummag} [amomum-seed].  \\
\noindent{\color{black}\rule[0.5ex]{\linewidth}{.5pt}}
\begin{tabular}{@{}p{0.25\linewidth}@{}p{0.75\linewidth}@{}}
Plant species: & \taxonn{Pimenta dioica}{(L.) Merr.} (syn. \taxonn{Pimenta officinalis}{Lindl.}) \\
Family: & \textit{Myrtaceae} \\
part used: & unripe fruit; leaf \\
Region of origin: & S. Mexico to C. America; Caribbean \\
Cultivated in: & Jamaica; Mexico; Honduras \\
Color: & dark brown \\
\end{tabular}
\end{spice}