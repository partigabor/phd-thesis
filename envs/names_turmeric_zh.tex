\begin{table}[!ht]
    \caption{Various names for turmeric in Chinese.}
\centering
\begin{tabularx}{\textwidth}{@{}l>{\itshape \small}ll>{\itshape}lL>{\small}l@{}}
\toprule
\textbf{\#} & \multicolumn{1}{l}{\textbf{Species}} & \multicolumn{1}{l}{\textbf{Name}} & \multicolumn{1}{l}{\textbf{Tr.}} & \multicolumn{1}{l}{\textbf{Gloss}} & \multicolumn{1}{l}{\textbf{Source}} \\
\midrule
1	& Curcuma longa	& \tc{黃薑}	& huángjiāng	& yellow-ginger	& \textcite{defrancis_abc_2003} \\
\textbf{2}	& \textbf{Curcuma longa}	& \textbf{\tc{薑黃}}	& \textbf{jiānghuáng}	& \textbf{ginger-yellow}	& \textbf{\textcite{kleeman_oxford_2010}} \\
3	& Curcuma longa	& \tc{鬱金}	& yùjīn	& yü-gold	& \textcite{schafer_golden_1985} \\
\bottomrule
\end{tabularx}
\label{table:names_turmeric_zh}
\end{table}

