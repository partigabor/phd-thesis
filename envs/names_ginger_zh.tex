\begin{table}[!ht]
    \caption{Various names for ginger in Chinese.}
\centering
\begin{tabularx}{\textwidth}{@{}l>{\itshape \small}ll>{\itshape}lL>{\small}l@{}}
\toprule
\textbf{\#} & \multicolumn{1}{l}{\textbf{Species}} & \multicolumn{1}{l}{\textbf{Name}} & \multicolumn{1}{l}{\textbf{Tr.}} & \multicolumn{1}{l}{\textbf{Gloss}} & \multicolumn{1}{l}{\textbf{Source}} \\
\midrule
1	& Zingiber officinale	& \tc{幹薑}	& gānjiāng	& dry-ginger	& \textcite{defrancis_abc_2003} \\
\textbf{2}	& \textbf{Zingiber officinale}	& \textbf{\tc{薑}}	& \textbf{jiāng}	& \textbf{ginger}	& \textbf{\textcite{kleeman_oxford_2010}} \\
3	& Zingiber officinale	& \tc{鮮薑}	& xiānjiāng	& fresh-ginger	& \textcite{defrancis_abc_2003} \\
\bottomrule
\end{tabularx}
\label{table:names_ginger_zh}
\end{table}

