\begin{table}[!ht]
\centering
\begin{tabularx}{\textwidth}{@{}ll>{\itshape}lLl>{\small}l@{}}
\toprule
\textbf{\#} & \textbf{Language} & \multicolumn{1}{l}{\textbf{Term}} & \textbf{Gloss} & \textbf{Loan} & \multicolumn{1}{l}{\textbf{Source}} \\
\midrule
1	& English	& black ginger	& 	& no	& \textcite{oed} \\
2	& English	& ginger	& 	& yes	& \textcite{oed} \\
3	& English	& ginger root	& 	& no	& \textcite{oed} \\
4	& English	& ginger spice	& 	& no	& \textcite{oed} \\
5	& English	& green ginger	& 	& no	& \textcite{oed} \\
6	& English	& white ginger	& 	& no	& \textcite{oed} \\
\midrule
1	& Arabic	& janzabīl	& 	& yes	& \textcite{wehr_dictionary_1976} \\
2	& Arabic	& zanjabīl	& 	& yes	& \textcite{wehr_dictionary_1976} \\
\midrule
1	& Chinese	& gānjiāng	& dry-ginger	& no	& \textcite{defrancis_abc_2003} \\
2	& Chinese	& jiāng	& ginger	& no	& \textcite{kleeman_oxford_2010} \\
3	& Chinese	& xiānjiāng	& fresh-ginger	& no	& \textcite{defrancis_abc_2003} \\
\bottomrule
\end{tabularx}
\caption{Conventionalized names for ginger in English, Arabic, and Chinese, found in dictionaries.}
\label{table:names_ginger}
\end{table}

