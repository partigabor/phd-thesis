\begin{etymology}\label{ety:doukou}
Mandarin Chinese {\tc{豆蔻}} \textit{dòukòu} \gls{MC} /dəuH həuH/ `cardamom' [bean-cardamom], compound of 豆 `bean(-like)' + \tc{蔻} `many; profusion' (BCGM); or phono-semantic matching (confused with nutmeg at first), ca. 863
<\textss{?} Middle Chinese {多骨} \textit{duōgǔ} \gls{MC} /tɑ kuət/ `round cardamom'
<\textss{?} Pali \textit{takkola} `Bdellium, a perfume made from the berry of the kakkola plant'
<\textss{?} Sanskrit {तक्कोल, कक्कोल} \textit{takkola, kakkola} `plant with aromatic berry; the perfume made from it'; cf. Pali \textit{takkola}; Tibetan \ti{ཀ་ཀོ་ལ} \textit{kakola}; Chinese 嘎哥拉 \textit{gágēlā}\footnote{\textcite[22]{donkin_between_2003}; \textcite[18:55]{yyzz}; \textcite[292]{pali_text_society_pali_1921}; \textcite[431, 241]{monier-williams_sanskrit-english_1899}}
\end{etymology}