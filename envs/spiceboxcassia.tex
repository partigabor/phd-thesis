\begin{spice}\label{spice:cassia}
\textsc{Cassia} \hfill \href{https://powo.science.kew.org/taxon/463288-1}{POWO} \\
\textbf{English:} \textit{cassia}. 
\textbf{Arabic:} {\arabicfont{سليخة}} \textit{salīkha} [peel; bark]. 
\textbf{Chinese:} {\tradchinesefont{肉桂}} \textit{ròuguì} [flesh-cinnamon]. 
\textbf{Hungarian:} \textit{kasszia(fahéj)} [cassia (tree-bark)].  \\
\noindent{\color{black}\rule[0.5ex]{\linewidth}{.5pt}}
\begin{tabular}{@{}p{0.25\linewidth}@{}p{0.75\linewidth}@{}}
Plant species: & \taxonn{Cinnamomum cassia}{(L.) J.Presl.} (syn. \taxonn{C. aromaticum}{Nees}); \textit{et al.} \\
Family: & \textit{Lauraceae} \\
Plant part used: & bark; fruit \\
Region of origin: & Southeast China \\
Cultivated in: & Indonesia; China; Vietnam; Timor-Leste; etc. \\
Color: & reddish brown \\
\end{tabular}
\end{spice}