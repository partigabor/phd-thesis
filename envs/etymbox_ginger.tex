\begin{etymology}\label{ety:ginger}
\textbf{English} \textit{ginger}, ca. 925
< reinforced by \textbf{Old French} \textit{gingivere, gingibre } `ginger'
< \textbf{Medieval Latin} \textit{gingiber} `ginger'
< \textbf{Latin} \textit{zingiber} `ginger'
< \textbf{Ancient Greek} {\gr{ζιγγίβερις}} \textit{ziggiberis} `ginger'
< \textbf{Pali} \textit{siṅgivera } `ginger'; cf. cognates Sanskrit शृङ्गवेर \text{śṛṇgavera}
< \textbf{Dravidian} \textit{*cinki-wēr} `ginger', South dravidian nominal compound  from the etyma of Tamil and Malayalam \textit{iñci} (both with regular loss of an initial sibilant) + \textit{vēr} (Proto-Dravidian \textit{wēr}); the base of \textit{*cinki} is a loanword
< \textbf{unknown} \textit{?} `ginger', unidentified Southeast Asian language; cf. cognates Khasi \textit{sying} /sʔiŋ/, Thai \textit{khing}, Vietnamese \textit{gừng}, Chinese \textit{jiāng}
<\textss{?} \textbf{Proto-Sino-Tibetan} \textit{*kjaŋ} `ginger'\footnote{\textcite{oed, ross_ginger_1952}; \textcite[5]{krishnamurti_dravidian_2003}; }
\end{etymology}