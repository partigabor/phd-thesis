\begin{etymology}\label{ety:ginger}
English \textit{ginger}, ca. 925
< Middle English \textit{gingivere}
< Old English \textit{ġinġifer}
< reinforced by Old French \textit{gingivere, gingibre }
< Late Latin \textit{gingiber}
< Latin \textit{zingiber}
< Ancient Greek {\gr{ζιγγίβερις}} \textit{ziggiberis}
< Pali \textit{siṅgivera }; cf. cognates Sanskrit शृङ्गवेर \text{śṛṇgavera}
< Dravidian \textit{*}, compound  from the etymon of Tamil and Malayalam \textit{iñci} (both with regular loss of an initial sibilant) + \textit{vēr}; the base of  \textit{iñci} is a loanword from a Southeast Asian language\footnote{\textcite{oed; ross_ginger_1952}; }
\end{etymology}