\begin{etymology}\label{ety:long_pepper}
English \textit{long pepper}, eOE; cf. cognates Anglo-Norman as poivre lonc (13th cent.; Middle French, French poivre long) and also Middle Dutch lanc peper (Dutch lange peper), Middle Low German lanc pēper, lancpēper, Old High German langpfeffar (Middle High German langer pheffer, German langer Pfeffer), Old Swedish langa pipar (Swedish långpeppar)
< Latin \textit{piper longus} `id.' [pepper-long]\footnote{\textcite[long pepper]{oed}}
\end{etymology}