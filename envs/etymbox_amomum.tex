\begin{etymology}\label{ety:amomum}
English \textit{amomum} `any of several spices of genus Amomum, family Zingiberaceae, including cardamom.', An odoriferous plant. The Amomum of the ancients not being certainly identified, the word was used with uncertain denotation by earlier writers;, a. 1398
< Latin \textit{amomum} `amomum and a balm containing this spice'
< Ancient Greek {ἄμωμον} \textit{ámōmon} `an Indian spice-plant, black cardamom (Amomum subulatum)', an Oriental loanword, cf. κιννάμωμον
< Semitic `id.'; cf. cognates Classical Syriac \sy{ܚܡܵܡܵܐ} \textit{ḥəmāmā} → Arabic \ar{حماما} \textit{ḥamāmā}; Akkadian \textit{ḫamīmu}
<
<\footnote{\textcite[s.v. amomum]{oed}; \textcite{lewis_latin_1879}; \textcites[]{liddell_greek-english_1940}[97]{beekes_etymological_2010}; \textcite[169]{low_aramaeische_1881}; \textcite[100]{lev_practical_2008}; \textcite[vol. 6, p. 66]{roth_assyrian_2004}}
\end{etymology}