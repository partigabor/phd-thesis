\begin{spice}\label{spice:vanilla}
\textsc{Vanilla} \hfill \href{https://powo.science.kew.org/taxon/262578-2}{POWO} \\
\textbf{English:} \textit{vanilla}. 
\textbf{Arabic:} {\arabicfont{فانيليا}} \textit{fānīliyā}. 
\textbf{Chinese:} {\tradchinesefont{香草}} \textit{xiāngcǎo} [fragrant-herb]; Cantonese 雲呢拿 \textit{wan4 nei1 laa4-2}. 
\textbf{Hungarian:} \textit{vanília}.  \\
\noindent{\color{black}\rule[0.5ex]{\linewidth}{.5pt}}
\begin{tabular}{@{}p{0.25\linewidth}@{}p{0.75\linewidth}@{}}
Plant species: & \taxonn{Vanilla planifolia}{Jacks. ex Andrews} (syn. \taxonn{Vanilla fragrans}{Ames}); \textit{\taxonn{V. tahitensis}{J.W. Moore}; \taxonn{V. pompona}{Schiede}} \\
Family: & \textit{Orchidaceae} \\
part used: & fruit \\
Region of origin: & Tropical America \\
Cultivated in: & Madagascar; Indonesia; Mexico; Papua New Guinea; China \\
Color: & dark brown pod; creamy white extract \\
\end{tabular}
\end{spice}