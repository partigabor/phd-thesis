\begin{table}[!ht]
    \caption{Various names for cassia in Arabic.}
\centering
\begin{tabularx}{\textwidth}{@{}l>{\itshape \small}lr>{\itshape}lL>{\small}l@{}}
\toprule
\textbf{\#} & \multicolumn{1}{l}{\textbf{Species}} & \multicolumn{1}{l}{\textbf{Name}} & \multicolumn{1}{l}{\textbf{Tr.}} & \multicolumn{1}{l}{\textbf{Gloss}} & \multicolumn{1}{l}{\textbf{Source}} \\
\midrule
1	& Cinnamomum cassia	& دارصيني الدون	& dārṣīnī al-dūn	& inferior cinnamon	&  \\
2	& Cinnamomum cassia	& قرفة صينية	& qirfa ṣīnīyya 	& Chinese bark	& \textcite{wikipedia} \\
\textbf{3}	& \textbf{Cinnamomum cassia}	& \textbf{سليخة}	& \textbf{salīkha}	& \textbf{peel, strip}	& \textbf{\textcite{wehr_dictionary_1976}} \\
4	& Cinnamomum spp.	& الحاد المذاق	& al-ḥādd al-madhāq	& the sharp taste	& \textcite{dietrich_dar_2004} \\
5	& Cinnamomum spp.	& دارصيني الصين	& dārṣīnī al-ṣīn	& Chinese wood of China	& \textcite{dietrich_dar_2004} \\
\bottomrule
\end{tabularx}
\label{table:names_cassia_ar}
\end{table}

