\begin{spice}\label{spice:saffron}
\textsc{Saffron} \hfill \href{https://powo.science.kew.org/taxon/436688-1}{POWO} \\
\textbf{English:} \textit{saffron}. 
\textbf{Arabic:} {\arabicfont{زعفران}} \textit{zaʿfarān}. 
\textbf{Chinese:} {\tradchinesefont{藏紅花}} \textit{zànghónghuā} [Tibetan-red-flower]; 西紅花 xīhónghuā sai1 hung4 faa1 . 
\textbf{Hungarian:} \textit{sáfrány}.  \\
\noindent{\color{black}\rule[0.5ex]{\linewidth}{.5pt}}
\begin{tabular}{@{}p{0.25\linewidth}@{}p{0.75\linewidth}@{}}
Plant species: & \taxonn{Crocus sativus}{L.} \\
Family: & \textit{Iridaceae} \\
Plant part used: & stigma (style) \\
Region of origin: & Greece \\
Cultivated in: & Iran; Spain; Kashmir; etc. \\
Color: & deep red; dyes in orange \\
\end{tabular}
\end{spice}