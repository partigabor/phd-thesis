\begin{table}[!ht]
    \caption{Various names for coriander in Chinese.}
\centering
\begin{tabularx}{\textwidth}{@{}l>{\itshape \small}ll>{\itshape}lL>{\small}l@{}}
\toprule
\textbf{\#} & \multicolumn{1}{l}{\textbf{Species}} & \multicolumn{1}{l}{\textbf{Name}} & \multicolumn{1}{l}{\textbf{Tr.}} & \multicolumn{1}{l}{\textbf{Gloss}} & \multicolumn{1}{l}{\textbf{Source}} \\
\midrule
1	& Coriandrum sativum	& \tc{胡荽}	& húsuī	& barbarian-coriander	& \textcite{laufer_sino-iranica_1919} \\
2	& Coriandrum sativum	& \tc{香菜}	& xiāngcài	& fragrant-vegetable	& \textcite{hu_food_2005} \\
3	& Coriandrum sativum	& \tc{香茜}	& xiāngqiàn	& fragrant-madder	& \textcite{wikipedia} \\
4	& Coriandrum sativum	& \tc{芫茜}	& yánqiàn	& lilac-madder	& \textcite{wikipedia} \\
\textbf{5}	& \textbf{Coriandrum sativum}	& \textbf{\tc{芫荽}}	& \textbf{yánsuī}	& \textbf{lilac-coriander}	& \textbf{\textcite{hu_food_2005}} \\
\bottomrule
\end{tabularx}
\label{table:names_coriander_zh}
\end{table}

