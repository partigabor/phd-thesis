\begin{spice}\label{spice:nutmeg}
\textsc{Nutmeg} \hfill \href{https://powo.science.kew.org/taxon/586076-1}{POWO} \\
\textbf{English:} \textit{nutmeg}. 
\textbf{Arabic:} {\arabicfont{جوز الطيب}} \textit{jawz al-ṭīb} [fragrant nut]. 
\textbf{Chinese:} {\tradchinesefont{肉豆蔻}} \textit{ròudòukòu} [flesh-bean-cardamom]; juk6 dau6 dau2 kau3 . 
\textbf{Hungarian:} \textit{szerecsendió} [Saracen nut]; \textit{muskátdió} [musk-nut]; \textit{mácisdió} [mace-nut].  \\
\noindent{\color{black}\rule[0.5ex]{\linewidth}{.5pt}}
\begin{tabular}{@{}p{0.25\linewidth}@{}p{0.75\linewidth}@{}}
Plant species: & \taxonn{Myristica fragrans}{Houtt.} \\
Family: & \textit{Myristicaceae} \\
Plant part used: & seed \\
Region of origin: & Moluccas (Indonesia) \\
Cultivated in: & Grenada, Indonesia \\
Color: & pale brown nut, dark when powdered \\
\end{tabular}
\end{spice}