\begin{etymology}\label{ety:qaqulla}
Arabic {قاقلة} \textit{qāqulla} `cardamom; black cardamom'
< Classical Syriac {{קָקוּלָא}/\sy{ܩܳܩܘܽܠܴܐ‎}} \textit{qāqullā} `cardamom'
< Akkadian {\cu{𒋡𒄣𒌌𒇻𒊬} (qa-qu-ul-lu.SAR)} \textit{qāqullu} `cardamom'
<\textss{?} Sanskrit {तक्कोल, कक्कोल} \textit{takkola, kakkola} `plant with aromatic berry; the perfume made from it'; cf. Pali \textit{takkola}; Tibetan \ti{ཀ་ཀོ་ལ} \textit{kakola}\footnote{\textcite[863]{wehr_dictionary_1976}; \textcite[Vol. 1, p. 489]{low_flora_1924}; \textcite[58]{zimmern_akkadische_1915}; \textcite[431, 241]{monier-williams_sanskrit-english_1899}}
\end{etymology}