\begin{spice}\label{spice:mace}
\textsc{Mace} \hfill \href{https://powo.science.kew.org/taxon/586076-1}{POWO} \\
\textbf{English:} \textit{mace}. 
\textbf{Arabic:} {\arabicfont{بسباسة}} \textit{basbāsa}; {\arabicfont{{قشرة جوز الطيب} \textit{qishrat jawz al-ṭīb} [the peel of the fragrant nut]}}. 
\textbf{Chinese:} {\tradchinesefont{肉豆蔻皮}} \textit{ròudòukòupí} [flesh-bean-cardamom-skin]. 
\textbf{Hungarian:} \textit{szerecsendió-virág} [Saracen nut flower].  \\
\noindent{\color{black}\rule[0.5ex]{\linewidth}{.5pt}}
\begin{tabular}{@{}p{0.25\linewidth}@{}p{0.75\linewidth}@{}}
Plant species: & \taxonn{Myristica fragrans}{Houtt.} \\
Family: & \textit{Myristicaceae} \\
part used: & aril \\
Region of origin: & Moluccas (Indonesia) \\
Cultivated in: & Grenada; Indonesia \\
Color: & crimson red aril whn fresh, pale yellow when dried \\
\end{tabular}
\end{spice}