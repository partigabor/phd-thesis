\begin{etymology}\label{ety:cassia}
English \textit{cassia}, ca. 1000
< Middle English \textit{cassia}
< Old English \textit{cassia}
< Latin \textit{casia} `id.', \nth{1} c. \AD{}
< Ancient Greek {κασία} \textit{kasía} `id.', \nth{6} c. \BC{}
< Ancient Hebrew {קְצִיעָה} \textit{qəṣîʿâ} `a bark resembling cinnamon, but less aromatic, so called from being stripped off', from \textit{qṣaʿ} `to cut off, strip off bark' (hapax legomenon in the Bible; Ps. 45:9)\footnote{\textcite[s.v. cassia]{oed}; \textcite{rosol_early_2018}; \textcite[653]{beekes_etymological_2010}; \textcite[589]{klein_comprehensive_1987}}
\end{etymology}