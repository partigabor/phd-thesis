\begin{etymology}\label{ety:zafaran}
Arabic {زعفران} \textit{zaʿfarān} `saffron', (not connected with ṣafrā' feminine of aṣfar yellow); cf. Turkish, Persian, and Hindi; Jewish Aramaic zaʿperānā; Spanish azafran, Portuguese açafrão; the word without this prefix gives rise to Italian zafferano, zaffrone, Provençal safran, safrá, Catalan safrá, French safran, medieval Latin safranum, medieval Greek ζαϕρᾶς, modern Greek σαϕράνι, Russian šafran. 
<\textss{?} Pahlavi \textit{zarparān} `saffron' [golden thread], \textit{zar} `gold' + \textit{par} `feather' + \textit{-ān} `pl.', a pseudo-etymological explanation
<\textss{?} Akkadian {\cu{𒌑𒄯𒊕}} \textit{azupīru, azupīrānu} `a spice and medicinal plant', (unlikely etymon)\footnote{\textcite{wehr_dictionary_1976}; \textcites[]{asbaghi_persische_1988}[65, 98]{mackenzie_concise_1986}[safran]{ns}; \textcites[33]{black_concise_2000}[vol. 2, 530-531]{roth_assyrian_2004}}
\end{etymology}