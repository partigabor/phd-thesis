\begin{table}[!ht]
\centering
\begin{tabularx}{\textwidth}{@{}ll>{\itshape}lLl>{\small}l@{}}
\toprule
\textbf{\#} & \textbf{Language} & \multicolumn{1}{l}{\textbf{Term}} & \textbf{Gloss} & \textbf{Loan} & \multicolumn{1}{l}{\textbf{Source}} \\
\midrule
1	& English	& anise	& 	& yes	& \textcite{oed} \\
2	& English	& aniseed	& 	& no	& \textcite{oed} \\
3	& English	& sweet cumin	& 	& no	& \textcite{oed} \\
\midrule
1	& Arabic	& anīsūn	& phonetic	& yes	& \textcite{wehr_dictionary_1976} \\
2	& Arabic	& kammūn ḥulw	& sweet cumin	& no	& \textcite{wehr_dictionary_1976} \\
3	& Arabic	& yānisūn	& phonetic	& yes	& \textcite{wehr_dictionary_1976} \\
4	& Arabic	& ḥabba ḥulwa	& sweet grain, seed	& no	& \textcite{wehr_dictionary_1976} \\
\midrule
1	& Chinese	& huíqín	& hui-celery	& no	& \textcite{kleeman_oxford_2010} \\
2	& Chinese	& huíxiāng	& hui-spice	& no	& \textcite{kleeman_oxford_2010} \\
\bottomrule
\end{tabularx}
\caption{Conventionalized names for anise in English, Arabic, and Chinese, found in dictionaries.}
\label{table:names_anise}
\end{table}

