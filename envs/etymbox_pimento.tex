\begin{etymology}\label{ety:pimento}
English \textit{pimento} `allspice; sweet pepper', ca. 1660
< partly Portuguese \textit{pimenta} `allspice; sweet pepper; black pepper'
< and partly Spanish \textit{pimiento} `hot and sweet pepper; formerly also black pepper; pepper plant of both kinds', earlier \textit{pimienta} `black pepper; peppercorn; ground pepper' \nth{13} c., 1495
< Medieval Latin \textit{pigmenta} `plant juice; food seasoning; condiment; spices; perfumes', plural of \textit{pigmentum}
< Latin \textit{pigmentum} `colour, paint; ointment; drug; spiced wine', from \textit{pingō} `to paint' + \textit{-mentum} `instrument'\footnote{\textcite[s.v. pimento]{oed}; \textcite[s.v. pimento]{oed}; \textcite[s.v. pimiento]{oed}; \textcites[415]{gomez_de_silva_elseviers_1985}[495]{corominas_breve_1987}; \textcite[s.v. pigmentum]{lewis_latin_1879}}
\end{etymology}