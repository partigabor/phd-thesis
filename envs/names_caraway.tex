\begin{table}[!ht]
\centering
\begin{tabularx}{\textwidth}{@{}ll>{\itshape}lLl>{\small}l@{}}
\toprule
\textbf{\#} & \textbf{Language} & \multicolumn{1}{l}{\textbf{Term}} & \textbf{Gloss} & \textbf{Loan} & \multicolumn{1}{l}{\textbf{Source}} \\
\midrule
1	& English	& Armenian cumin	& 	& no	& \textcite{oed} \\
2	& English	& caraway	& 	& yes	& \textcite{oed} \\
3	& English	& caraway-seed	& 	& no	& \textcite{oed} \\
4	& English	& mountain cumin	& 	& no	& \textcite{oed} \\
5	& English	& royal cumin	& 	& no	& \textcite{oed} \\
\midrule
1	& Arabic	& karāwiyā	& phonetic	& yes	& \textcite{wehr_dictionary_1976} \\
\midrule
1	& Chinese	& gě​lǚ​zi	& phonetic?	& yes	& \textcite{kleeman_oxford_2010} \\
2	& Chinese	& yèhāo	& leaf-wormwood	& 	& \textcite{mdbg} \\
3	& Chinese	& zànghuíxiāng	& Tibetan-hui-spice	& no	& \textcite{mdbg} \\
\bottomrule
\end{tabularx}
\caption{Conventionalized names for caraway in English, Arabic, and Chinese, found in dictionaries.}
\label{table:names_caraway}
\end{table}

