\begin{etymology}\label{ety:ziran}
\textbf{Mandarin Chinese} \tc{孜然} \textit{zī​rán} `cumin', modern loan from Uyghur (the historic term is \tc{蒔蘿} from Middle Persian \textit{*zīra} during Tang dynasty)
< \textbf{Uyghur} {زىرە} \textit{zirä} `cumin'
< \textbf{Persian} {زیره} \textit{zīra} `cumin', distantly related to Sanskrit \textit{jīraka} (zire-ye siyāh [black cumin] `caraway'; zire-ye sabz [green cumin] `cumin'); cf. cognates Sogdian zyr'kk /zîrê/; Hindi-Urdu \textit{zīrā}
<\textss{?} \textbf{Sanskrit} {जीर} \textit{jīra} `cumin'; cf. Hindi जीरा \textit{jīrā}; English \textit{jeera}\footnote{\textcites[383]{laufer_sino-iranica_1919}[45]{sulaiman_uyghur_2020}[]{liu_hanyu_1985}; \textcite[561]{schwarz_uyghur-english_1992}; \textcite[634]{steingass_comprehensive_1892}; \textcites[375]{mcgregor_oxford_1993}}
\end{etymology}