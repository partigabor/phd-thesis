\begin{etymology}\label{ety:ziran}
Mandarin Chinese {孜然} \textit{zī​rán} `cumin', modern loan from Uyghur, the historic term is 蒔蘿 from Middle Persian \textit{*zīra} during Tang dynasty
< Uyghur {زىرە} \textit{zire} `cumin'
< Persian {زیره} \textit{zire} `id.', distantly related to Sanskrit \textit{jīraka} (zire-ye siyāh [black cumin] `caraway'; zire-ye sabz [green cumin] `cumin'); cf. cognates Sogdian zyr'kk /zîrê/; Hindi-Urdu \textit{zīrā}
<\textss{?} Sanskrit {जीर} \textit{jīra}; cf. Hindi जीरा \textit{jīrā}; English \textit{jeera}\footnote{\textcite[383]{laufer_sino-iranica_1919}; }
\end{etymology}