\begin{etymology}\label{ety:darsini}
Arabic {دارصيني} \textit{dārṣīnī} `cinnamon'
< Persian {دارچین} \textit{dārchīnī} `id.' [Chinese wood], from Persian \textit{dār} `wood' + \textit{cīn} `China'; cf. cognates Sanskrit \textit{dāru} (PIE *dóru)
< Pahlavi {*dār ī čēnīg} \textit{*dār ī čēnīg} `id.', (cf. Armenian \textit{daričenik})\footnote{\textcite[311]{wehr_dictionary_1976}; \textcite{dietrich_dar_1960}; \textcite{alam_darcini_2011}}
\end{etymology}