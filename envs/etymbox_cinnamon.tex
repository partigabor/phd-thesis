\begin{etymology}\label{ety:cinnamon}
English \textit{cinnamon} `cinnamon', (Middle English \textit{sinamome, synamome}), ca. 1430
< French \textit{cinnamome} `cinnamon', (earlier \textit{cynnamome}; also \nth{16} c. \textit{cinamonde}), 1211
< Latin \textit{cinnamōmum} `cinnamon', \nth{1} c. \AD{}
< Ancient Greek {κιννάμωμον} \textit{kinnámōmon} `cinnamon', later refashioned as \textit{kínnamon} after Latin \textit{cinnamum/cinnamon}, which partly influenced the current English form (of Semitic origin), \nth{5} c. \BC{}; cf. cognates Coptic \cop{ⲕⲓⲛⲛⲁⲙⲱⲙⲟⲛ} \textit{kinnamomon}
< Semitic* \textit{*qnmwn} `cinnamon'; cf. cognates Ancient Hebrew \he{קִנָּמוֹן} \textit{qināmōn}; Judeo-Aramaic \sy{ܩܢܡܘ} \textit{qnmw}
< unknown\footnote{\textcite[cinnamon]{oed}; \textcite{tlfi}; \textcite{lewis_latin_1879}; \textcite[701]{beekes_etymological_2010}; \textcite[585]{klein_comprehensive_1987}; \textcite{rosol_early_2018}}
\end{etymology}