\begin{spice}\label{spice:turmeric}
\textsc{Turmeric} \hfill \href{https://powo.science.kew.org/taxon/796451-1}{POWO} \\
\textbf{English:} \textit{turmeric}. 
\textbf{Arabic:} {\arabicfont{كركم}} \textit{kurkum}. 
\textbf{Chinese:} {\tradchinesefont{薑黃}} \textit{jiānghuáng} [ginger-yellow]; 黃薑 \textit{huángjiāng} [yellow-ginger]. 
\textbf{Hungarian:} \textit{kurkuma}.  \\
\noindent{\color{black}\rule[0.5ex]{\linewidth}{.5pt}}
\begin{tabular}{@{}p{0.25\linewidth}@{}p{0.75\linewidth}@{}}
Plant species: & \taxonn{Curcuma longa}{L.} (syn. \taxonn{C. domestica}{Valeton}) \\
Family: & \textit{Zingiberaceae} \\
Plant part used: & rhizome \\
Region of origin: & India \\
Cultivated in: & China, Honduras, India, Indonesia, Jamaica \\
Color: & orange-yellow \\
\end{tabular}
\end{spice}