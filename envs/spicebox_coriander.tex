\begin{spice}\label{spice:coriander}
\textsc{Coriander} \hfill \href{https://powo.science.kew.org/taxon/840760-1}{POWO} \\
\textbf{English:} \textit{coriander}; \textit{cilantro; Chinese parsley}. 
\textbf{Arabic:} {\arabicfont{كزبرة}} \textit{kuzbara}. 
\textbf{Chinese:} {\tradchinesefont{芫荽}} \textit{yán​sui} [lilac-coriander]. 
\textbf{Hungarian:} \textit{koriander}; \textit{cigánypetrezselyem} [gipsy-parsley].  \\
\noindent{\color{black}\rule[0.5ex]{\linewidth}{.5pt}}
\begin{tabular}{@{}p{0.25\linewidth}@{}p{0.75\linewidth}@{}}
Plant species: & \taxonn{Coriandrum sativum}{L.} \\
Family: & \textit{Apiaceae/Umbelliferae} \\
Plant part used: & fruit; leaf \\
Region of origin: & Mediterranean; W. Asia; India \\
Cultivated in: & Argentina, India, Morocco, Romania, Spain, Yugoslavia \\
Color: & light yellow \\
\end{tabular}
\end{spice}