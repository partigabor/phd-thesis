\begin{etymology}\label{ety:mace}
\textbf{English} \textit{mace} `aril surrounding the nutmeg', taken as a plural in Middle English (\textit{macis}) and a new singular mace was formed from it, 1234
< \textbf{Old French} \textit{macis} `id.', (although only attested slightly later than in Middle English)
< or \textbf{Medieval Latin} \textit{macis} `id.'; cf. cognates Hungarian \textit{mácisdió}
<\textss{?} \textbf{Latin} \textit{macir}
<\textss{?} \textbf{Ancient Greek} {μάκιρ} \textit{mákir}
<\textss{?} \textbf{unknown} \textit{?}\footnote{; }
\end{etymology}