\begin{spice}\label{spice:fennel}
\textsc{Fennel} \hfill \href{https://powo.science.kew.org/taxon/842680-1}{POWO} \\
\textbf{English:} \textit{fennel}. 
\textbf{Arabic:} {\arabicfont{شمر}} \textit{shamar}. 
\textbf{Chinese:} {\tradchinesefont{茴香}} \textit{huíxiāng} [hui-spice]. 
\textbf{Hungarian:} \textit{édeskömény} [sweet-cumin]; \textit{ánizskapor} [anise-dill].  \\
\noindent{\color{black}\rule[0.5ex]{\linewidth}{.5pt}}
\begin{tabular}{@{}p{0.25\linewidth}@{}p{0.75\linewidth}@{}}
Plant species: & \taxonn{Foeniculum vulgare}{Mill.} \\
Family: & \textit{Apiaceae/Umbelliferae} \\
Plant part used: & fruit; leaf \\
Region of origin: & Mediterranean; W. Asia; India \\
Cultivated in: & Argentina, Bulgaria, Germany, Greece, India, Lebanon \\
Color: & light green to light brown \\
\end{tabular}
\end{spice}