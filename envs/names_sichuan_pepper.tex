\begin{table}[!ht]
\centering
\begin{tabularx}{\textwidth}{@{}ll>{\itshape}lLl>{\small}l@{}}
\toprule
\textbf{\#} & \textbf{Language} & \multicolumn{1}{l}{\textbf{Term}} & \textbf{Gloss} & \textbf{Loan} & \multicolumn{1}{l}{\textbf{Source}} \\
\midrule
1	& English	& Chinese pepper	& 	& no	& \textcite{oed} \\
2	& English	& Japanese pepper	& 	& no	& \textcite{oed} \\
3	& English	& prickly ash	& 	& no	& \textcite{oed} \\
\midrule
\midrule
1	& Chinese	& huā​jiāo	& flower-pepper	& no	& \textcite{defrancis_abc_2003} \\
2	& Chinese	& jiāo	& pepper	& no	& \textcite{defrancis_abc_2003} \\
\bottomrule
\end{tabularx}
\caption{Conventionalized names for Sichuan pepper in English, Arabic, and Chinese, found in dictionaries.}
\label{table:names_Sichuan_pepper}
\end{table}

