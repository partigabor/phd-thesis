\begin{etymology}\label{ety:anise}
English \textit{anise}, ca. 1325
< French \textit{anis} `anise', 1236
< Latin \textit{anīsum} `anise', (dill is \textit{anēthum})
< Ancient Greek {ἄνισον} \textit{ánison} `anise; dill', and other Greek dialectal variants, e.g.: \textit{ánēthon}; included both plants, only later distinguished (probaby of substrate origin)
<\textss{?} Egyptian (Ancient) \textit{jnst} `a medicinal, edible plant (probably anise)'\footnote{\textcite[s.v. anise]{[anise]{oed}[anise]{ahd}}; \textcite[s.v. anis]{tlfi}; \textcite{lewis_latin_1879}; \textcite{liddell_greek-english_1940}; \textcites[99]{erman_worterbuch_1926}[240]{hemmerdinger_noms_1968}}
\end{etymology}