\begin{etymology}\label{ety:huiqin}
Mandarin Chinese {茴芹} \textit{huíqín} `anise' [hui-celery], from \textit{hui} `anise/fennel' + \textit{qin} `celery' (茴 \textit{huí} could be interpreted as `Muslim spice', see 茴香 \textit{huíxiāng} `fennel'), 1841\footnote{\textcite{kleeman_oxford_2010; hu_food_2005}}
\end{etymology}