\begin{table}[!ht]
\centering
\begin{tabularx}{\textwidth}{@{}ll>{\itshape}lLl>{\small}l@{}}
\toprule
\textbf{\#} & \textbf{Language} & \multicolumn{1}{l}{\textbf{Term}} & \textbf{Gloss} & \textbf{Loan} & \multicolumn{1}{l}{\textbf{Source}} \\
\midrule
1	& English	& long pepper	& 	& yes	& \textcite{oed} \\
\midrule
1	& Arabic	& dārfilfil	& 	& yes	& \textcite{wehr_dictionary_1976} \\
\midrule
1	& Chinese	& bìbá	& phonetic	& yes	& \textcite{defrancis_abc_2003} \\
2	& Chinese	& bìbō	& phonetic	& yes	& \textcite{hu_food_2005} \\
\bottomrule
\end{tabularx}
\caption{Conventionalized names for long pepper in English, Arabic, and Chinese, found in dictionaries.}
\label{table:names_long_pepper}
\end{table}

