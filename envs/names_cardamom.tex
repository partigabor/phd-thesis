\begin{table}[!ht]
\centering
\begin{tabularx}{\textwidth}{@{}ll>{\itshape}lLl>{\small}l@{}}
\toprule
\textbf{\#} & \textbf{Language} & \multicolumn{1}{l}{\textbf{Term}} & \textbf{Gloss} & \textbf{Loan} & \multicolumn{1}{l}{\textbf{Source}} \\
\midrule
1	& English	& cardamom	& 	& yes	& \textcite{oed} \\
\midrule
1	& Arabic	& hāl	& phonetic	& yes	& \textcite{wehr_dictionary_1976} \\
2	& Arabic	& khayr buwwā'	& good-scent	& yes	& \textcite{lane_arabic-english_1863} \\
3	& Arabic	& ḥabb al-hāl	& cardamom-seed	& 	& \textcite{baalbaki_-mawrid_1995} \\
4	& Arabic	& ḥabb al-hān, ḥabhān	& cardamom-seed	& 	& \textcite{wehr_dictionary_1976} \\
\midrule
1	& Chinese	& xiǎodòukòu	& little cardamom	& no	& \textcite{defrancis_abc_2003} \\
\bottomrule
\end{tabularx}
\caption{Conventionalized names for cardamom in English, Arabic, and Chinese, found in dictionaries.}
\label{table:names_cardamom}
\end{table}

