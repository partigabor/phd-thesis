\begin{etymology}\label{ety:cardamom}
English \textit{cardamom}, (via post-classical Latin \textit{cardimomum}, a. 1398), ?ca. 1425
< later also from Old French \textit{cardemome} `id.', ca. 1170; cf. modern French \textit{cardamome}
< Latin \textit{cardamōmum} `id.', \nth{1} c. \AD{}
< Hellenistic Greek {καρδάμωμον} \textit{kardámōmon} `id.', haplological \gr{κάρδαμ-} \textit{kárdam-} `cress' + \gr{ἄμωμον} \textit{ámōmon} `an Indian spice plant', \nth{3} c. \BC{}
< Ancient Greek {κάρδαμον} \textit{kárdamon} `garden cress \taxon{Lepidium sativum}', \nth{4} c. \BC{}; cf. cognates classical Latin \textit{cardamum}
<\textss{?} \textit{unknown}, prehaps a loanword\footnote{\textcite[s.v. cardamom]{oed}; \textcite[s.v. cardamome]{tlfi}; \textcite[s.v. cardamomum]{lewis_latin_1879}; \textcite[s.v. καρδάμωμον]{liddell_greek-english_1940}; \textcite[s.v. κάρδαμον]{liddell_greek-english_1940}; \textcite[644]{beekes_etymological_2010}}
\end{etymology}