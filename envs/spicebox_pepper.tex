\begin{spice}\label{spice:pepper}
\textsc{Pepper} \hfill \href{https://powo.science.kew.org/taxon/682369-1}{POWO} \\
\textbf{English:} \textit{pepper}; \textit{black pepper}. 
\textbf{Arabic:} {\arabicfont{فلفل}} \textit{filfil, fulful}; {\arabicfont{{فلفل أسود} \textit{fulful aswad} [black pepper]}}. 
\textbf{Chinese:} {\tradchinesefont{胡椒}} \textit{hújiāo} [barbarian-pepper]; {\tradchinesefont{黑胡椒 \textit{hēihújiāo} [black-barbarian-pepper]}}. 
\textbf{Hungarian:} \textit{bors} [pepper]; \textit{fekete bors} [black pepper].  \\
\noindent{\color{black}\rule[0.5ex]{\linewidth}{.5pt}}
\begin{tabular}{@{}p{0.25\linewidth}@{}p{0.75\linewidth}@{}}
Plant species: & \taxonn{Piper nigrum}{L.} \\
Family: & \textit{Piperaceae} \\
part used: & fruit \\
Region of origin: & Malabar coast (South India) \\
Cultivated in: & Vietnam; Brazil; Indonesia; India; Sri Lanka; etc. \\
Color: & black; white; green \\
\end{tabular}
\end{spice}