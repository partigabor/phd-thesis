\begin{spice}\label{spice:long pepper}
\textsc{Long pepper} \hfill \href{https://powo.science.kew.org/taxon/682031-1}{POWO} \\
\textbf{English:} \textit{long pepper}. 
\textbf{Arabic:} {\arabicfont{دارفلفل}} \textit{dārfilfil}. 
\textbf{Chinese:} {\traditionalchinesefont{蓽撥}} \textit{bìbō}. 
\textbf{Hungarian:} \textit{hosszú bors} [long-pepper].  \\
\noindent{\color{black}\rule[0.5ex]{\linewidth}{.5pt}}
\begin{tabular}{@{}p{0.25\linewidth}@{}p{0.75\linewidth}@{}}
Plant species: & \taxonn{Piper longum}{L.}; \textit{\taxonn{P. retrofactum}{Vahl}} \\
Family: & \textit{Piperaceae} \\
part used: & fruit \\
Region of origin: & E. Himalaya to S. China; Indo-China \\
Cultivated in: & India; Indonesia; Thailand \\
Color: & dreen to red when ripe, dark brown when dried \\
\end{tabular}
\end{spice}