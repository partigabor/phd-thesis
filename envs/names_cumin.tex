\begin{table}[!ht]
\centering
\begin{tabularx}{\textwidth}{@{}ll>{\itshape}lLl>{\small}l@{}}
\toprule
\textbf{\#} & \textbf{Language} & \multicolumn{1}{l}{\textbf{Term}} & \textbf{Gloss} & \textbf{Loan} & \multicolumn{1}{l}{\textbf{Source}} \\
\midrule
1	& English	& cumin	& 	& yes	& \textcite{oed} \\
2	& English	& cumin seed	& 	& no	& \textcite{oed} \\
\midrule
1	& Arabic	& kammūn	& 	& yes	& \textcite{wehr_dictionary_1976} \\
\midrule
1	& Chinese	& huíxiāngzǐ	& hui-spice-seed	& no	& \textcite{mdbg} \\
2	& Chinese	& kūmíng	& phonetic	& yes	& \textcite{mdbg} \\
3	& Chinese	& shíluó	& phonetic	& yes	& \textcite{kleeman_oxford_2010} \\
4	& Chinese	& zīrán	& phonetic	& yes	& \textcite{mdbg} \\
5	& Chinese	& zī​ránqín	& cumin-celery	& no	& \textcite{mdbg} \\
6	& Chinese	& ālābó huíxiāng	& Arabian fennel	& no	& \textcite{mdbg} \\
7	& Chinese	& ānxī huíxiāng	& Parthian fennel	& no	& \textcite{mdbg} \\
8	& Chinese	& Ōu​shí​luó	& European dill	& 	& \textcite{mdbg} \\
\bottomrule
\end{tabularx}
\caption{Conventionalized names for cumin in English, Arabic, and Chinese, found in dictionaries.}
\label{table:names_cumin}
\end{table}

