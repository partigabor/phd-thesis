\begin{table}[!ht]
\centering
\begin{tabularx}{\textwidth}{@{}ll>{\itshape}lLl>{\small}l@{}}
\toprule
\textbf{\#} & \textbf{Language} & \multicolumn{1}{l}{\textbf{Term}} & \textbf{Gloss} & \textbf{Loan} & \multicolumn{1}{l}{\textbf{Source}} \\
\midrule
1	& English	& saffron	& 	& yes	& \textcite{oed} \\
\midrule
1	& Arabic	& jādī	& 	& no	& \textcite{baalbaki_-mawrid_1995} \\
2	& Arabic	& zaʿfarān	& saffron	& yes	& \textcite{wehr_dictionary_1976} \\
3	& Arabic	& ḥuṣṣ	& 	& no	& \textcite{wehr_dictionary_1976} \\
\midrule
1	& Chinese	& fānhónghuā	& foreign-red-flower	& no	& \textcite{defrancis_abc_2003} \\
2	& Chinese	& zànghónghuā	& Tibetan-red-flower	& no	& \textcite{kleeman_oxford_2010} \\
\bottomrule
\end{tabularx}
\caption{Conventionalized names for saffron in English, Arabic, and Chinese, found in dictionaries.}
\label{table:names_saffron}
\end{table}

